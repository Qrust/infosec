\documentclass[10pt,a4paper]{book}
%\documentclass[12pt,report,russian]{ncc}
%\usepackage{a4wide}
% Для векторых русских шрифтов в PDF не забудьте установить пакеты cm-super & cm-unicode
\usepackage{cmap}                       % Поддержка поиска русских слов в PDF (pdflatex)
\usepackage[X2, T2A]{fontenc}
%\usepackage[T2, OT1]{fontenc}
\usepackage[utf8]{inputenc}
\usepackage[english,russian]{babel}
\usepackage{indentfirst}                % Красная строка в первом абзаце
%\usepackage{misccorr}
%Может быть установлено 8pt, 9pt, 10pt, 11pt, 12pt, 14pt, 17pt, and 20pt
%\usepackage[12pt]{extsizes}
%\usepackage[mag=1000,a4paper,left=3cm,right=2cm,top=2cm,bottom=2cm,noheadfoot]{geometry}

\usepackage{amsmath} % разрешить \texttt и аналогичные в формулах
\usepackage{amssymb, } % дополнительные математические символы
\usepackage{graphicx} % поддержка изображений

%\usepackage{amsfonts, eucal, bm, color, }

\usepackage{algorithm, algorithmic}     % 'algorithm' environments
\floatname{algorithm}{Алгоритм}
\usepackage{multirow}                   % multirow cells in tables
\usepackage{arydshln}                   % dash lines in tables
\usepackage{subfig, float, wrapfig}     % sub figures
\usepackage{caption}                    % titles for figures
\usepackage{makeidx}                   % index
%\usepackage[xindy]{imakeidx}
\usepackage[totoc=true]{idxlayout}      % балансировка индексов на последней странице, индекс в ToC
\usepackage{enumerate}
\usepackage{fancybox}                   % страница в рамке
%\usepackage{fancyhdr}                  % глава и секция вверху страницы
%\usepackage{layout}
\usepackage[left=1.84cm, right=1.5cm, paperwidth=14cm, top=1.8cm, bottom=2cm, height=19.8cm, paperheight=20cm]{geometry}
\usepackage[parentracker=true,
  backend=biber,
  hyperref=auto,
  language=auto,
  citestyle=gost-numeric,
  defernumbers=true,
  bibstyle=gost-numeric,
  sortlocale=ru_RU
]{biblatex}								% библиография по ГОСТу
\addbibresource{bibliography.bib}

% поддержка гиперссылок; гиперссылки в pdf, должен быть последним загруженным пакетом
\ifx\pdfoutput\undefined
    \usepackage[unicode,dvips]{hyperref}
\else
    \usepackage[pdftex,colorlinks,unicode,bookmarks]{hyperref}
\fi

%\paperwidth=16.8cm \oddsidemargin=0cm \evensidemargin=0cm \hoffset=-0.33cm \textwidth=13.2cm
%\paperheight=24cm \voffset=-0.4cm \topmargin=0cm \headsep=0cm \headheight=0cm \textheight=19.8cm \footskip=0.9cm

% параметры PDF файла
\hypersetup{
    pdftitle={Защита информации},
    pdfauthor={Э. М. Габидулин, А. С. Кшевецкий, А. И. Колыбельников, С. М. Владимиров},
    pdfsubject=учебное пособие,
    pdfkeywords={защита информации, криптография, МФТИ}
}

% добавить точку после номера секции, раздела и т.д.
\makeatletter
\def\@seccntformat#1{\csname the#1\endcsname.\quad}
\def\numberline#1{\hb@xt@\@tempdima{#1\if&#1&\else.\fi\hfil}}
\makeatother

% перенос слов с тире
%\lccode`\-=`\-
%\defaulthyphenchar=127

% изменить подписи к рисункам, таблицам и т.д.
\captionsetup{labelsep=period}          % заменить : на .
\captionsetup{textformat=period}        % Подписи завершать точкой
%\captionsetup[table]{position=above}    % вертикальные отступы подписи таблицы для случая, когда подпись вверху
%\captionsetup[figure]{position=below}   % вертикальные отступы подписи рисунка для случая, когда подпись внизу

%% стиль главы и секции вверху страницы
%\pagestyle{fancy}
%%\renewcommand{\chaptermark}[1]{\markboth{#1}{}}
%\renewcommand{\sectionmark}[1]{\markright{#1}{}}
%
%%\fancyhf{}
%%\fancyfoot[СE,CO]{\thepage}
%%\fancyhead[LE]{\textsc{\nouppercase{\leftmark}}}
%\fancyhead[RO]{\textsc{\nouppercase{\rightmark}}}
%
%\fancypagestyle{plain}{ %
%\fancyhf{}                              % remove everything
%\renewcommand{\headrulewidth}{0pt}      % remove lines as well
%\renewcommand{\footrulewidth}{0pt}}

% запретить выходить за границы страницы
\sloppy

\newtheorem{theorem}{Теорема}[section]
\newtheorem{lemma}[theorem]{Лемма}
\newtheorem{definition}[theorem]{Определение}
\newtheorem{property}[theorem]{Утверждение}
\newtheorem{corollary}[theorem]{Следствие}
%\newtheorem{algorithm}[theorem]{Алгоритм}
\newtheorem{remark}[theorem]{Замечание}
\newcommand{\proof}{\noindent\textsc{Доказательство.\ }}

%\newtheorem{example}{\textsc{\textbf{Пример}}}
\newcommand{\example}{\textsc{\textbf{Пример.}} }
\newcommand{\exampleend}

\newcommand{\set}[1]{\mathbb{#1}}
\newcommand{\group}[1]{\mathbb{#1}}
\newcommand{\E}{\group{E}}
\newcommand{\F}{\group{F}}
\newcommand{\GF}[1]{\group{GF}(#1)}
\newcommand{\Gr}{\group{G}}
\newcommand{\R}{\group{R}}
\newcommand{\Z}{\group{Z}}
\newcommand{\MAC}{\textrm{MAC}}
\newcommand{\HMAC}{\textrm{HMAC}}
\newcommand{\PK}{\textrm{PK}}
\newcommand{\SK}{\textrm{SK}}

%Наконец, существует способ дублировать знаки операций, который мы приведем безо всяких пояснений. Включив
%\newcommand*{\hm}[1]{#1\nobreak\discretionary{}{\hbox{\mathsurround=0pt #1}}{}}
%в преамбулу, можно написать $a\hm+b\hm+c\hm+d$, при этом в формуле a\hm+b\hm+c\hm+d при переносе знак + будет продублирован.

% Дублирование символов бинарных операций ("+", "-", "="), набранных в строчных формулах, при переносе на другую строку:
%%begin{latexonly}
%\renewcommand\ne{\mathchar"3236\mathchar"303D\nobreak
%      \discretionary{}{\usefont
%      {OMS}{cmsy}{m}{n}\char"36\usefont
%      {OT1}{cmr}{m}{n}\char"3D}{}}
%\begingroup
%\catcode`\+\active\gdef+{\mathchar8235\nobreak\discretionary{}%
% {\usefont{OT1}{cmr}{m}{n}\char43}{}}
%\catcode`\-\active\gdef-{\mathchar8704\nobreak\discretionary{}%
% {\usefont{OMS}{cmsy}{m}{n}\char0}{}}
%\catcode`\=\active\gdef={\mathchar12349\nobreak\discretionary{}%
% {\usefont{OT1}{cmr}{m}{n}\char61}{}}
%\endgroup
%\def\cdot{\mathchar8705\nobreak\discretionary{}%
% {\usefont{OMS}{cmsy}{m}{n}\char1}{}}
%\def\times{\mathchar8706\nobreak\discretionary{}%
% {\usefont{OMS}{cmsy}{m}{n}\char2}{}}
%\mathcode`\==32768
%\mathcode`\+=32768
%\mathcode`\-=32768
%%end{latexonly}

\makeindex

\begin{document}
\selectlanguage{russian}

%\layout

% рамка границ страницы http://www.ctan.org/tex-archive/macros/latex/contrib/fancybox/fancybox-doc.pdf
% сделать поиск по fancypage, thisfancypage
%\thisfancypage{}{\fbox}
%\thisfancypage{\fbox}{}
%\fancypage{}{\fbox}         % закомментировать
%\fancypage{\fbox}{\fbox}    % закомментировать
%\fancypage{\setlength{\fboxsep}{32pt}\fbox}{}

\title{Защита информации \\ Учебное пособие}
\author{Габидулин Эрнст Мухамедович \\ Кшевецкий Александр Сергеевич \\ Колыбельников Александр Иванович \\ Владимиров Сергей Михайлович}
\date{
 %   \textbf{\textsc{Черновой вариант. Может содержать ошибки.}} \\
%    \today
}
\maketitle
\setcounter{page}{3}

\newpage
%\thispagestyle{empty}
\setcounter{tocdepth}{2}
\tableofcontents
%\thispagestyle{empty}
\newpage

%\lhead[\leftmark]{}
%\rhead[]{\rightmark}

\input{foreword}

\chapter{Основные понятия и определения}

\input{Short_history_of_cryptography}

\input{model_of_the_transmission_system_with_crypto}

\section{Классификация криптосистем}

\input{classification_by_symmetry}

\subsection{Шифры замены и перестановки}

Шифры, по способу преобразования открытого текста в шифротекст, разделяются на шифры замены и шифры перестановки.

\input{substitution_ciphers}

\input{permutation_ciphers}

\input{composite_chiphers}

\subsection{Примеры современных криптопримитивов}

Приведем примеры названий некоторых современных криптографических примитивов, из которых строят системы защиты информации:
\begin{itemize}
    \item DES\index{шифр!DES}, AES, ГОСТ 28147-89, Blowfish\index{шифр!Blowfish}, RC5\index{шифр!RC5}, RC6\index{шифр!RC6} -- блоковые симметричные шифры, скорость обработки -- десятки мегабайт в секунду,
    \item A5/1, A5/2, A5/3\index{шифр!A5}, RC4\index{шифр!RC4} -- потоковые симметричные шифры с высокой скоростью, семейство A5 применяется в мобильной связи GSM, RC4 -- в компьютерных сетях для SSL соединения между браузером и вебсервером,
    \item RSA\index{шифр!RSA} -- криптосистема с открытым ключом для шифрования,
    \item RSA\index{электронная подпись!RSA}, DSA\index{электронная подпись!DSA}, ГОСТ Р 34.10-2001\index{электронная подпись!ГОСТ Р 34.10-2001} -- криптосистемы с открытым ключом для электронной подписи,
    \item MD5\index{хэш-функция!MD5}, SHA-1\index{хэш-функция!SHA-1}, SHA-2\index{хэш-функция!SHA-2}, ГОСТ Р 34.11-94\index{хэш-функция!ГОСТ Р 34.11-94} -- криптографические хэш-функции.
\end{itemize}

\section{Методы криптоанализа и типы атак}
\selectlanguage{russian}

Нелегальный пользователь-криптоаналитик получает информацию путем дешифрования. Сложность этой процедуры определяется числом стандартных операций, которые надо выполнить для достижения цели. \textbf{Двоичной сложностью}\index{сложность!двоичная} (или битовой сложностью) алгоритма называется количество двоичных операций, которые необходимо выполнить для его завершения.
% Наиболее сложным является дешифрование полиалфавитных шифров.

Попытка криптоаналитика $E$ получить информацию называется \textbf{атакой} или криптоатакой\index{атака}. Как правило, легальным пользователям нужно обеспечить защиту информации на протяжении от нескольких дней до 100 лет. Если попытка атаки оказалась удачной для нелегального пользователя $E$, и информация получена или может быть получена в ближайшем будущем, то такое событие называется  \textbf{взломом криптосистемы}\index{взлом криптосистемы} или \textbf{вскрытием криптосистемы}. Метод вскрытия криптосистемы называется \textbf{криптоанализом}\index{криптоанализ}. Криптосистема называется \textbf{криптостойкой}\index{криптостойкость}, если число стандартных операций для ее взлома превышает возможности современных вычислительных средств в течение всего времени ценности информации (до 100 лет).

В общем случае, в криптоанализе под \textbf{взломом} криптосистемы понимается построение алгоритма криптоатаки для получения доступа к информации с количеством операций, меньшим, чем планировалось при создании этой криптосистемы. Взлом криптосистемы -- это не обязательно реально осуществленное извлечение информации, так как количество операций для извлечения информации может быть вычислительно недостижимым как в настоящее время, так и в течение всего времени защиты.
%, но предполагается достижимым в будущем.

Рассмотрим основные сценарии работы криптоаналитика $E$. В первом сценарии криптоаналитик может осуществлять подслушивание и (или) перехват сообщений. Его вмешательство не нарушает целостности информации: $Y=\widetilde{Y}$. Эта роль криптоаналитика называется \textbf{пассивной}. Так как он получает доступ к информации, то здесь нарушается конфиденциальность.

Во втором сценарии роль криптоаналитика \textbf{активная}. Он может подслушивать, перехватывать сообщения и преобразовывать их по своему усмотрению: задерживать, искажать с помощью перестановок пакетов, устраивать обрыв связи, создавать новые сообщения и т.~п. Так что в этом случае выполняется условие $Y \neq \widetilde{Y}$. Это значит, что одновременно нарушается целостность и конфиденциальность передаваемой информации.

Приведем примеры пассивных и активных атак.
\begin{itemize}
    \item Атака <<\textbf{человек посередине}>>\index{атака!<<человек посередине>>} (man-in-the-middle) подразумевает криптоаналитика, который разрывает канал связи, встраиваясь между $A$ и $B$, получает сообщения от $A$ и от $B$, а от себя отправляет новые, фальсифицированные сообщения. В результате $A$ и $B$ не замечают, что общаются с $E$, а не друг с другом.
    \item Атака \textbf{воспроизведения}\index{атака!воспроизведения} (replay attack) -- когда криптоаналитик может записывать и в будущем воспроизводить шифротексты, имитируя легального пользователя.
    \item Атака на \textbf{различение} сообщений\index{атака!на различение} означает, что криптоаналитик, наблюдая одинаковые шифротексты, может извлечь информацию об идентичности исходных открытых текстов.
    \item Атака на \textbf{расширение} сообщений\index{атака!на расширение} означает, что криптоаналитик может дополнить шифротекст осмысленной информацией без знания секретного ключа.
    \item \textbf{Фальсификация} шифротекстов\index{атака!фальсификацией} криптоаналитиком без знания секретного ключа.
\end{itemize}

Часто для нахождения секретного ключа криптоатаки строят в предположениях о доступности дополнительной информации. Приведем примеры.
\begin{itemize}
    \item Атака на основе известного открытого текста\index{атака!с известным открытым текстом} (CPA, chosen plaintext attack) предполагает возможность криптоаналитику выбирать открытый текст и получать для него соответствующий шифротекст.
    \item Атака на основе известного шифротекста\index{атака!с известным шифротекстом} (CCA, chosen ciphertext attack) предполагает возможность криптоаналитику выбирать шифротекст и получать для него соответствующий открытый текст.
\end{itemize}

Обязательным требованием к современным криптосистемам является устойчивость ко всем известным типам атак: пассивным, активным и с дополнительной информацией.


%Приведем примеры возможных вариантов работы активного криптоаналитика.
%\begin{itemize}
%\item Криптоаналитик имеет $m$ шифрованных сообщений $Y_{1},Y_{2},\ldots Y_{m}$ и пытается определить ключ или прочитать открытый текст $X_{1},X_{2},\ldots X_{m}.$
%\item Криптоаналитик имеет несколько пар открытого и шифрованного текстов
%
%$(Y_{1},X_{1}),(Y_{2}X_{2}),\ldots (Y_{m}X_{m})$ и пытается дешифровать остальной текст или определить алгоритм шифрования или определить ключ.
%\item
%\item
%\item
%\end{itemize}

Для защиты информации от активного криптоаналитика и обеспечения целостности дополнительно к шифрованию сообщений применяют имитовставку\index{имитовставка}. Для неё используют обозначение $\MAC$ (message authentication code). Как правило, $\MAC$ строится на основе хэш-функций, которые будут описаны далее.

Существуют ситуации, когда пользователи $A$ и $B$ не доверяют друг другу. Например, $A$ -- банк, $B$ -- получатель денег. $A$ утверждает, что деньги переведены, $B$ утверждает, что не переведены. Решение задачи аутентификации и неотрицаемости состоит в обеспечении \textbf{электронной подписью}\index{электронная подпись} каждого из абонентов. Предварительно надо решить задачу о генерировании и распределении секретных ключей.

В общем случае системы защиты информации должны обеспечивать:
\begin{itemize}
    \item конфиденциальность (защита от наблюдения),
    \item целостность (защита от изменения),
    \item аутентификацию (защита от фальсификации пользователя и сообщений),
    \item доказательство авторства информации (доказательство авторства и защита от его отрицания)
\end{itemize}
как со стороны получателя, так и со стороны отправителя.

Важным критерием для выбора степени защиты является сравнение стоимости реализации взлома для получения информации и экономического эффекта от владения ею. Очевидно, что если стоимость взлома превышает ценность информации, взлом нецелесообразен.

%Сценарии защиты информации
%   Сценарий 1. A -- передающая сторона. B -- принимающая сторона. E -- пассивный
%криптоаналитик, который может подслушивать передачу, но не может вмешиваться
%в процесс передачи. Цель защиты: обеспечение конфиденциальности. Средства
%-- методы шифрования с секретным ключом (симметричные системы шифрования)
%и методы шифрования с открытым ключом (асимметричные системы шифрования).
%Сценарий 2. E -- активный криптоаналитик, который может изменять, удалять и вставлять
%сообщения или их части. Цель защиты -- обеспечение конфиденциальности (не
%всегда) и обеспечение целостности. Средства -- методы шифрования и добавление
%имитовставки\index{имитовставка} (Message Autentication Code -- $\MAC$).
%Сценарий 3. A и B не доверяют друг другу. Цель защиты -- аутентификация пользователя.
%Средства -- электронная подпись.


\input{The_minimum_key_lengths}

\chapter{Классические шифры}

В главе приведены наиболее известные \emph{классические} шифры, которыми можно было пользоваться до появления роторных машин. К ним относятся такие шифры, как: шифр Цезаря, шифр Плейфера--Витстона, шифр Хилла, шифр Виженера. Они очень наглядно демонстрируют различные классы шифров.

\input{monoalphabetic_ciphers}

\input{bigrammnye_substitution_ciphers}

\input{hills_cipher}

% \subsection{Омофонные замены}
%
% Омофонными заменами называют криптопримитивы, в основе которых лежит замена групп символов открытого текста $M$ на группу символов $C$ с использованием ключа $K$. Такой метод шифрования вносит неоднозначность между $M$ и $C$, это позволяет защититься от методов частотного криптоанализа.
%  \subsection{шифрокоды}
%  Шифрокоды -- это класс шифров сочетающих в себе свойства кодов и помехозащищенности со свойствами шифра и обеспечения конфиденциальности.

\input{vigeneres_chipher}

\input{polyalphabetic_cipher_cryptanalysis}

\chapter[Совершенная криптостойкость]{Криптосистемы совершенной криптостойкости}
\selectlanguage{russian}

Рассмотрим модель криптосистемы, в которой Алиса выступает источником сообщений $m \in \group{M}$. Алиса использует некоторую функцию шифрования, результатом вычисления которой является шифротекст $c \in \group{C}$:

	\[c = E_{K_1}\left(m\right).\]

Шифротекст $c$ передаётся по открытому каналу легальному пользователю Бобу, причём по пути он может быть перехвачен нелегальным пользователем (криптоаналитиком) Евой.

Боб, обладая ключом расшифрования $K_2$, расшифровывает сообщение с использованием функции расшифрования:
	\[m' = D_{K_2}\left(c \right).\]

Рассмотрим теперь исходное сообщение, передаваемый шифротекст и ключи шифрования (и расшифрования, если они отличаются) в качестве случайных величин, описывая их свойства с точки зрения теории информации. Далее полагаем, что в криптосистеме ключи шифрования и расшифрования совпадают.

Будет называть криптосистему \textit{корректной}, если она обладает следующими свойствами.
\begin{itemize}
	\item Легальный пользователь имеет возможность однозначно восстановить исходное сообщение, то есть
					\[H \left( M | C, K \right) = 0, \]
					\[m' = m.\]
	\item Выбор ключа шифрования не зависит от исходного сообщения
					\[ I \left( K ; M \right) = 0, \]
					\[ H \left( K | M \right) = H \left( K \right). \]
\end{itemize}

Второе условие является в некотором виде условием на возможность отделить ключ шифрования от данных и алгоритма шифрования.

\section[Определения]{Определения совершенной криптостойкости}

Понятие совершенной секретности (или стойкости) введено американским ученым Клодом Шенноном. В конце Второй мировой войны он закончил работу, посвященную теории связи в секретных системах\cite{Shannon:1949:CTS}. Эта работа вошла составной частью в собрание его трудов, вышедшее в русском переводе в 1963 году.~\cite{Shannon:1963} Понятие о стойкости шифров по Шеннону связано с решением задачи криптоанализа по одной криптограмме.

Криптосистемы совершенной стойкости могут применяться как в современных вычислительных сетях, так и для шифрования любой бумажной корреспонденции. Основной проблемой применения данных шифров для шифрования больших объемов информации является необходимость распространения ключей объемом не менee чем передаваемые сообщения.

\begin{definition}\label{perfect_by_probabilities}
Будем называть криптосистему \textbf{совершенно криптостойкой}, если апостериорное распределение вероятностей исходного случайного сообщения $m_i \in \group{M}$ при регистрации случайного шифротекста $c_k \in \group{C}$ совпадает с априорным распределением~\cite{Gultyaeva:2010}:

	\[\forall m_j \in \group{M}, c_k \in \group{C}: P \left( m = m_j | c = c_k \right) = P \left( m = m_j \right).\]
\end{definition}

Данное условие можно переформулировать в терминах статистических свойств сообщения, ключа и шифротекста как случайных величин.

\begin{definition}\label{perfect_by_enthropy}
Криптосистема называется совершенно криптостойкой, если условная энтропия сообщения при известном шифротексте равна безусловной:
	\[H \left( M | C \right) = H \left( M \right),\]
	\[I \left( M; C \right) = 0.\]
\end{definition}

Можно показать, что определения \ref{perfect_by_probabilities} и \ref{perfect_by_enthropy} тождественны.

\section[Условие]{Условие совершенной криптостойкости}

Найдем оценку количества информации, которое содержит шифротекст $C$ относительно сообщения $M$
\[ I(M; C) = H(M) - H(M | C). \]
Очевидны следующие соотношения условных и безусловных энтропий \cite{GabPil:2007}:
\[H(K|C)=H(K|C)+H(M|K,C)=H(M,K|C),\]
\[H(M,K|C)=H(M|C)+H(K|M,C)\geq H(M|C),\]
\[H(K)\geq H(K|C)\geq H(M|C).\]
Отсюда получаем:
 \[ I(M; C) = H(M) - H(M | C)\geq H(M)-H(K). \]
Из последнего неравенства следует, что взаимная информация между сообщением и шифротекстом равна нулю, если энтропия ключа не меньше энтропии сообщений. С другой стороны, взаимная информация между сообщением и шифротекстом равна нулю, если они статистически независимы. Таким образом, условием совершенной криптостойкости является неравенство
\[ H(M) \leq H(K).\]
%Если утверждение верно, то количество информации в шифротексте относительно открытого текста $I(M; C)$ равно нулю:
%  \[ I(M; C) = H(M) - H(M | C) = 0, \]
%так как для статистически независимых величин условная энтропия равна безусловной энтропии, то есть $H(M) = H(M | C)$.

%Функцию шифрования обозначим $E: \{ M, K \} \rightarrow C$. Процедура шифрования состоит из следующих шагов.
%\begin{itemize}
%    \item Легальный пользователь $A$ выбирает ключ $k \in K$ и секретно сообщает его легальному пользователю $B$ (дополнительная задача -- распределение ключей).
%    \item По открытому сообщению $m \in M$ и выбранному ключу $k$ вычисляют шифрованное сообщение $c = E_k(m) \in C$.
%\end{itemize}

%Основное требование при шифровании состоит в том, чтобы при выбранном ключе $k$ вычисление $c$  было легкой задачей для любого сообщения $m$.

%Функцию расшифрования обозначим $D: \{ C, K \} \rightarrow M$. Процедура расшифрования состоит из следующих шагов.
%\begin{itemize}
%    \item Легальный пользователь $B$ получает от $A$ секретный ключ $k \in K$.
 %   \item $B$ по принятому шифрованному сообщению $c \in C$ и известному ключу $k$ вычисляет открытое сообщение $m = D_k(c) \in M$.
%\end{itemize}

%Основное требование: при выбранном ключе $k$ вычисление $m$ должно быть легкой задачей для любого $c$. С другой стороны, при неизвестном ключе $k$ вычисление открытого сообщения $m$ по известному шифрованному сообщению $c$ должно быть трудной задачей для любого $c$.

%Криптостойкость шифра оценивается числом операций, необходимым для определения: открытого текста $m$ по шифротексту $c$, либо ключа шифрования $k$ по открытому тексту $m$ и шифротексту $c$.

%$M, C, K$ интерпретируются как случайные величины.
%Пусть заданы распределения вероятностей $P_m(M), P_c(C), P_k(K)$. По определению шифрование $C = E_K(M)$ -- детерминированная функция своих аргументов.
%Если при выбранном шифре оказалось, что открытый текст $M$ и шифротекст $C$ -- статистически независимые случайные величины, то считается, что такая система обладает совершенной криптостойкостью.


%\subsection{Длина ключа}

%Пусть сообщения $m\in M$ и ключи $r\in K$ являются независимыми случайными величинами. Это значит, что их совместная вероятность $P_{mk}(M, K)$ равна произведению отдельных вероятностей:
%\[P_{mk}(M, K) = P_m(M) \cdot P_k(K).\]
%Пусть $C = E_K(M)$ -- множество шифрованных текстов, $M = D_K(C)$ -- множество расшифрованных текстов. Можно найти вероятности $P_c(C), P_{mck}(M,C,K)$.

%Используя известные соотношения о безусловной и условной энтропии~\cite{GabPil:2007}, оценим энтропию открытых текстов $M$ с учетом статистической независимости $M$ и $C$:
 %   \[ H(M) = H(M | C) \leq H(MK | C) = H(K | C) + H(M | CK) = \]     \[ = H(K | C) \leq H(K). \]

%Так как энтропия открытого текста при заданном шифротексте и известном ключе равна нулю, то $H(M|CK)=0$. В результате получаем     \[ H(M) \leq H(K). \]

Обозначим $L(M)$ и $L(K)$ длину сообщений и ключа соответственно. Известно~\cite{GabPil:2007}, что $H(M)\leq L(M)$ и равенство достигается, когда сообщения состоят из статистически независимых и равновероятных символов. Такое же свойство выполняется и для случайных ключей $H(K)\leq L(K)$. Таким образом, достаточным условием совершенной криптостойкости системы можно считать неравенство
 \[ L(M) \leq L(K)\]
при случайном выборе ключа.

%С другой стороны, энтропия открытого текста $H(M)$ характеризует длину последовательности для описания случайной величины $M$ (открытого сообщения), а $H(K)$ характеризует длину последовательности для описания ключа. Следовательно, совершенная криптостойкость возможна только тогда, когда длина ключа не меньше, чем длина шифруемого сообщения, то есть     \[ H(M) \leq H(K). \] Как правило, длина сообщения заранее неизвестна и ограничена большим числом. Выбрать ключ длины не меньшей, чем возможное сообщение не представляется возможным или рациональным, и один и тот же ключ (или его преобразования) используется многократно для шифрования блоков сообщения фиксированной длины. То есть, $H(K) \ll H(M)$.

На самом деле, сообщение может иметь произвольную (заранее не ограниченную) длину. Поэтому генерация и главным образом доставка легальным пользователям случайного и достаточного длинного ключа становятся критическими проблемами. Практическим решением этих проблем является многократное использование одного и того же ключа при условии, что его длина гарантирует вычислительную невозможность любой известной атаки на подбор ключа.


\section{Криптосистема Вернама}
\index{криптосистема!Вернама}
Приведем пример системы с совершенной криптостойкостью.

Пусть сообщение представлено двоичной последовательностью длины $N$:
    \[ m = (m_1, m_2, \dots, m_N). \]
Распределение вероятностей сообщений $P_m(m)$ может быть любым. Ключ также представлен двоичной последовательностью $ k = (k_1, k_2, \dots, k_N)$ той же длины, но с равномерным распределением
    \[ P_k(k) = \frac{1}{2^N} \]
для всех ключей.

Шифрование в криптосистеме \textbf{Вернама} осуществляется путем покомпонентного суммирования по модулю алфавита последовательностей открытого текста и ключа:
    \[ C = M \oplus K = (m_1 \oplus k_1, ~ m_2 \oplus k_2, \dots,  m_N \oplus k_N). \]

Легальный пользователь знает ключ и осуществляет расшифрование:
    \[ M =C \oplus K = (m_1 \oplus k_1, ~ m_2 \oplus k_2, \dots, m_N \oplus k_N). \]

Найдем вероятностное распределение $N$-блоков шифротекстов, используя формулу
    \[ P(c = a) = P(m \oplus k = a) = \sum_{m} P(m) P(m \oplus k = a | m) = \]
    \[ = \sum_{m} P(m) P(k \oplus m) = \sum_{m} P(m) \frac{1}{2^N} = \frac{1}{2^N}. \]

Получили подтверждение известного факта: сумма двух случайных величин, одна из которых имеет равномерное распределение, является случайной величиной с равномерным распределением. В нашем случае распределение ключей равномерное, поэтому распределение шифротекстов тоже равномерное.

Запишем совместное распределение открытых текстов и шифротекстов:
    \[ P(m = a, c = b) ~=~ P(m = a) ~ P(c = b | m = a). \]

Найдем условное распределение
    \[ P(c = b | m = a) ~=~ P(m \oplus k = b | m = a) ~= \]
    \[ =~ P(k = b \oplus a | m = a) ~=~ P(k = b \oplus a) ~=~ \frac{1}{2^N}, \]
так как ключ и открытый текст являются независимыми случайными величинами. Итого:
    \[ P(c=b | m=a) = \frac{1}{2^N}. \]

Подстановка правой части этой формулы в формулу для совместного распределения дает
    \[ P(m=a,c=b)=P(m=a)\frac{1}{2^N}, \]
что доказывает независимость шифротекстов и открытых текстов в этой системе. По доказанному выше, количество информации в шифротексте относительно открытого текста равно нулю. Это значит, что рассмотренная криптосистема Вернама обладает совершенной секретностью (криптостойкостью) при условии, что для каждого $N$-блока (сообщения) генерируется случайный (одноразовый) $N$-ключ.

\input{unicity_distance}


\chapter{Блоковые шифры}\label{chapter-block-ciphers}

\input{block_ciphers}

\input{lucifer}

\section{Ячейка Фейстеля}
\selectlanguage{russian}

Одним из основных методов построения современных блоковых шифров является ячейка \textbf{Фейстеля} (Feistel), изображенная на рисунке \ref{fig:Feistel}. Главная особенность шифрования с ячейкой Фейстеля состоит в том, что обратимость шифрования (т.е. расшифрование) не зависит от обратимости преобразования $F$ внутри ячейки. Широкое применение ячеек Фейстеля в шифрах 1970--90-х годов вызвано бурным развитием персональных компьютеров. Шифрование выполняется \textbf{раундами}, на каждом раунде выполняется одно и то же преобразование ячейки Фейстеля, но с разными ключами. Общее количество раундов 16--32.

\begin{figure}[!ht]
    \centering
    \includegraphics[width=0.6\textwidth]{pic/feistel}
    \caption{Ячейка Фейстеля\label{fig:Feistel}}
\end{figure}

Здесь введены обозначения: $X$ -- блок двоичных символов, который записан в регистр памяти, состоящий из двух частей, $X = (L,R)$, где $L$ -- начальное содержимое левого регистра, $\tilde{L}$ -- содержимое левого регистра сдвига после преобразования, $R$ -- начальное содержимое правого регистра, $\tilde{R}$ -- содержимое правого регистра после преобразования, $K$ -- ключ шифрования, задающий преобразование $F(K,R)$. Знак $\oplus$ определяет операцию побитового суммирования по модулю 2, то есть операцию XOR. Перекрестные линии указывают на замену частей регистра. После одного элементарного преобразования содержимое правого регистра заменяется содержимым левого регистра и наоборот. $\tilde{L},\tilde{R}$ -- результат элементарного шифрования, выполненного за один раунд. $L_{1}$ -- содержимое правого регистра после замены, $R_{1}$ -- содержимое левого регистра после замены. Основным шифрующим преобразованием является функция $F$.

Ячейка Фейстеля -- произведение двух перестановок $T$ и $G$, где $T$ -- замена левой части на правую и наоборот. Запишем преобразование $Y=TG(X,L)$, выполняемое этой ячейкой:
\[
  \begin{array}{l}
    \tilde{X} = (\tilde{L}, \tilde{R}) = (L \oplus F(K,R), R) \equiv G(X, L), \\
    Y = TGX. \\
  \end{array}
\]

Если дважды применим перестановку, то получим снова открытый текст:
\[
    \begin{array}{l}
        \tilde{\tilde{L}} = \tilde{L} \oplus F(K, \tilde{R}) = (L \oplus F(K,R) \oplus F(K,R)) = L, \\
        \tilde{\tilde{R}} = R.\\
    \end{array}
\]

Многократное применение преобразования $Y=TG(X,L)$ с различными ключами представим в виде
\[
  \begin{array}{l}
    Y_1 = T G_1 X,\\
    Y_2 = T G_2 Y_1 = T G_2 T G_1 X, \\
    \ldots, \\
    Y_{m-1} = T G_{m-1} Y_{m-2} = T G_{m-1} T G_{m-2} \ldots T G_1 X.\\
  \end{array}
\]
В первом уравнении показан результат первого шифрования с ключом $K_{1}$, во втором уравнении -- результат шифрования с ключом $K_{2}$ и т.д., в $(m-1)$-м уравнении -- результат с ключом $K_{m-1}$. В последнем, $(m)$-м уравнении, перестановку $T$ можно не использовать:
\[
   Y_{m}= G_{m} Y_{m-1} = G_{m} T G_{m-1} \ldots T G_{1} X.\\
\]

Как видно из приведенных соотношений, пара величин -- содержимое регистра и первый ключ $X, K_{1}$ -- влияет на все позиции шифрованного текста. Полностью разрушается статистическая структура исходного текста за счет преобразований, вызывающих \emph{лавинный эффект}\index{лавинный эффект}. \textbf{Лавинный эффект} -- это распространение <<влияния>> одного бита открытого текста (или ключа) на все остальные биты шифруемого блока за определенное количество раундов.
%Для ячеек Фейстеля это количество равно 2.

Одной из характеристик блокового шифра является число раундов, за которое достигается полная диффузия (конфузия) -- зависимость всех битов выхода (входа) от всех битов входа (выхода). Вход -- это открытый текст и ключ.

Криптостойкость ячейки Фейстеля подтверждается тем фактом, что не существует примеров ее взлома (в случае шифра DES взлом был сделан полным перебором 56-битного ключа, а не взломом самой криптосистемы; например, российский стандарт ГОСТ 28147-89 на ячейке Фейстеля с 256-битовым ключом не взломан).

Рассмотрим процедуру расшифрования. Легальный пользователь знает все ключи и последовательность их применения. Он выполняет следующие операции. Имеем шифрованное сообщение $Y_{m}$. На первом шаге вычисляет
\[
    G_{m} Y_{m} = G_{m} G_{m} Y_{m-1} = Y_{m-1}.
\]
На втором шаге использует найденное сообщение $Y_{m-1}$ и аналогично находит $Y_{m-2}$:
\[
    G_{m-1} T Y_{m-1} = G_{m-1} T T G_{m-1} Y_{m-2} = Y_{m-2}.
\]
Продолжает этот процесс до получения $Y_{1}$. После этого находит $X$:
\[
    G_{1} T Y_{1} = G_{1} T T G_{1} X = X.
\]
Как показали эти операции, вычислительная сложность устройства расшифрования ячейки Фейстеля такая же, как сложность устройства шифрования.

Раундовые блоковые шифры должны обеспечивать \emph{диффузию}, при которой каждый бит входа и ключа влияет на все биты выхода, и \emph{конфузию}, при которой каждый бит выхода нелинейно зависит от всех битов входа и ключа.

Основные свойства, которыми должна обладать функция $F$:
\begin{itemize}
    \item создание лавинного эффекта;
    \item нелинейность по отношению к операции XOR.
\end{itemize}

Как правило, функция $F$ включает таблицы перестановки $P$ и подстановки групп бит, так называемые $s$-блоки\index{$s$-блок} (от слова substitution), и функцию перестановки, перемешивающую биты между последовательно исполняемыми $s$-блоками. В совокупности эти действия и обеспечивают требуемые свойства ячейки.


\input{GOST_28147-89}

\section{Американский стандарт шифрования AES}
\selectlanguage{russian}

До 2001 г. американским стандартом шифрования данных был DES (аббревиатура от Data Encryption Standard), который был принят в 1980 году. Входной блок открытого текста и выходной блок шифрованного текста DES составляли по 64 бита каждый, длина ключа -- 56 бит (до процедуры расширения). Алгоритм основан на ячейке Фейстеля с $s$-блоками и таблицами расширения и перестановки бит. Количество раундов -- 16.

Для повышения криптостойкости и замены стандарта DES был объявлен конкурс на новый стандарт AES (аббревиатура от Advanced Encryption Standard). Победителем конкурса стал шифр Rijndael. Название составлено с использованием первых слогов фамилий его создателей (Rijmen and Daemen). В русскоязычном варианте читается как <<Рейндолл>>. Стандарт утвержден 26 мая в 2002 г.

AES -- это итеративный блочный шифр с переменной длиной ключа (128, 192 или 256 бит) и фиксированной длиной входного и выходного блоков (128 бит).


\subsection[Состояние, ключ шифрования и число раундов]{Состояние, ключ шифрования и число \protect\\ раундов}

Различные преобразования воздействуют на результат промежуточного шифрования, называемый \textit{состоянием} ($\mathsf{State}$). Состояние представлено $(4 \times 4)$-матрицей из байтов.

\textit{Ключ шифрования раунда} ($\mathsf{Key}$) также представляется прямоугольной $(4 \times \mathsf{Nk})$-матрицей из байтов $k_{i,j}$, где $\mathsf{Nk}$ равно длине ключа, разделенной на 32, то есть 4, 6 или 8.

Эти представления приведены ниже.
\[
    \mathsf{State} = \left[ \begin{array}{cccc}
        a_{0,0} & a_{0,1} & a_{0,2} & a_{0,3} \\
        a_{1,0} & a_{1,1} & a_{1,2} & a_{1,3} \\
        a_{2,0} & a_{2,1} & a_{2,2} & a_{2,3} \\
        a_{3,0} & a_{3,1} &a_{3,2} & a_{3,3}  \\
    \end{array} \right],
\] \[
    \mathsf{Key} = \left[ \begin{array}{cccc}
        k_{0,0} & k_{0,1} & k_{0,2} & k_{0,3} \\
        k_{1,0} & k_{1,1} & k_{1,2} & k_{1,3} \\
        k_{2,0} & k_{2,1} & k_{2,2} & k_{2,3} \\
        k_{3,0} & k_{3,1} & k_{3,2} & k_{3,3} \\
    \end{array} \right].
\]

Иногда блоки символов интерпретируются как одномерные последовательности из 4-байтовых векторов, где каждый вектор является соответствующим столбцом прямоугольной таблицы. В этих случаях таблицы можно рассматривать как наборы из 4, 6 или 8 векторов, нумеруемых в диапазоне $0 \dots 3, 0 \dots 5$ или $0 \dots 7$. Сами 4-байтовые векторы называют словами. В тех случаях, когда нужно пометить индивидуальный байт внутри 4-байтового вектора или слова, используется обозначение $(a, b, c, d)$, где $a, b, c, d$ соответствуют байтам в одной из позиций $0, 1, 2, 3$ в столбце, векторе или слове.

\textit{Входные} и \textit{выходные} блоки шифра AES рассматриваются как последовательности 16 байт $(a_0, a_1, \dots, a_{15})$. Преобразование входного блока $(a_0, \dots, a_{15})$ в исходную $(4 \times 4)$ матрицу состояния $\mathsf{State}$ или конечной матрицы состояния в выходную последовательность проводится по правилу (запись по столбцам):
    \[ a_{i,j} = a_{i + 4j}, ~ i = 0 \dots 3, ~ j = 0 \dots 3. \]

Аналогично ключ шифрования может рассматриваться как последовательность байт $(k_0, k_1, \dots, k_{4 \cdot \mathsf{Nk} - 1})$, где $\mathsf{Nk} = 4, 6, 8$. Число байт в этой последовательности равно 16, 24 или 32, а номера этих байт находятся в интервалах $0 \dots 15, ~ 0 \dots 23$ или $0 \dots 31$ соответственно. $(4 \times \mathsf{Nk})$-матрица ключа шифрования $\mathsf{Key}$ задается по правилу:
    \[ k_{i,j} = k_{i + 4j}, ~ i = 0 \dots 3, ~ j = 0 \dots \mathsf{Nk} - 1. \]

Число раундов $\mathsf{Nr}$ зависит от длины ключа. Его значения приведены в таблице ниже.

\begin{center}
    \begin{tabular}{|l|c|c|c|}
    \hline
    Длина ключа, биты           &128 & 192 & 256 \\
    $\mathsf{Nk}$               & 4  & 6   & 8 \\
    Число раундов $\mathsf{Nr}$ & 10 & 12 & 14 \\
    \hline
    \end{tabular}
\end{center}


\subsection{Операции в поле}

При переходе от одного раунда к другому матрицы \textit{состояния} и \textit{ключа шифрования раунда} подвергаются ряду преобразований. Преобразования могут осуществляться над:
\begin{itemize}
    \item отдельными байтами или парами байтов (необходимо определить операции сложения и умножения);
    \item столбцами матрицы, которые рассматриваются как 4-мерные векторы с соответствующими байтами в качестве элементов;
    \item строками матрицы.
\end{itemize}

В алгоритме шифрования AES байты рассматриваются как элементы поля $\GF{2^8}$, а вектор-столбцы из четырех байтов -- как многочлены третьей степени над полем $\GF{2^8}$. В Приложении \ref{chap:discrete-math} дано подробное описание этих операций.

\subsubsection{Сложение и умножение байтов}

Байты $a_1$ и $a_2$, как элементы поля Галуа $\GF{2^8}$, представлены многочленами 7-ой степени $a_1(x)$ и $a_2(x)$ с двоичными коэффициентами. Сложение байтов выполняется как сложение многочленов $a_1(x) + a_2(x)$ в поле $\GF{2}$. Умножение байта $a_1(x)$ на байт $a_2(x)$ в поле $\GF{2^8}$ производится по модулю неприводимого многочлена
    \[ m(x) = x^{8} + x^{4} + x^{3} + x + 1. \]

\subsubsection{Сложение и умножение вектор-столбцов из байтов}

Следующий тип операций -- это операции над столбцами матриц. Вектор-столбец $\mathbf{a}$, состоящий из байтов $(a_0, a_1, a_2, a_3)$, интерпретируется как многочлен третьей степени $\mathbf{a}(y)$ над полем $\GF{2^8}$, то есть
    \[ \mathbf{a}(y) = a_3 y^{3} + a_2 y^{2} + a_1 y + a_0, ~ a_i \in \GF{2^8}. \]

Сложение двух векторов $\mathbf{a}_1$ и $\mathbf{a}_2$ из 4 байтов определяется как сложение многочленов $\mathbf{a}_1(y) + \mathbf{a}_2(y)$ над полем $\GF{2^8}$. Умножение вектора $\mathbf{a}_1$ на вектор $\mathbf{a}_2$ задано как умножение многочлена $\mathbf{a}_1(y)$ на многочлен $\mathbf{a}_2(y)$ над полем $\GF{2^8}$ по модулю многочлена
    \[ \mathbf{M}(y)= \mathrm{'01'} y^4 + \mathrm{'01'} = y^4 + 1, ~ \mathrm{'01'}=1 \in \GF{2^8}, \]
    \[ \mathbf{M}(y)= (01, 00, 00, 01), \]
который не является неприводимым над $\GF{2^8}$. Чтобы подчеркнуть, что коэффициенты многочлена являются элементами поля $\GF{2^8}$, используется нотация из описания алгоритма в виде значений байтов в 16-ричной системе в кавычках, например $\mathrm{'01'}$.

Операция умножения по этому модулю обозначается как $\otimes$:
    \[ \mathbf{a}_1(y) ~ \mathbf{a}_2(y) \mod \mathbf{M}(y) \equiv \mathbf{a}_1(y) \otimes \mathbf{a}_2(y). \]

Третий тип операции <<Перемешивание столбца>> состоит в умножении многочлена вектора-столбца из 4 байтов на многочлен
    \[ \mathbf{c}(y) = (03, 01, 01, 02) = \mathrm{'03'} y^3 + \mathrm{'01'} y^2 + \mathrm{'01'} y + \mathrm{'02'} \]
по модулю $\mathbf{M}(y)$. Для многочлена  $\mathbf{c}(y)$ существует обратный многочлен
    \[ \mathbf{d}(y) = \mathbf{c}^{-1}(y) \mod \mathbf{M}(y) = (0B, 0D, 09, 0E) = \]
        \[ = \mathrm{'0B'} y^3 + \mathrm{'0D'} y^2 + \mathrm{'09'} y + \mathrm{'0E'}, \]
    \[ \mathbf{c}(y) \otimes \mathbf{d}(y) = (00, 00, 00, 01) = 1. \]
Многочлен $\mathbf{d}(y)$ используется вместо $\mathbf{c}(y)$ при расшифровании.


\subsection{Операции одного раунда шифрования}

В каждом раунде шифра AES, кроме последнего раунда, производятся следующие 4 операции с использованием таблиц:

\begin{itemize}
  \item замена байтов, $\mathsf{SubBytes}$;
  \item сдвиг строк, $\mathsf{ShiftRows}$;
  \item перемешивание столбцов, $\mathsf{MixColumns}$;
  \item добавление текущего ключа, $\mathsf{AddRoundKey}$.
\end{itemize}
В последнем раунде исключается операция <<Перемешивание столбцов>>. В обозначениях, близких к языку С, можно записать программу в виде
\[
    \begin{array}{l}
        \mathsf{Round(State, RoundKey)} \{ \\
        ~~~~ \mathsf{SubBytes(State)}; \\
        ~~~~ \mathsf{ShiftRows(State)}; \\
        ~~~~ \mathsf{MixColumns(State)}; \\
        ~~~~ \mathsf{AddRoundKey(State, RoundKey)}; \\
        \} \\
    \end{array}
\]
Последний раунд слегка отличается и записан в виде
\[
    \begin{array}{l}
        \mathsf{Round(State, RoundKey)} \{ \\
        ~~~~ \mathsf{SubBytes(State)}; \\
        ~~~~ \mathsf{ShiftRows(State)}; \\
        ~~~~ \mathsf{AddRoundKey(State, RoundKey)}; \\
        \} \\
    \end{array}
\]
В этих обозначениях все <<функции>>, а именно, $\mathsf{Round}$, $\mathsf{SubBytes}$, $\mathsf{ShiftRows}$, $\mathsf{MixColumns}$, $\mathsf{AddRoundKey}$, воздействуют на матрицы, определяемые указателем $\mathsf{(State, RoundKey)}$. Сами преобразования описаны в следующих разделах.


\subsubsection{Замена байтов $\mathsf{SubBytes}$}

Нелинейная операция <<Замена байтов>> действует независимо на каждый байт $a_{i,j}$ текущего состояния. Таблица замены (или $s$-блок) является обратимой и формируется последовательным применением двух преобразований.

\begin{enumerate}
    \item Сначала байт $a$ представляется как элемент $a(x)$ поля Галуа $\GF{2^8}$ и заменяется на обратный элемент $a^{-1} \equiv a^{-1}(x)$ в поле. Байт $\mathrm{'00'}$, для которого обратного элемента не существует, переходит сам в себя.
    \item Затем к обратному байту $a^{-1} = (x_0, x_1, x_2, x_3, x_4, x_5, x_6, x_7)$ применяется аффинное преобразование над полем $\GF{2}$ следующего вида:
        \[
            \left[  \begin{array}{c}
                y_{0} \\ y_{1} \\ y_{2} \\ y_{3} \\ y_{4} \\ y_{5} \\ y_{6} \\ y_{7} \\
            \end{array} \right] = \left[ \begin{array}{cccccccc}
                1 & 0 & 0  & 0 & 1 & 1 & 1 & 1 \\
                1 & 1 & 0  & 0 & 0 & 1 & 1 & 1 \\
                1 & 1 & 1  & 0 & 0 & 0 & 1 & 1 \\
                1 & 1 & 1  & 1 & 0 & 0 & 0 & 1 \\
                1 & 1 & 1  & 1 & 1 & 0 & 0 & 0 \\
                0 & 1 & 1  & 1 & 1 & 1 & 0 & 0 \\
                0 & 0 & 1  & 1 & 1 & 1 & 1 & 0 \\
                0 & 0 & 0  & 1 & 1 & 1 & 1 & 1  \
            \end{array} \right] \cdot \left[ \begin{array}{c}
                x_{0} \\ x_{1} \\ x_{2} \\ x_{3} \\ x_{4} \\ x_{5} \\ x_{6} \\ x_{7} \\
            \end{array} \right] + \left[ \begin{array}{c}
                1 \\ 1 \\ 0 \\ 0 \\ 0 \\ 1 \\ 1 \\ 0 \\
            \end{array} \right].
        \]
\end{enumerate}

В полиномиальном представлении это аффинное преобразование имеет вид:
\[Y(z)=(z^4)X(z)(1+z+z^2+z^3+z^4)\mod(1+z^8) + F(z).\]
Применение описанных операций $s$-блока ко всем байтам текущего состояния обозначено
    \[ \mathsf{SubBytes(State)}. \]

Обращение операции $\mathsf{SubBytes(State)}$ также является заменой байтов. Сначала выполняется обратное аффинное преобразование, а затем от полученного байта берется обратный.


\subsubsection{Сдвиг строк $\mathsf{ShiftRows}$}

Для выполнения операции <<Сдвиг строк>> строки в таблице текущего состояния циклически сдвигаются влево. Величина сдвига различна для различных строк. Строка $0$ не сдвигается вообще. Строка $1$ сдвигается на $C_1=1$ позицию, строка $2$ –- на $C_2=2$ позиции, строка $3$ -– на $C_3=3$ позиции.
%Величины $C1,C2$ и $C3$ зависят от $Nb$. Их значения приведены в табл. \ref{tab:AES-shift-rows}.
%
%\begin{table}[!ht]
%    \centering
%    \begin{tabular}{|c|c|c|c|}
%        \hline
%        Nb & C1 & C2 & C3 \\
%        \hline
%        4  & 1  & 2  & 3  \\
%        \hline
%        6  & 1  & 2  & 3  \\
%        \hline
%        8  & 1  & 3  & 4  \\
%        \hline
%    \end{tabular}
%    \caption{Сдвиг $C$ и длина блока $Nb$.}
%    \label{tab:AES-shift-rows}
%\end{table}


\subsubsection{Перемешивание столбцов $\mathsf{Mix Columns}$}

При выполнении операции <<Перемешивание столбцов>> столбцы матрицы текущего состояния рассматриваются как многочлены над полем $\GF{2^8}$ и умножаются по модулю многочлена $y^4 +1$ на фиксированный многочлен $\mathbf{c}(y)$, где
    \[ \mathbf{c}(y) = \mathrm{'03'} y^3 + \mathrm{'01'} y^2 + \mathrm{'01'} y + \mathrm{'02'}. \]
Этот многочлен взаимно прост с многочленом $y^4 + 1$ и, следовательно, обратим. Перемножение удобнее проводить в матричном виде. Если $\mathbf{b}(y) = \mathbf{c}(y) \otimes \mathbf{a}(y)$, то
\[
    \left[ \begin{array}{c}
        b_{0} \\ b_{1} \\ b_{2} \\ b_{3} \\
    \end{array}\right] =  \left[ \begin{array}{cccc}
        \mathrm{'02'} & \mathrm{'03'} & \mathrm{'01'} & \mathrm{'01'} \\
        \mathrm{'01'} & \mathrm{'02'} & \mathrm{'03'} & \mathrm{'01'} \\
        \mathrm{'01'} & \mathrm{'01'} & \mathrm{'02'} & \mathrm{'03'} \\
        \mathrm{'03'} & \mathrm{'01'} & \mathrm{'01'} & \mathrm{'02'} \\
    \end{array} \right] \cdot \left[ \begin{array}{c}
        a_{0} \\ a_{1} \\ a_{2} \\ a_{3} \\
     \end{array} \right].
\]

Обратная операция состоит в умножении на многочлен $\mathbf{d}(y)$, обратный многочлену $\mathbf{c}(y)$ по модулю $y^4 + 1$, то есть
\[
    (\mathrm{'03'} y^{3} + \mathrm{'01'} y^{2} + \mathrm{'01'} y + \mathrm{'02'}) \otimes \mathbf{d}(y) = \mathrm{'01'}.
\]
Этот многочлен равен
\[
    \mathbf{d}(y) = \mathrm{'0B'} y^3 + \mathrm{'0D'} y^2 + \mathrm{'09'} y + \mathrm{'0E'}.
\]


\subsubsection{Добавление ключа раунда $\mathsf{AddRoundKey}$}

Операция <<Добавление ключа раунда>> состоит в том, что к матрице текущего состояния добавляется по модулю $2$ матрица ключа текущего раунда. Обе матрицы должны иметь одинаковые размеры. Матрица ключа раунда вычисляется с помощью процедуры \emph{расширения ключа}, описанной ниже. Операция <<Добавление ключа раунда>> обозначается $\mathsf{AddRoundKey(State, RoundKey)}$.

\[
    \left[ \begin{array}{cccc}
        a_{0,0} & a_{0,1} & a_{0,2} & a_{0,3} \\
        a_{1,0} & a_{1,1} & a_{1,2} & a_{1,3} \\
        a_{2,0} & a_{2,1} & a_{2,2} & a_{2,3} \\
        a_{3,0} & a_{3,1} & a_{3,2} & a_{3,3}
    \end{array} \right]
    \oplus
    \left[ \begin{array}{cccc}
        k_{0,0} & k_{0,1} & k_{0,2} & k_{0,3} \\
        k_{1,0} & k_{1,1} & k_{1,2} & k_{1,3} \\
        k_{2,0} & k_{2,1} & k_{2,2} & k_{2,3} \\
        k_{3,0} & k_{3,1} & k_{3,2} & k_{3,3}
    \end{array} \right] =
\] \[
    = \left[ \begin{array}{cccc}
        b_{0,0} & b_{0,1} & b_{0,2} & b_{0,3} \\
        b_{1,0} & b_{1,1} & b_{1,2} & b_{1,3} \\
        b_{2,0} & b_{2,1} & b_{2,2} & b_{2,3} \\
        b_{3,0} & b_{3,1} & b_{3,2} & b_{3,3}
    \end{array} \right].
\]


\subsection{Процедура расширения ключа}

Матрица ключа текущего раунда получается из исходного ключа шифра с помощью специальной процедуры, состоящей из расширения ключа и выбора раундового ключа. Основные принципы этой процедуры состоят в следующем.
\begin{itemize}
    \item Суммарная длина ключей всех раундов равна длине блока, умноженной на увеличенное на 1 число раундов. Для блока длины 128 бит и 10 раундов общая длина всех ключей раундов равна 1408.
    \item С помощью ключа шифра находят \textit{расширенный ключ}.
    \item Ключи \emph{раунда} выбираются из \emph{расширенного} ключа по правилу: ключ первого раунда состоит из первых 4-х столбцов матрицы расширенного ключа, второй ключ –- из следующих 4-х столбцов и т.д.
\end{itemize}

Расширенный ключ –- это матрица $\mathsf{W}$, состоящая из $4 \cdot (\mathsf{Nr} + 1)$ 4-байтовых вектор-столбцов, каждый столбец $i$ обозначается $\mathsf{W}[i]$.

Далее рассматривается только случай, когда ключ шифра состоит из $16$ байтов. Первые $\mathsf{Nk} = 4$ столбцов содержат ключ шифра. Остальные столбцы вычисляются рекурсивно из столбцов с меньшими номерами.

Для $\mathsf{Nk} = 4$ имеем 16-байтовый ключ
\[
    \mathsf{Key} = (\mathsf{Key}[0], \mathsf{Key}[1], \dots, \mathsf{Key}[15]).
\]
Приведем алгоритм расширения ключа для $\mathsf{Nk} = 4$.
\begin{algorithm}[iht]
    \caption{$\mathsf{KeyExpansion}(\mathsf{Key}, \mathsf{W})$\label{alg:AES-key-exp}}
    \begin{algorithmic}
        \FOR{ $i=0$ \TO $\mathsf{Nk} - 1$}
            \STATE $\mathsf{W}[i] = (\mathsf{Key}[4i], ~ \mathsf{Key}[4i+1], ~ \mathsf{Key}[4i+2], ~ \mathsf{Key}[4i+3])^T$;
        \ENDFOR
        \FOR{ $i = \mathsf{Nk}$ \TO $4 \cdot (\mathsf{Nr} + 1) - 1$}
            \STATE $\mathsf{temp} = \mathsf{W}[i-1]$;
            \IF{ ($i = 0 \mod \mathsf{Nk}$)}
                \STATE $\mathsf{temp} = \mathsf{SubWord}(\mathsf{RotWord}(\mathsf{temp})) ~ \oplus ~ \mathsf{Rcon}[i / \mathsf{Nk}]$;
            \ENDIF
            \STATE $\mathsf{W}[i] = \mathsf{W}[i - \mathsf{Nk}] ~ \oplus ~ \mathsf{temp}$;
        \ENDFOR
    \end{algorithmic}
\end{algorithm}

%\[
%    \begin{array}{l}
%        \mathsf{KeyExpansion}(\mathsf{Key}, \mathsf{W}) \{ \\
%        ~~~~ \mathsf{for ~ (i = 0; ~ i < Nk = 4; ~ i++)} \\
%        ~~~~~~~~ \mathsf{W[i] = (Key[4 \cdot i], ~ Key[4*i+1], ~ Key[4*i+2], ~ Key[4*i+3]);} \\
%        ~~~~ \mathsf{for ~ (i = Nk; ~ i < 4 * (Nr + 1); ~ i++)} ~ \{ \\
%        ~~~~~~~~ \mathsf{temp = W[i-1];} \\
%        ~~~~~~~~ \mathsf{if ~ (i ~ \% ~ Nk ~ == ~ 0)} \\
%        ~~~~~~~~~~~~ \mathsf{temp = SubWord(RotWord(temp))} ~ \oplus ~ \mathsf{Rcon[i / Nk];} \\
%        ~~~~~~~~ \mathsf{W[i] = W[i - Nk]} ~ \oplus ~ \mathsf{temp;} \\
%        ~~~~ \} \\
%        \} \\
%    \end{array}
%\]

Здесь $\mathsf{SubWord}(\mathsf{W})[i]$ обозначает функцию, которая применяет операцию <<Замена байтов>> (или s-блок) $\mathsf{SubBytes}$ к каждому из 4-х байтов столбца $\mathsf{W}[i]$. Функция $\mathsf{RotWord}(\mathsf{W}[i])$  осуществляет циклический сдвиг вверх байтов столбца $\mathsf{W}[i]$: если $\mathsf{W}[i] = (a, b, c, d)^T$, то $\mathsf{RotByte}(\mathsf{W}[i]) = (b, c, d, a)^T$. Векторы-константы $\mathsf{Rcon}[i]$ определены ниже.

Как видно из этого описания, первые $\mathsf{Nk} = 4$ столбцов заполняются ключом шифра. Все следующие столбцы $\mathsf{W}[i]$ равны сумме по модулю 2 предыдущего столбца $\mathsf{W}[i-1]$ и столбца $\mathsf{W}[i-4]$. Для столбцов $\mathsf{W}[i]$ с номерами $i$, кратными $\mathsf{Nk} = 4$, к столбцу $\mathsf{W}[i-1]$ применяются операции $\mathsf{RotWord(W)}$ и $\mathsf{SubWord(W)}$, а затем производится суммирование по модулю 2 со столбцом $\mathsf{W}[i-4]$ и константой раунда $\mathsf{Rcon}[i ~/~ 4]$.

%Для $\mathsf{Nk}>6$ имеем
%\[
%\begin{array}{l}
% \mathsf{KeyExpansion\,(byte\,Key\,[4*Nk]\,\, word \,\, W[Nb*(Nr+1)])}\\
%  \{\\
% \quad\quad \mathsf{for\,\,(i=0;\,\, i<Nk;\,\,i++)} \\
%  \qquad \quad\quad\quad \mathsf{W[i]=(Key[4*i];Key[4*i+1];Key[4*i+2];Key[4*i+3]);}\\
%  \quad\quad \mathsf{for \,\,(i=Nk;\,\,i<Nb*(Nr+1);\,\,i++)}\\
%  \quad\quad \{ \\
%  \quad \quad\quad\quad \mathsf{temp=W[i-1]}; \\
%  \quad \quad\quad\quad \mathsf{if\,\,(i\quad\% \quad Nk==0)}\\
%  \qquad \qquad \qquad \quad \mathsf{temp=SubByte(RotByte(temp))\quad\widehat{\,}\quad Rcon[i/Nk]};\\
%\quad \quad\quad\quad \mathsf{else \,\,if\,\,(i\quad\% \quad Nk==4)}\\
% \qquad \qquad \qquad \quad \mathsf{temp=SubByte(temp)};\\
%  \quad \quad\quad\quad \mathsf{W[i]=W[i-Nk] \quad\widehat{\,}\quad temp};\\
%  \quad\quad \} \\
%  \}\, \\
%\end{array}
%\]
%Различие между этими двумя случаями состоит в том, что во втором случае к столбцу $\mathsf{W[i-1]}$ применяются операции
% $\mathsf{RotByte(W)}$ и $\mathsf{SubByte(W)}$, если $\mathsf{i-4}$ кратно $\mathsf{Nk}$.\\

Константы раундов определяются следующим образом:
    \[ \mathsf{Rcon}[i] = (\mathsf{RC}[i], \mathrm{'00'}, \mathrm{'00'}, \mathrm{'00'})^T, \]
где байт $\mathsf{RC}[1]$ равен $\mathsf{RC}[1] = \mathrm{'01'}$, а байты $\mathsf{RC}[i] = \alpha^{i-1}, ~ i = 2, 3, \dots$. Байт $\alpha = \mathrm{'02'}$ –- это примитивный элемент поля $\GF{2^8}$.

\example
Пусть $\mathsf{Nk} = 4$. В этом случае ключ шифра имеет длину 128 бит. Найдем столбцы расширенного ключа. Столбцы $\mathsf{W}[0], \mathsf{W}[1], \mathsf{W}[2], \mathsf{W}[3]$ непосредственно заполняются битами ключа шифра. Номер следующего столбца $\mathsf{W}[4]$ кратен $\mathsf{Nk}$, поэтому
\[
    \mathsf{W}[4] = \mathsf{SubWord}(\mathsf{RotWord}(\mathsf{W}[3])) \oplus \mathsf{W}[0] \oplus
        \left[ \begin{array}{c}
            \mathrm{'01'} \\ \mathrm{'00'} \\ \mathrm{'00'} \\ \mathrm{'00'} \\
        \end{array} \right].
\]
Далее имеем
\[
    \begin{array}{l}
        \mathsf{W}[5] = \mathsf{W}[4] \oplus \mathsf{W}[1], \\
        \mathsf{W}[6] = \mathsf{W}[5] \oplus \mathsf{W}[2], \\
        \mathsf{W}[7] = \mathsf{W}[6] \oplus \mathsf{W}[3].  \\
    \end{array}
\]
Затем
\[
    \mathsf{W}[8] = \mathsf{SubWord}(\mathsf{RotWord}(\mathsf{W}[7])) \oplus \mathsf{W}[4] \oplus
        \left[ \begin{array}{c}
            \alpha \\
            \mathrm{'00'}\\
            \mathrm{'00'}\\
            \mathrm{'00'}\\
        \end{array} \right] ,
\] \[
    \begin{array}{l}
        \mathsf{W}[9] = \mathsf{W}[8] \oplus \mathsf{W}[5], \\
        \mathsf{W}[10] = \mathsf{W}[9] \oplus \mathsf{W}[6], \\
        \mathsf{W}[11] = \mathsf{W}[10] \oplus \mathsf{W}[7] \\
    \end{array}
\]
и т.д.
\exampleend

%\example
%Пусть $\mathsf{Nk=6}.$ В этом случае ключ шифра имеет длину 192 бита. Найдем столбцы расширенного ключа. Столбцы $\mathsf{W[0],W[1],W[2],W[3],W[4],W[5]}$ непосредственно заполняются
%битами ключа шифра. Номер следующего столбца $\mathsf{W[6]}$ кратен $\mathsf{Nk}$, поэтому
%\[
%\begin{array}{ccccccc}
% \mathsf{W[6]} & = & \mathsf{SubByte(RotByte(W[5]))} &\oplus  &  \mathsf{W[0]} & \oplus  & \left[ \begin{array}{c}
% \mathsf{`01'} \\
%  \mathsf{`00'}\\
%  \mathsf{`00'}\\
%  \mathsf{`00'}\\
%\end{array}
%\right]    \\
%\end{array}
%\].
%
%Далее имеем
%\[
%\begin{array}{ccc}
% \mathsf{W[7]=W[6]}\oplus \mathsf{W[1]}; & \mathsf{W[8]=W[7]}\oplus \mathsf{W[2]}; & \mathsf{W[9]=W[8]}\oplus \mathsf{W[3]}; \\
% \mathsf{W[10]=W[9]}\oplus \mathsf{W[4]}; &\mathsf{ W[11]=W[10]}\oplus \mathsf{W[5]}.\\
%\end{array}
%\]
%Затем
%\[
%\begin{array}{ccccccc}
% \mathsf{W[12]} & = & \mathsf{SubByte(RotByte(W[11]))} &\oplus  &  \mathsf{W[6]} & \oplus  & \left[ \begin{array}{c}
% \mathsf{\alpha} \\
%  \mathsf{`00'}\\
%  \mathsf{`00'}\\
%  \mathsf{`00'}\\
%\end{array}
%\right] ,   \\
%\end{array}
%\]
%\[
%\begin{array}{ccc}
% \mathsf{W[13]=W[12]}\oplus \mathsf{W[7]}; & \mathsf{W[14]= W[13]}\oplus \mathsf{W[8]};  & \mathsf{W[15]=W[14]}\oplus \mathsf{W[9]},  \\
%\end{array}
%\]
%и т.д.
%\exampleend
%
%\example
%Пусть $\mathsf{Nk=8}.$ В этом случае ключ шифра имеет длину $256$ бита. Найдем столбцы расширенного ключа. Столбцы
%$\mathsf{W[0],W[1],W[2],W[3],W[4],W[5],W[6],W[7]}$  непосредственно заполняются битами ключа шифра. Номер следующего столбца
%$\mathsf{W[8]}$ кратен $\mathsf{Nk}$, поэтому
%\[
%\begin{array}{ccccccc}
% \mathsf{W[8]} & = & \mathsf{SubByte(RotByte(W[7]))} &\oplus  &  \mathsf{W[0]} & \oplus  & \left[ \begin{array}{c}
% \mathsf{`01'} \\
%  \mathsf{`00'}\\
%  \mathsf{`00'}\\
%  \mathsf{`00'}\\
%\end{array}
%\right]    \\
%\end{array}
%\].
%Далее имеем
%\[
%\begin{array}{ccc}
%\mathsf{ W[7]=W[6]}\oplus \mathsf{W[1]}; & \mathsf{W[8]=W[7]}\oplus \mathsf{W[2]}; & \mathsf{W[9]=W[8]}\oplus \mathsf{W[3]}; \\
%\mathsf{ W[10]=W[9]}\oplus \mathsf{W[4]}; & \mathsf{W[11]=W[10]}\oplus \mathsf{W[5]}.\\
%\end{array}
%\]
%Номер следующего столбца $\mathsf{W[12]}$ равен $12$. Так как $12-4$ кратно $\mathsf{Nk}$, то
%\[
%\begin{array}{ccc}
%\mathsf{ W[12]=SubByte(RotByte(W[11]))}\oplus \mathsf{W[4]}; & \mathsf{W[13]=W[12]}\oplus \mathsf{W[5]}; & \mathsf{W[14]=W[13]}\oplus \mathsf{W[6]}; \\
%\mathsf{ W[15]=W[14]}\oplus \mathsf{W[7]}. &  &\\
%\end{array}
%\]
%Затем
%\[
%\begin{array}{ccccccc}
% \mathsf{W[16]} & = & \mathsf{SubByte(RotByte(W[15]))} &\oplus  &  \mathsf{W[8]} & \oplus  & \left[ \begin{array}{c}
% \mathsf{\alpha} \\
%  \mathsf{`00'}\\
%  \mathsf{`00'}\\
%  \mathsf{`00'}\\
%\end{array}
%\right] ,   \\
%\end{array}
%\]
%\[
%\begin{array}{ccc}
% \mathsf{W[17]=W[16]}\oplus \mathsf{W[9]}; & \mathsf{W[18]=W[17]}\oplus \mathsf{W[10]};  &\mathsf{ W[19]=W[18]}\oplus \mathsf{W[10]}, \\
%\end{array}
%\]
%
%\[
%\begin{array}{ccc}
%\mathsf{ W[20]=SubByte(RotByte(W[19]))}\oplus \mathsf{W[12]}; & \mathsf{W[21]=W[20]}\oplus \mathsf{W[13]}; & \mathsf{W[22]=W[21]}\oplus \mathsf{W[14]}; \\
%\mathsf{ W[23]=W[22]}\oplus \mathsf{W[15]}, &  &\\
%\end{array}
%\]
%и т.д.

Ключ $i$-го раунда состоит из столбцов матрицы расширенного ключа
\[
    \mathsf{RoundKey} = (\mathsf{W}[4(i-1)], \mathsf{W}[4(i-1) + 1], \ldots, \mathsf{W}[4i-1]).
\]
%Если длина блока равна 192 битам $Nb=6$, то ключ 5-го раунда состоит из столбцов $W[24],W[25],W[26],W[27],W[28],W[29].$
%\exampleend

В настоящее время американский стандарт шифрования AES де-факто используется международно в негосударственных системах передачи данных, если позволяет законодательство страны. C 2010 г. процессоры Intel поддерживают специальный набор инструкций для шифра AES.


\section{Режимы работы блоковых шифров}\label{chapter-block-chaining}
\selectlanguage{russian}

Открытый текст $M$, представленный как двоичный файл, перед шифрованием разбивают на части $M_1, M_2, \dots, M_n$, называемые пакетами. Предполагается, что размер в битах каждого пакета существенно превосходит длину блока шифрования, которая равна 64 бит для российского стандарта и 128 для американского стандарта AES.

В свою очередь, каждый пакет $M_i$ разбивается на блоки размера, равного размеру блока шифрования:
    \[ M_i = \left[ M_{i,1}, M_{i,2}, \dots, M_{i,n_i} \right]. \]
Число блоков $n_i$ в разных пакетах может быть разным. Кроме того, последний блок пакета $M_{i,n_i}$ может иметь размер, меньший размера блока шифрования. В этом случае для него применяют процедуру дополнения (удлинения) до стандартного размера. Процедура должна быть обратимой: после расшифрования последнего блока пакета лишние байты должны быть обнаружены и удалены. Некоторые способы дополнения:
\begin{itemize}
  \item добавить один байт со значением $128$, а остальные байты пусть будут нулевые;
  \item определить, сколько байтов надо добавить к последнему блоку, например $b$, и добавить $b$ байтов со значением $b$ в каждом.
\end{itemize}
В дальнейшем предполагается, что такое дополнение сделано для каждого пакета. При шифровании блоков внутри одного пакета первый индекс в нумерации блоков опускается, то есть вместо обозначения $M_{i,j}$ используется $M_j$.

Для шифрования всего открытого текста $M$ и, следовательно, всех пакетов используется один и тот же  \emph{сеансовый} ключ шифрования  $K$. Процедуру передачи одного пакета будем называть \emph{сеансом}.

Существует несколько режимов работы блоковых шифров: режим электронной кодовой книги, режим шифрования зацепленных блоков, режим обратной связи, режим шифрованной обратной связи, режим счетчика. Рассмотрим особенности каждого из этих режимов.


\subsection{Электронная кодовая книга}

В режиме электронной кодовой книги (аббревиатура ECB от английского названия Electronic Code Book) открытый текст в пакете разделен на блоки
    \[ \left[ M_1, M_2, \dots, M_{n-1}, M_n \right]. \]

В процессе шифрования каждому блоку $M_j$ соответствует свой шифротекст $C_j$, определяемый с помощью ключа $K$:
    \[ C_j = E_K(M_j), ~ j = 1, 2, \dots, n. \]

Если в открытом тексте есть одинаковые блоки, то в шифрованном тексте им также соответствуют одинаковые блоки. Это дает дополнительную информацию для криптоаналитика, что является недостатком этого режима. Другой недостаток состоит в том, что криптоаналитик может подслушивать, перехватывать, переставлять, воспроизводить ранее записанные блоки, нарушая конфиденциальность и целостность информации. Поэтому при работе в режиме электронной кодовой книги нужно вводить аутентификацию сообщений.

Шифрование в режиме электронной кодовой книги не использует сцепление блоков и синхропосылку\index{синхропосылка} (вектор инициализации)\index{вектор инициализации}. Поэтому для данного режима применима атака на различение сообщений, так как два одинаковых блока или два одинаковых открытых текста шифруются одинаково.

На рис. \ref{fig:ecb-demo} приведен пример шифрования графического файла морской звезды в формате BMP, 24 бит цветности на пиксел (рис. \ref{fig:starfish}), блоковым шифром AES с длиной ключа 128 бит в режиме электронной кодовой книги  (рис. \ref{fig:starfish-aes-128-ecb}). В начале зашифрованного файла был восстановлен стандартный заголовок формата BMP. Как видно, в зашифрованном файле изображение все равно различимо.
\begin{figure}[!ht]
    \centering
    \subfloat[Исходный рисунок]{\label{fig:starfish} \includegraphics[width=0.45\textwidth]{pic/starfish}}
    ~~~
    \subfloat[Рисунок, зашифрованный AES-128]{\label{fig:starfish-aes-128-ecb} \includegraphics[width=0.45\textwidth]{pic/starfish-aes-128-ecb}}
    \caption{Шифрование в режиме электронной кодовой книги\label{fig:ecb-demo}}
\end{figure}
BMP файл в данном случае содержит в самом начале стандартный заголовок (ширина, высота, количество цветов) и далее идет массив 24-битных значений цвета пикселов, взятых построчно сверху вниз. В массиве много последовательностей нулевых байтов, так как пикселы белого фона кодируются 3 нулевыми байтами. В AES размер блока равен 16 байтов и, значит, каждые $\frac{16}{3}$ подряд идущих пикселов белого фона шифруются одинаково, позволяя различить изображение в зашифрованном файле.

%На рис. \ref{fig:ecb-demo} приведен пример шифрования графического файла логотипа Википедии в формате BMP, 24 бит цветности на пиксел (рис. \ref{fig:wikilogo}), блоковым шифром AES с длиной ключа 128 бит в режиме электронной кодовой книги  (рис. \ref{fig:wikilogo-aes-128-ecb}). В начале зашифрованного файла был восстановлен стандартный заголовок BMP формата. Как видно, на зашифрованном рисунке возможно даже прочитать надпись.
%\begin{figure}[!ht]
%    \centering
%    \subfloat[Исходный рисунок]{\label{fig:wikilogo}\includegraphics[width=0.45\textwidth]{pic/wikilogo}}
%    ~~~
%    \subfloat[Рисунок, зашифрованный AES-128]{\label{fig:wikilogo-aes-128-ecb}\includegraphics[width=0.45\textwidth]{pic/wikilogo-aes-128-ecb}}
%    \caption{Шифрование в режиме электронной кодовой книги.}
%    \label{fig:ecb-demo}
%\end{figure}

%Возможно воссоздание структуры информации -- например, пингвин на рис. \ref{fig:tux-ecbmode}. Картинка с пингвином записана в формате BMP и зашифрована DES в режиме электронной кодовой книги.
%\begin{figure}[!ht]
%    \centering
%    \includegraphics[width=0.3\textwidth]{pic/tux-ecb}
%    \caption{Картинка с пингвином, зашифрованная в режиме электронной кодовой книги.}
%    \label{fig:tux-ecbmode}
%\end{figure}


\subsection{Сцепление блоков шифротекста}

В режиме сцепления блоков шифротекста (аббревиатура CBC от английского названия Cipher Block Chaining) перед шифрованием текущего блока открытого текста предварительно производится его суммирование по модулю 2 с предыдущим блоком зашифрованного текста, что и осуществляет <<сцепление>> блоков. Процедура шифрования имеет вид
\[ \begin{array}{l}
    C_1 = E_K(M_1 \oplus C_0), \\
    C_j = E_K(M_j \oplus C_{j-1}), ~ j = 1, 2, \dots,  n,
\end{array} \]
где $C_0 = \textrm{IV}$ --  вектор, называемый вектором инициализации (обозначение $\textrm{IV}$ от Initialization Vector). Другое название -- синхропосылка.

Благодаря сцеплению \emph{одинаковым} блокам открытого текста соответствуют \emph{различные} шифрованные блоки. Это затрудняет криптоаналитику статистический анализ потока шифрованных блоков.

На приемной стороне расшифрование осуществляется по правилу
\[ \begin{array}{l}
    D_K(C_j) = M_j \oplus C_{j-1}, ~ j=1, 2, \dots, n,\\
    M_{j} = D_K(C_j) \oplus C_{j-1}.
\end{array} \]

Блок $C_0 = \textrm{IV}$ должен быть известен легальному получателю шифрованных сообщений. Обычно криптограф выбирает его случайно и вставляет на первое место в поток шифрованных блоков. Сначала передают блок $C_0$, а затем шифрованные блоки $C_1, C_2, \ldots, C_n$.

В разных пакетах блоки $C_0$ должны выбираться независимо. Если их выбрать одинаковыми, то возникают проблемы, аналогичные проблемам в режиме ECB. Например, часто первые нешифрованные блоки $M_1$ в разных пакетах бывают одинаковыми. Тогда одинаковыми будут и первые шифрованные блоки.

Однако случайный выбор векторов инициализации также имеет свои недостатки. Для выбора такого вектора необходим хороший генератор случайных чисел. Кроме того, каждый пакет удлиняется на один блок.

Нужны такие процедуры выбора $C_0$ для каждого сеанса передачи пакета, которые известны криптографу и легальному пользователю. Одним из решений является использование так называемых \emph{одноразовых меток}. Каждому сеансу присваивается уникальное число. Его уникальность состоит в том, что оно используется только один раз и никогда не должно повторяться в других пакетах. В англоязычной научной литературе оно обозначается как \emph{Nonce}, то есть сокращение от <<Number used once>>\index{одноразовая метка}.

Обычно одноразовая метка состоит из номера сеанса и дополнительных данных, обеспечивающих уникальность. Например, при двустороннем обмене шифрованными сообщениями одноразовая метка может состоять из номера сеанса и индикатора направления передачи. Размер одноразовой метки должен быть равен размеру шифруемого блока. После определения одноразовой метки $\textrm{Nonce}$ вектор инициализации вычисляется в виде
    \[ C_0 = \textrm{IV} = E_K(\textrm{Nonce}). \]

Этот вектор используется в данном сеансе для шифрования открытого текста в режиме CBC. Заметим, что блок $C_0$ передавать в сеансе не обязательно, если приемная сторона знает заранее дополнительные данные для одноразовой метки. Вместо этого достаточно вначале передать только номер сеанса в открытом виде. Приемная сторона добавляет к нему дополнительные данные и вычисляет блок $C_0$, необходимый для расшифрования в режиме CBC. Это позволяет сократить издержки, связанные с удлинением пакета. Например, для шифра AES длина блока $C_0$ равна $16$ байтов. Если число сеансов ограничить величиной $2^{32}$ (вполне приемлемой для большинства приложений), то для передачи номера пакета понадобится только $4$ байта.


\subsection{Обратная связь по выходу}

В предыдущих режимах входными блоками для устройств шифрования были непосредственно блоки открытого текста.
В режиме обратной связи по выходу (OFB от Output FeedBack) блоки открытого текста непосредственно на вход устройства шифрования не поступают. Вместо этого устройство шифрования генерирует псевдослучайный поток байтов, который суммируется по модулю $2$ с открытым текстом для получения шифрованного текста. Шифрование осуществляют по правилу
\[ \begin{array}{l}
    K_0 = \textrm{IV}, \\
    K_j = E_K(K_{j-1}), ~ j = 1, 2, \dots, n, \\
    C_j = K_j \oplus M_j.
\end{array} \]

Здесь текущий ключ $K_j$ есть результат шифрования предыдущего ключа $K_{j-1}$. Начальное значение $K_0$ известно криптографу и легальному пользователю. На приемной стороне расшифрование выполняют по правилу
\[ \begin{array}{l}
    K_0 = \textrm{IV}, \\
    K_j = E_K(K_{j-1}), ~ j = 1, 2, \dots, n, \\
    M_j = K_j \oplus C_j.
\end{array} \]

Как и в режиме CBC, вектор инициализации $\textrm{IV}$ может быть выбран случайно и передан вместе с шифрованным текстом либо вычислен на основе одноразовых меток. Здесь особенно важна уникальность вектора инициализации.

Достоинство этого режима состоит в полном совпадении операций шифрования и расшифрования. Кроме того, в этом режиме не надо проводить операцию дополнения открытого текста.


\subsection{Обратная связь по шифрованному тексту}

В режиме обратной связи по шифрованному тексту (CFB от Cipher FeedBack) ключ $K_j$ получается с помощью процедуры шифрования предыдущего шифрованного блока $C_{j-1}$. Может быть использован не весь блок $C_{j-1}$, а только часть его. Как и в предыдущем случае, начальное значение ключа $K_0$ известно криптографу и легальному пользователю:
\[ \begin{array}{l}
    K_0 = \textrm{IV}, \\
    K_j = E_K(C_{j-1}), ~ j = 1, 2, \dots, n,\\
    C_j = K_j \oplus M_j.
\end{array} \]

У этого режима нет особых преимуществ по сравнению с другими режимами.


\subsection{Счетчик}

В режиме счетчика (CTR от Counter) правило шифрования имеет вид, похожий на режим обратной связи по выходу (OFB), но позволяющий вести независимое (параллельное) шифрование и расшифрование блоков:
\[ \begin{array}{l}
    K_j = E_K(\textrm{Nonce} \| j - 1), ~ j = 1, 2, \dots, n, \\
    C_j = M_j \oplus K_j,
\end{array} \]
где $\textrm{Nonce} \| j - 1$ -- конкатенация битной строки одноразовой метки $\textrm{Nonce}$ и номера блока уменьшенного на единицу $j-1$.
%Для стандарта AES значение $\textrm{Nonce}$ занимает 16 бит, номер блока -- 48 бит. С одним ключом выполняется шифрование $2^{48}$ блоков.

Правило расшифрования идентичное:
\[ \begin{array}{l}
    M_j = C_j \oplus K_j. \\
\end{array} \]


\section{Некоторые свойства блоковых шифров}

\input{feistel_network_reversibility}

\subsection{Схема Фейстеля без s-блоков}
\selectlanguage{russian}

Пусть функция $F$ является простой линейной комбинацией некоторых бит правой части и ключа раунда относительно операции XOR. Тогда можно записать систему линейных уравнений битов выхода $y_i$ относительно битов входа $x_i$ и ключа $k_i$ после всех 16 раундов в виде
    \[ y_i = (\sum_{i=0}^{n_1} a_i x_i) \oplus (\sum_{i=0}^{n_2} b_i k_i), \]
где суммирование производится по модулю 2, коэффициенты $a_i$ и $b_i$ известны и равны 0, 1, количество бит в блоке открытого текста равно $n_1$, количество битов ключа равно $n_2$.

Имея открытый текст и шифротекст, легко найти ключ. Без знания открытых текстов, выполняя XOR шифротекстов, найдем XOR открытых текстов. Во-первых, это атака на различение сообщений. Во-вторых, часто известны форматы сообщений, отдельные поля или распределение символов открытого текста, что приводит к атаке перебором с учетом множества уравнений, полученных XOR шифротекстов.

$s$-блоки замены создают нелинейность в уравнениях выхода $y_i$ относительно сообщения и ключа.


\subsubsection[Схема Фейстеля в ГОСТ 28147-89 без $s$-блоков]{Схема Фейстеля в~ГОСТ~28147-89 без~$s$-блоков}

В отличие от устаревшего алгоритма DES блоковый шифр ГОСТ без $s$-блоков намного сложнее для взлома, так как для него нельзя записать систему линейных уравнений:
\[
    \begin{array}{l}
        L_1 = R_0, \\
        R_1 = L_0 \oplus ((R_0 \boxplus K_1) \lll 11), \\
    \end{array}
\] \[
    \begin{array}{l}
        L_2 = R_1 = L_0 \oplus ((R_0 \boxplus K_1) \lll 11), \\
        R_2 = L_1 \oplus (R_1 \boxplus K_2)  = \\
        ~~~~~= R_0 \oplus (((L_0 \oplus ((R_0 \boxplus K_1) \lll 11)) \boxplus K_2) \lll 11). \\
    \end{array}
\]

Операция $\boxplus$ нелинейна по XOR. Например, только на трех операциях $\oplus$, $\boxplus$ и $\lll f(R_i)$ без использования $s$-блоков построен блоковый шифр RC5, который по состоянию на 2010 г. не был взломан.


\subsection{Лавинный эффект}
\selectlanguage{russian}

\subsubsection{Лавинный эффект в DES}

Оценим число раундов, за которое в DES достигается полный лавинный эффект\index{лавинный эффект}, предполагая \emph{случайное} расположение бит перед расширением, $s$-блоками ($s$ -- substitute, блоки замены) и XOR.

Пусть на входе правой части $R_i$ содержится $r$ бит, на которые уже распространилось влияние 1 вначале выбранного бита. После расширения получим
    \[ n_1 \approx \min(1.5 \cdot r, 32) \]
зависимых бит. Предполагая случайные попадания в 8 $s$-блоков, мы увидим, что, согласно задаче о размещении, биты попадут в
    \[ s_2 = 8 \left( 1 - \left( 1 - \frac{1}{8} \right)^{n_1} \right) \approx 8 \left( 1 - e^{-\frac{n_1}{8}} \right) \]
$s$-блоков. Одно из требований NSA к $s$-блокам заключалось в том, чтобы изменение каждого бита входа \emph{изменяло} 2 бита выхода. Мы предположим, что каждый бит входа $s$-блока \emph{влияет} на все 4 бита выхода. Зависимыми станут
    \[ n_2 = 4 \cdot s_2 = 32 \left( 1 - e^{-\frac{n_1}{8}} \right) \]
бит. При дальнейшем XOR с величиной $L_i$, содержащей $l$ зависимых бит, результатом будет
    \[ n_3 \approx n_2 + l  - \frac{n_2 l}{32} \]
зависимых бит.

\begin{table}[!ht]
    \centering
    \caption{Распространение влияния 1 бита левой части в DES\label{tab-DES-avalance-effect}}
    \begin{tabular}{||c||c||c|c|c||}
        \hline
        \multirow{3}{*}{Раунд} & $L_i$ & \multicolumn{3}{|c||}{$R_i$} \\
        \cline{2-5}
        & & Расширение & $s$-блоки & $R_{i+1} = f(R_i) \oplus L_i$ \\
        & $l$ & $r \rightarrow n_1$ & $n_1 \rightarrow n_2$ & $(n_2, l) \rightarrow n_3$ \\
        \hline \hline
        0 & 1 & 0 & 0 & 0 \\
        1 & 0 & 0 & 0 & $(0,1) \rightarrow 1$ \\
        2 & 1 & $1 \rightarrow 1.5$ & $1.5 \rightarrow 5.5$ & $(5.5, 0) \rightarrow 5.5$ \\
        3 & 5.5 & $5.5 \rightarrow 8.2$ & $8.2 \rightarrow 20.5$ & $(20.5, 1) \rightarrow 20.9$ \\
        4 & 20.9 & $20.9 \rightarrow 31.3$ & $31.3 \rightarrow 32$ & $(32, 20.9) \rightarrow 32$ \\
        5 & 32 & 32 & 32 & 32 \\
      \hline
    \end{tabular}
\end{table}

В таблице \ref{tab-DES-avalance-effect} приводится расчет распространения 1 бита левой части. Посчитано число зависимых битов по раундам в предположении об их случайном расположении и том, что каждый бит на входе $s$-блока \emph{влияет} на все биты выхода. Полная диффузия достигается за 5 раундов, что совпадает с экспериментальной проверкой. Для достижения максимального лавинного эффекта требуется аккуратно выбрать расширение, $s$-блоки, а также перестановку в функции $F$.


\subsubsection{Лавинный эффект в ГОСТ 28147-89}

Лавинный эффект\index{лавинный эффект} по входу обеспечивается $(4 \times 4)$ $s$-блоками и циклическим сдвигом влево на $11 \neq 0 \mod 4$.

\begin{table}[!ht]
    \centering
    \caption{Распространение влияния 1 бита левой части в ГОСТ 28147-89\label{tab:GOST-avalance-effect}}
    \resizebox{\textwidth}{!}{ \begin{tabular}{||c||c|c|c|c|c|c|c|c||c|c|c|c|c|c|c|c||}
        \hline
        \multirow{2}{*}{Раунд} & \multicolumn{8}{|c||}{$L_i$} & \multicolumn{8}{|c||}{$R_i$} \\
        \cline{2-17}
              & 1 & 2 & 3 & 4 & 5 & 6 & 7 & 8   &   1 & 2 & 3 & 4 & 5 & 6 & 7 & 8 \\
        \hline \hline
        0     &   &   &   &   &   &   &   & 1   &     &   &   &   &   &   &   &   \\
        1     &   &   &   &   &   &   &   &     &     &   &   &   &   &   &   & 1 \\
        2     &   &   &   &   &   &   &   & 1   &     &   &   &   & 3 & 1 &   &   \\
        3     &   &   &   &   & 3 & 1 &   &     &     & 3 & 4 & 1 &   &   &   & 1 \\
        4     &   & 3 & 4 & 1 &   &   &   & 1   &   4 & 1 &   &   & 3 & 1 & 3 & 4 \\
        5     & 4 & 1 &   &   & 3 & 1 & 3 & 4   &     & 3 & 4 & 4 & 4 & 4 & 4 & 1 \\
        6     &   & 3 & 4 & 4 & 4 & 4 & 4 & 1   &   4 & 4 & 4 & 4 & 4 & 3 & 3 & 4 \\
        7     & 4 & 4 & 4 & 4 & 4 & 3 & 3 & 4   &   4 & 4 & 4 & 4 & 4 & 4 & 4 & 4 \\
        8     & 4 & 4 & 4 & 4 & 4 & 4 & 4 & 4   &   4 & 4 & 4 & 4 & 4 & 4 & 4 & 4 \\
      \hline
    \end{tabular} }
\end{table}

Из таблицы \ref{tab:GOST-avalance-effect} видно, что на каждом раунде число зависимых бит увеличивается в среднем на 4 в результате сдвига и попадания выхода $s$-блока предыдущего раунда в два $s$-блока следующего раунда. Показано распространение зависимых битов в группах по 4 бита в левой и правой частях без учета сложения с ключом раунда. Предполагается, что каждый бит на входе $s$-блока влияет на все биты выхода. Число раундов для достижения полного лавинного эффекта без учета сложения с ключом -- 8. Экспериментальная проверка для $s$-блоков, используемых Центробанком РФ, показывает, что требуется 8 раундов.


\subsubsection{Лавинный эффект в AES}

В первом раунде один бит оказывает влияние на один байт в операции <<Замена байтов>> и затем на столбец из четырех байтов в операции <<Смешивание столбцов>>\index{лавинный эффект}.

Во втором раунде операция <<Сдвиг строки>> сдвигает байты измененного столбца на разное число байтов по строкам, в результате получаем диагональное расположение измененных байт, то есть в каждой строке присутствует по измененному байту. Далее, в результате операции <<Смешивание столбцов>> изменение распространяется от байта в столбце на весь столбец и, следовательно, на всю матрицу.

Диффузия по входу достигается за 2 раунда.


\input{double_and_triple_ciphering}

\input{stream-ciphers}

\input{hash-functions}

\chapter{Криптосистемы с открытым ключом}\label{chapter-public-key}
\selectlanguage{russian}

\textbf{Криптосистемой с открытым ключом} (Public-Key Cryptosystem, PKC) называется криптографическое преобразование, использующее два ключа -- открытый и секретный. Пара из \textbf{закрытого}\index{ключ!закрытый} (Private Key, Secret Key)\footnote{В контексте криптосистем с открытым ключом можно ещё встретить использование термина <<секретный ключ>>. Мы не рекомендуем использовать данный термин, чтобы не путать с секретным ключом\index{ключ!секретный}, используемым в симметричных криптосистемах} и \textbf{открытого}\index{ключ!открытый} (Public Key, PK) ключей создается пользователем, который свой закрытый ключ держит в секрете, а открытый ключ делает общедоступным для всех пользователей. Криптографическое преобразование в одну сторону (шифрование) можно выполнить зная только открытый ключ, а в другую (расшифрование) -- только зная секретный ключ. Во многих криптосистемах из закрытого ключа теоретически можно вычислить открытый ключ, однако это является сложной вычислительной задачей.

Если прямое преобразование выполняется открытым ключом, а обратное -- закрытым, то криптосистема называется \textbf{схемой шифрования с открытым ключом}. Все пользователи, зная открытый ключ получателя, могут зашифровать для него сообщение, которое может расшифровать только владелец закрытого ключа.

Если прямое преобразование выполняется закрытым ключом, а обратное -- открытым, то криптосистема называется \textbf{схемой электронной подписи (ЭП)}. Владелец закрытого ключа может \emph{подписать} сообщение, а все пользователи, зная открытый ключ, могут проверить, что подпись была создана только владельцем закрытого ключа и никем другим.

Криптосистемы с открытым ключом снижают требования к каналам связи, которые требуются для передачи данных. В симметричных криптосистемах перед началом связи (перед шифрованием сообщения и его передачей) требуется по защищённому каналу связи передать или согласовать секретный ключ шифрования. Злоумышленник не должен иметь возможность ни прослушать данный канал связи, ни подменить передаваемую информацию (ключ). Криптосистемы с открытым ключом требуют только то, что злоумышленник не должен иметь возможности подменить открытый ключ, который получатель передаёт отправителю для будущего шифрования. Говоря другими словами, криптосистема с открытым ключом, в случае использования открытых и незащищённых каналов связи устойчива к пассивному криптоаналитику\index{криптоаналитик!пассивный}, но всё ещё должна предпринимать меры по защите от активного криптоаналитика\index{криптоаналитик!активный}.

Для предотвращения атак <<человек посередине>> (man-in-the-middle attack)\index{атака!человек посередине} с активным криптоаналитиком\index{криптоаналитик!активный}, который бы подменял открытый ключ получателя во время его передачи будущему отправителю сообщений, используют \textbf{сертификаты открытых ключей}\index{сертификат открытого ключа}. Сертификат представляет собой информацию о соответствии открытого ключа и его владельца, подписанную электронной подписью третьего лица. В корпоративных информационных системах достаточно, если на всю организацию такое лицо, подписывающее сертификаты, будет одно. В этом случае его называют \textbf{доверенным центром сертификации} или \textbf{удостоверяющим центром}. В глобальной сети Интернет для защиты распространения программного обеспечения (например, защиты от подделок в ПО) и проверок сертификатов в протоколах на базе SSL/TLS\index{протокол!SSL/TLS} используется иерархия удостоверяющих центров, рассмотренная в разделе~\ref{section-CAs}. При обмене личными сообщениями и при распространении программного обеспечения с открытым кодом вместо жёсткой иерархии может использоваться \textbf{сеть доверия}\index{сеть доверия}. В сети доверия каждый участник может подписать сертификат любого другого участника. Предполагается, что подписывающий знает лично владельца сертификата и удостоверился о соответствии сертификата владельцу при личной встрече.

Криптосистемы с открытым ключом построены на основе односторонних (однонаправленных) функций c потайным входом. Под \textbf{односторонней} функцией понимают \emph{вычислительную} невозможность вычисления ее обращения: вычисление значения функции $y = f(x)$ при заданном аргументе $x$ является легкой задачей, вычисление аргумента $x$ при заданном значении функции $y$ -- трудной задачей.

Односторонняя функция $y = f(x,K)$ с \textbf{потайным входом}\index{функция!с потайным входом} $K$ определяется как функция, которая легко вычисляется при заданном $x$, и аргумент $x$ которой можно легко вычислить из $y$, если известен <<секретный>> параметр $K$, и вычислить невозможно, если параметр $K$ неизвестен.

Примером подобной функции является возведение в степень по модулю составного числа $n$:
	\[ c = f \left( m \right) = m ^ e \mod n.\]

Для того, чтобы быстро вычислить обратную функцию
	\[ m = f^{-1} \left( c \right) = \sqrt[e]{c} \mod n, \]
её можно представить в виде
	\[ m = y^{d} \mod m,\]
где
	\[ d = e^{-1} \mod \varphi \left( n \right). \]

В последнем выражении $\varphi \left( n \right)$ -- это функция Эйлера\index{функция!Эйлера}. В качестве <<потайной дверцы>> или секрета можно рассматривать или непосредственно само число <<$d$>>, или значение $\varphi \left( b \right)$. Последнее можно быстро найти только в том случае, если известно разложение числа $n$ на простые сомножители. Именно эта функция с потайной дверцей лежит в основе криптосистемы RSA\index{криптосистема!RSA}.

Необходимые математические основы модульной арифметики, групп, полей и простых чисел приведены в Приложении \ref{chap:discrete-math}.

\section{Криптосистемы RSA}
\selectlanguage{russian}
\index{криптосистема!RSA}

\subsection[Шифрование]{Шифрование RSA}

В 1978 г. Рональд Рив\'{е}ст, Ади Шамир и Леонард Адлеман  (R. Rivest, A. Shamir, L. Adleman) предложили алгоритм, обладающий рядом интересных для криптографии свойств. На его основе была построена первая система шифрования с открытым ключом, получившая название по первым буквам фамилий авторов -- система RSA.

Рассмотрим принцип построения криптосистемы шифрования RSA с открытым ключом.

\begin{enumerate}
    \item \textbf{Создание пары из закрытого и открытого ключей.}
        \begin{enumerate}
            \item Случайно выбрать большие простые различные числа $p$ и $q$, для которых $\log_2 p \simeq \log_2 q > 512$ бит.
            \item Вычислить произведение $n = pq$.
            \item Вычислить функцию Эйлера $\varphi(n) = (p-1)(q-1)$.
            \item Выбрать случайное целое число $e \in [2, \varphi(n)-1]$ взаимно простое с $\varphi(n)$: $~ \gcd(e, \varphi(n)) = 1$. Свойство проверяют с помощью алгоритма Евклида.
            \item Вычислить число $d$ такое, что  $d e= 1 \mod \varphi(n)$. Для вычисления используется расширенный алгоритм Евклида.
            \item Закрытый ключ -- $\SK$, открытый ключ -- $\PK$
                \[ \SK = (d), ~ \PK = (n, e). \]

        \end{enumerate}

Генерация модуля $n = pq$ RSA системы является трудной задачей. Действительно, количество нечетных целых длиной точно 500 бит равно $2^{(500-2)}$. Среди них имеется примерно
$(2^{500})/500 - (2^{499})/499 \approx (2^{500})/1000$ простых 500-разрядных чисел. Вероятность случайного выбора простого числа составляет примерно $1/250 $.
Поиск случайных больших простых чисел $p,q$ состоит в генерации случайного нечетного целого числа и проверке его по критериям простоты. Самый распространенный критерий -- вероятностный тест Миллера~---~Рабина\index{тест!Миллера~---~Рабина}. Все вероятностные тесты либо \emph{точно} определяют, что данное число составное, либо что оно \emph{возможно} простое. При $t$-кратной проверке тестом Миллера--Рабина со всеми положительными ответами <<возможно простое>> существует вероятность ошибки $P < \left( \frac{1}{4} \right)^t$, т.~е. ненулевая вероятность того, что число окажется на самом деле составным. Существуют и многие другие детерминированные и вероятностные тесты на простоту числа.

Криптостойкость RSA системы определяется сложностью разложения на сомножители целого $n$-разрядного числа и отсутствием <<лишних>> делителей.

    \item \textbf{Шифрование на открытом ключе $\PK$.}
        \begin{enumerate}
            \item Сообщение представляют целым числом $m \in [1, n-1]$.
            \item Шифротекст вычисляется как
                \[ c = m^e \mod n. \]
                Шифротекст -- тоже целое число из диапазона $[1, n-1]$.
        \end{enumerate}
    \item \textbf{Расшифрование на закрытом ключе $\SK$.}
        \begin{enumerate}
            \item Владелец закрытого ключа вычисляет
                \[ m = c^d \mod n. \]
            \item Покажем верность расшифрования. Пусть
                \[ ed = 1 + a \varphi(n). \]
                Если $m$ и $n$ взаимно простые, то по теореме Эйлера (по модулю $n$):
                \[ c^d = m ^{ed} = m^1 m^{a\varphi(n)} = m \cdot 1^a = m \mod n. \]

                В общем случае $m$ и $n$ могут иметь общие делители, но расшифрование тоже оказывается верным. Пусть $m = 0 \mod p$. По китайской теореме об остатках:
                \[
                     m = c^d \mod n ~\Leftrightarrow~
                     \left\{ \begin{array}{l}
                        m = c^d \mod p, \\
                        m = c^d \mod q. \\
                     \end{array} \right..
                \]
                Подставляя $c=m^e$, получаем тождество
                \[ \left\{ \begin{array}{l}
                    m^{ed} = 0 = m \mod p, \\
                    m^{ed} = m  \left( m^{q-1} \right)^{a(p-1)} = m \cdot 1^{a(p-1)} = m \mod q. \\
                \end{array} \right. \]
                Следовательно, $m^{ed} = m \mod pq$.
        \end{enumerate}
\end{enumerate}


Что касается вычислительной сложности других операций, то применение алгоритма Евклида для проверки, являются ли число $e$  взаимно простым с числами $p-1, q-1$, а также вычисление обратного элемента $d$, считается легкой задачей (задачей с квадратичной сложностью, не более).
Возведение числа в заданную степень $d$ выполняется с помощью последовательного \emph{возведения в квадрат и перемножения}. Пусть
    \[ d = d_0 + d_1 2^1 + d_2 2^2 + \ldots + d_{k-1} 2^{k-1} \]
двоичное представление с коэффициентами $d_{i} \in \{ 0, 1 \}$. Степень $c^d$ вычисляется рекуррентным образом:
  \[ c^d =((... (((c^ {d_{k-1}})^2  (c^{d_{k-2}}))^2)\dots(c^{d_2}))^2 (c^{d_1}))^2 (c^{d_0}).\]

%    \[ c^d = c^ {d_0} \cdot (c^2)^{d_1} \cdot (c^{2^2})^{d_2} \dots  (c^{2^{k-1}})^{d_{k-1}}, \]
Всего выполняется  $k-1$ операций возведения в квадрат и не более $k-1$ умножений, что считается легкой задачей.


\subsubsection{Пример схемы}

%\example
%Схема шифрования RSA.
\begin{enumerate}
    \item Генерирование параметров.
        \begin{enumerate}
            \item Выберем числа $p=13, q=11, n = 143$.
            \item Вычислим $\varphi(n) = (p-1)(q-1) = 12 \cdot 10 = 120$.
            \item Выберем $e=23: ~ \gcd(e, \varphi(n))=1, ~ e \in [2, 119]$.
            \item Найдем $d = e^{-1} \mod \varphi(n) = 23^{-1} \mod 120 = 47$.
            \item Открытый и закрытые ключи:
                \[ \PK = (e:23, n:143), ~ \SK = (d:47). \]
        \end{enumerate}
    \item Шифрование.
        \begin{enumerate}
            \item Пусть сообщение $m = 22 \in [1, n-1]$.
            \item Вычислим шифротекст
                \[ c = m^e = 22^{23} = 55 \mod 143. \]
        \end{enumerate}
    \item Расшифрование.
        \begin{enumerate}
            \item Полученный шифротекст $c = 55$.
            \item Вычислим открытый текст
                \[ m = c^d = 55^{47} = 22 \mod 143. \]
        \end{enumerate}
\end{enumerate}

%Рассмотрим ее основные положения на примере криптосистемы с открытым ключом.
%Приведем общую схему алгоритма RSA.
%$C_i=M_{i}^{E_k}(mod N_j)$
%$N_j=P_{j}Q_{j}$
%$M_i=C_{i}^{D_k}(mod N_j)$
%$E_k\neq D_k$
%Вычислить $E_k$ из $D_k$  при длине блока сообщения  $L_{блока} > L_{дополнения}$ можно только с экспоненциальной сложностью. $E_k D_K=1(mod \varphi(N_j))$
%Данное сравнение не дает единственного решения. Решение данного сравнения и можно свести к следующему уравнению:
%$ax+by=1$
%$E_k D_k=k \varphi(N_j)+1$
%$1\leq E_k D_k <\varphi(N_j)$
%$\varphi(N_j)(-k)+ E_k D_k=1$
%Стандарт ISO X.509 определяет требования по реализации алгоритма RSA, в частности, требования к общесистемным параметрам и ключам, методы распространения сертификатов ключей и ключевых параметров, а также порядок ввода их в действие и многое другое.


\subsection[Электронная подпись]{Электронная подпись RSA}

Предположим, что пользователь $A$ сообщения не шифрует, но хочет посылать свои сообщения в виде открытых текстов с подписью. Для этого надо создать электронную подпись (ЭП). Это можно сделать, используя систему RSA. При этом должны быть выполнены следующие требования:
\begin{itemize}
    \item вычисление подписи от сообщения является вычислительно легкой задачей;
    \item фальсификация подписи при неизвестном закрытом ключе -- вычислительно трудная задача;
    \item подпись должна быть проверяемой открытым ключом.
\end{itemize}

Создание параметров ЭП RSA производится так же, как и для схемы шифрования RSA. Пусть  $A$ имеет закрытый ключ $\SK = (d)$, а получатель (проверяющий) $B$ -- открытый ключ $\PK = (e,n)$ пользователя $A$.

\begin{enumerate}
    \item $A$ вычисляет подпись сообщения $m \in [1,n-1]$ как
        \[ s = m^{d} \mod n \]
        на своем закрытом ключе $\SK$.
    \item $A$ посылает $B$ сообщение в виде $(m, s)$, где $m$ -- открытый текст, $s$ -- подпись.
    \item $B$ принимает сообщение $(m, s)$, возводит $s$ в степень $e$ по модулю $n$ ($e, n$ -- часть открытого ключа). В результате вычислений $B$ получает открытый текст
        \[ \left( m^{d} \mod n \right)^{e} \mod n = m. \]
    \item Сравнивает полученное значение с первой частью сообщения. При полном совпадении подпись принимается.
\end{enumerate}
Недостаток этой системы создания ЭП состоит в том, что подпись $m^{d} \mod n$ имеет большую длину, равную длине открытого сообщения $m$.

Для уменьшения длины подписи применяется другой вариант процедуры: вместо сообщения $m$ отправитель подписывает $h(m)$, где $h(x)$ -- известная криптографическая хэш-функция. Модифицированная процедура состоит в следующем.

\begin{enumerate}
    \item $A$ посылает $B$ сообщение в виде $(m, s)$, где $m$ -- открытый текст,
        \[ s = h(m)^d \mod n \]
        подпись.
    \item $B$ принимает сообщение $(m, s)$, вычисляет хэш $h(m)$ и возводит подпись в степень
        \[ h_1 = s^e \mod n. \]
    \item $B$ сравнивает значения $h(m)$ и $h_1$. При равенстве
        \[ h(m) = h_1 \]
        подпись считается подлинной, при неравенстве -- фальсифицированной.
\end{enumerate}


\subsubsection{Пример схемы}

\begin{enumerate}
    \item Генерирование параметров.
        \begin{enumerate}
            \item Выберем $p=13, q=17, n = 221$.
            \item Вычислим $\varphi(n) = (p-1)(q-1) = 12 \cdot 16 = 192$.
            \item Выберем $e=25: ~ \gcd(e = 25, \varphi(n) = 192) = 1, \\
                e \in [2, \varphi(n) - 1 = 191]$.
            \item Найдем $d = e^{-1} \mod \varphi(n) = 25^{-1} \mod 192 = 169$.
            \item Открытый и закрытые ключи:
                \[ \PK = (e:25, n:221), ~ \SK = (d:169). \]
        \end{enumerate}
    \item Подписание.
        \begin{enumerate}
            \item Пусть хэш сообщения $h(m) = 12 \in [1, n-1]$.
            \item Вычислим ЭП
                \[ s = h^d = 12^{169} = 90 \mod 221. \]
        \end{enumerate}
    \item Проверка подписи.
        \begin{enumerate}
            \item Пусть хэш полученного сообщения $h(m) = 12$, полученная подпись $s = 90$.
            \item Выполним проверку
                \[ h_1 = s^e = 90^{25} = 12 \mod 221, ~~ h_1 = h, \]
                подпись верна.
        \end{enumerate}
\end{enumerate}


\subsection[Рандомизация шифрования и ЭП]{Рандомизация шифрования и \protect\\ подписания RSA}

\textbf{Семантически безопасной}\index{криптосистема!семантически-безопасная} называется криптосистема, для которой вычислительно невозможно извлечь любую информацию из шифротекстов, кроме длины шифротекста. Алгоритм RSA не является семантически безопасным. Одинаковые сообщения шифруются одинаково и, следовательно, применима атака на различение сообщений.

Кроме того, сообщения длиной менее $\frac{k}{3}$ бит, зашифрованные на малой экспоненте $e=3$, \emph{дешифруются} нелегальным пользователем извлечением обычного кубического корня.

В приложениях RSA используется только в сочетании с рандомизацией\index{рандомизация шифрования}. В стандарте PKCS\#1 RSA Laboratories описана схема рандомизации перед шифрованием OAEP-RSA (Optimal Asymmetric Encryption Padding). Примерная схема:
\begin{enumerate}
    \item Выбирается случайное $r$.
    \item Для открытого текста $m$ вычисляется
        \[ x = m \oplus H_1(r), ~ y = r \oplus H_2(x), \]
        где $H_1$ и $H_2$ -- криптографические хэш-функции.
    \item Сообщение $M = x \| y$ далее шифруется RSA.
\end{enumerate}
Восстановление $m$ из $M$ при расшифровании:
    \[ r = y \oplus H_2(x), ~ m = x \oplus H_1(r). \]

В модификации OAEP+ $x$ вычисляется как
    \[ x = (m \oplus H_1(r)) \| H_3(m \| r). \]

В описанной выше схеме ЭП под $m$ понимается хэш открытого текста, вместо шифрования выполняется подписание, вместо расшифрования -- проверка подписи.


\subsection{Выбор параметров и оптимизация}

\subsubsection{Выбор экспоненты $e$}

В случайно выбранной экспоненте $e$ c битовой длиной $k = \lceil \log_2 e \rceil$ половина бит в среднем равна 0, половина -- 1. При возведении в степень $m^e \mod n$ по методу <<возводи в квадрат и перемножай>> получится $k-1$ возведений в квадрат и, в среднем,
 $\frac{1}{2}(k-1)$ умножений.

Если выбрать $e$, содержащим малое число единиц в двоичной записи, то число умножений уменьшится до числа единиц в $e$.

Часто экспонента $e$ выбирается  \emph{малым} \emph{простым} числом и/или содержащим малое число единиц в битовой записи для ускорения шифрования или проверки подписи, например:
\[
    \begin{array}{l}
        3 = [11]_2, \\
        17 = 2^4+1 = [10001]_2, \\
        257 = 2^8+1 = [100000001]_2, \\
        65537 = 2^{16}+1 = [10000000000000001]_2.
    \end{array}
\]

%Время шифрования или проверки подписи для малых экспонент становится $O(k^2)$ вместо $O(k^3)$, то есть в сотни раз быстрее для 1000-битовых чисел.


\subsubsection[Ускорение шифрования]{Ускорение шифрования по китайской \protect\\ теореме об остатках}

Возводя $m$ в степень $e$ отдельно по $\mod p$ и $\mod q$ и применяя китайскую теорему об остатках (Chinese Remainder Theorem, CRT), можно быстрее выполнить шифрование.

Однако ускорение шифрования в криптосистеме RSA через CRT может привести к уязвимостям в некоторых применениях, например в смарт-картах.

\example
Пусть $c = m^e \mod n$ передается на расшифрование на смарт-карту, где вычисляется
\[ \begin{array}{c}
    m_p = c^d \mod p, \\
    m_q = c^d \mod q, \\
    m = m_p q (q^{-1} \mod p) + m_q p (p^{-1} \mod q) \mod n. \\
\end{array} \]
Криптоаналитик внешним воздействием может вызвать сбой во время вычисления $m_p$ (или $m_q$), в результате получится $m_p'$ и $m'$ вместо $m$. Зная $m_p'$ и $m'$, криптоаналитик находит разложение числа $n$ на множители $p,q$:
    \[ \gcd(m' - m, ~ n) = \gcd( (m_p' - m) q (q^{-1} \mod p), ~ pq) = q. \]
\exampleend


\subsubsection{Длина ключей}

В 2005 году было разложено 663-битовое число вида RSA. Время разложения в эквиваленте составило 75 лет вычислений одного ПК. Самые быстрые алгоритмы факторизации -- субэкспоненциальные\index{задача!факторизации}. Минимальная рекомендуемая длина модуля $n$ -- 1024 бит, но лучше использовать 2048 или 4096 бит.

В приложении показано, что битовая сложность (количество битовых операций) вычисления произвольной степени $a^b \mod n$ является кубической $O(k^3)$, а возведения в квадрат $a^2 \mod n$ и умножения $a b \mod n$ -- квадратичными $O(k^2)$, где $k$ -- битовая длина чисел $a,b,n$.

%Увеличение длины модуля $n$ в 2 раза увеличивает время возведения в степень в $2^3$ раз для большой экспоненты $e$, а для маленькой экспоненты -- в $2^2$ раза.


\input{el-gamal}

\input{GOST_R_34.10-2001.tex}

\section{Длины ключей}
\selectlanguage{russian}

В табл. \ref{tab:recommended-key-lengths} приведены битовые длины ключей для криптосистем.
%Традиционные рекомендации основаны на аппроксимации существующих алгоритмов для взлома на 10-30 лет вперед.

\begin{table}[!ht]
    \centering
    \caption{Минимальные длины ключей в битах по стандартам России и США\label{tab:recommended-key-lengths}}
    \resizebox{\textwidth}{!}{ \begin{tabular}{|l|c|c|c|c|}
        \hline
        & \multirow{2}{*}{\parbox{1.5cm}{Блоковые шифры, $K$}} & \multicolumn{3}{|c|}{Схема ЭП} \\
        \cline{3-5}
        & & \parbox{1.5cm}{RSA\index{криптосистема!RSA}, $n$} & \parbox{2.3cm}{Эллипт. кривые, порядок точки} & \parbox{3.5cm}{Эль-Гамаль\index{криптосистема!Эль-Гамаля} $\mod p$: модуль / порядок (под)группы} \\
        \hline \hline
        \multicolumn{5}{|c|}{Взломано} \\
        \hline
        Биты & 56 & 663 & 109 & 503  \\
        Конкурс & \textsc{DesChal} & RSA-200 & ECC2K-108 &  \\
        Год & 1997 & 2005 & 2000 &  \\
        \hline \hline
        \multicolumn{5}{|c|}{Стандарт России} \\
        \hline
        Биты & 256 &  & 255 & \\
        ГОСТ & 28147—89 & --- & 34.10-2001 & --- \\
        Год & 1989 & & 2001 & \\
%       \hline
%       \multicolumn{2}{|l|}{\parbox{4cm}{Россия: нелицензируемая деятельность}} & \multicolumn{4}{c|}{40} \\
        \hline \hline
        \multicolumn{5}{|c|}{Стандарт США} \\
        \hline
        Биты & 128-256 & 1024-3072 & 151-480 & 1024-3072/160-256 \\
        FIPS № & 197 & draft 186-3 & draft 186-3 & draft 186-3 \\
        Год & 2001 & 2006 & 2006 & 2006 \\
%       \hline
%       \multicolumn{2}{|l|}{\parbox{4cm}{США: экспортные ограничения до 2001 г.}} & 56 & 512 & 112 & 512/112 \\
%       \hline \hline
%       \multicolumn{2}{|l|}{Традиционные} & 80 & 1024 & 160 & 1024/160 \\
%       \cline{3-6}
%       \multicolumn{2}{|l|}{рекомендации} & 112 & 2048 & 224 & 2048/224 \\
%       \hline
%       \multicolumn{2}{|l|}{\parbox{4cm}{Рекомендация Lenstra, Verheul для 2010 г.}} & 78 & 1369 & 146-160 & 1369/138 \\
        \hline
    \end{tabular} }
\end{table}
%}\end{center}


\subsection*{Скорость вычисления ЭП}

Сравним производительность схем ЭП, чтобы продемонстрировать преимущества ЭП вида Эль-Гамаля\index{криптосистема!Эль-Гамаля} перед RSA\index{криптосистема!RSA} для больших ключей. В приложении показано, что в модульной арифметике по модулю числа $n$ с битовой длиной $k \simeq \log_2 n$ операции имеют битовую сложность:
\[ \begin{array}{lcl}
    a^b \mod n & - & O(k^3), \\
    ab \mod n, ~ a^{-1} \mod n & - & O(k^2), \\
    a+b \mod n & - & O(k). \\
\end{array} \]

Так как все описанные схемы ЭП используют возведение в степень по модулю, то битовая сложность -- $O(k^3)$. Оценки количества целочисленных $t$-разрядных умножений при вычислении ЭП имеют вид:
\begin{enumerate}
    \item RSA\index{криптосистема!RSA}:
        \[ (2 \log_2 n) \cdot \left( \frac{\log_2 n}{t} \right)^2; \]
    \item DSA\index{DSA} (Digital Signature Algorithm, стандарт США~\cite{FIPS-PUB-186-4}), вычисляемая по принципу Эль-Гамаля\index{криптосистема!Эль-Гамаля} по модулю $p$ и с порядком циклической подгруппы $q$:
        \[ (2 \log_2 q) \cdot \left( \frac{\log_2 p}{t} \right)^2; \]
    \item ГОСТ Р 34.10-2001\index{ГОСТ Р 34.10-2001} (стандарт России~\cite{GOST-2001}) и ECDSA\index{ECDSA} (Elliptic Curve Digital Signature Algorithm, стандарт США~\cite{FIPS-PUB-186-4}), вычисляемые по принципу Эль-Гамаля\index{криптосистема!Эль-Гамаля} в группе точек эллиптической кривой по модулю $p$:
        \[ (2 \log_2 p) \cdot 4 \cdot \left( \frac{\log_2 p}{t} \right)^2; \]
\end{enumerate}

В табл. \ref{tab:signature-rate} приведены оценки скорости вычисления ЭП (оценки числа умножений 64-битных слов).

\begin{table}[!ht]
    \centering
    \caption{Оценочное число 64-битных умножений для вычисления ЭП\label{tab:signature-rate}}
    \begin{tabular}{|c|l|c|}
        \hline
        ЭП & Оценочное число 64-битных умножений \\
        \hline \hline
        RSA\index{криптосистема!RSA} 1024 & $(2 \cdot 1024) \cdot \left( \frac{1024}{64} \right)^2 \approx$ 500 000 \\
        RSA\index{криптосистема!RSA} 2048 & 4 000 000 \\
        RSA\index{криптосистема!RSA} 3072 & 14 000 000 \\
        RSA\index{криптосистема!RSA} 4096 & 34 000 000 \\
        \hline \hline
        DSA 1024/160 & $(2 \cdot 160) \cdot \left( \frac{1024}{64} \right)^2 \approx$ 82 000 \\
        DSA 3072/256 & 1 200 000 \\
        \hline \hline
        ECDSA 160 & $(2 \cdot 160) \cdot 4 \cdot \left( \frac{160}{64} \right)^2 \approx$ 8 000 \\
        ECDSA 512 & 260 000 \\
        \hline \hline
        ГОСТ Р 34.10-2001 & $(2 \cdot 256) \cdot 4 \cdot \left( \frac{256}{64} \right)^2 \approx$ 33 000 \\
        \hline
    \end{tabular}
\end{table}



\chapter{Распространение ключей}

Задачей распространения ключей между двумя пользователями является создание секретных псевдослучайных сеансовых ключей шифрования и аутентификация сообщений. Пользователи предварительно создают и обмениваются ключами аутентификации один раз. В дальнейшем для создания защищенной связи пользователи производят взаимную аутентификацию и вырабатывают сеансовые ключи\index{ключ!сеансовый}.

\section[Трехэтапный протокол Шамира]{Трехэтапный протокол Шамира на коммутативных шифрах}
\selectlanguage{russian}

Предположим, что две стороны $A$ и $B$ соединены незащищённым каналом связи. Каждая из этих сторон имеет свой секретный ключ: $A$ имеет ключ $K_A$, $B$ имеет ключ $K_B$. Сторона $A$ должна создать общий секретный ключ $K$ и передать стороне $B$.

Для решения этой задачи используют трехэтапный протокол Шамира с тремя <<замками>>. \textbf{Протокол Шамира}\index{протокол!Шамира} построен на \emph{коммутативных} функциях шифрования, для которых выполняется условие:
    \[ E_{K_{B}} (E_{K_{A}}(K))=E_{K_{A}} (E_{K_{B}}(K)). \]

Протокол предполагает следующие процедуры.
\begin{enumerate}
    \item $A$ создает секретный ключ $K$, шифрует его своей системой шифрования с помощью своего ключа $K_A$ и посылает сообщение стороне $B$:
        \[ A \rightarrow B: ~ E_{K_A}(K). \]
    \item $B$ получает это сообщение, шифрует его с помощью своего ключа $K_B$ и посылает сообщение стороне $A$:
        \[ A \leftarrow B: ~ E_{K_B}( E_{K_A}( K)). \]
    \item Сторона $A$, получив сообщение $E_{K_B}(E_{K_A}(K))$, использует свой секретный ключ $K_A$ для расшифрования:
            \[ D_{K_A}(E_{K_B} (E_{K_A}(K))) = E_{K_B}(K). \]
        Сторона $A$ передает стороне $B$ сообщение:
        \[ A \rightarrow B: ~ E_{K_B}(K). \]
    \item Сторона $B$, получив сообщение $E_{K_B}(K)$, использует свой секретный ключ $K_B$ для расшифрования:
            \[ D_{K_B}(E_{K_B}(K)) = K. \]
        В результате стороны получают общий секретный ключ $K$.
\end{enumerate}

Приведем пример неудачного шифрования с использованием коммутативных функций.

\begin{enumerate}
    \item $A$ имеет функцию шифрования совершенной секретности $E_{K_A}(K) = K \oplus K_A$, где $K_A$ -- двоичная последовательность с равномерным распределением символов. $A$ посылает это сообщение стороне $B$:
            \[ A \rightarrow B: ~ E_{K_A}(K) = K \oplus K_A. \]
    \item $B$ использует такую же функцию шифрования совершенной секретности с ключом $K_B$ (двоичная последовательность с равномерным распределением символов). $B$ шифрует полученное сообщение и отправляет $A$:
            \[ A \leftarrow B: ~ E_{K_A}(K) \oplus K_B = K \oplus K_A \oplus K_B. \]
    \item Сторона $A$, получив сообщение $K \oplus K_A \oplus K_B$, выполняет расшифрование:
            \[ K \oplus K_A \oplus K_B \oplus K_A = K \oplus K_B. \]
        Сторона $A$ передает стороне $B$ сообщение:
            \[ A \rightarrow B: ~ K \oplus K_B. \]
    \item Сторона $B$, получив сообщение $K \oplus K_B$, выполняет расшифрование:
            \[ K \oplus K_B \oplus K_B = K. \]
        Обе стороны получают общий секретный ключ $K$.
\end{enumerate}

Предложенный выбор коммутативной функции шифрования совершенной секретности был назван неудачным, так как существуют ситуации, при которых криптоаналитик может определить ключ $K$. Предположим, что криптоаналитик перехватил все три сообщения:
    \[ K \oplus K_A, ~~ K \oplus K_A \oplus K_B, ~~ K \oplus K_B. \]
Сложение по модулю 2 всех трех сообщений дает ключ $K$. Поэтому такая система шифрования не применяется.

Теперь приведем протокол надежной передачи секретного ключа, основанный на экспоненциальной (коммутативной) функции шифрования. Стойкость этого протокола связана с трудной задачей -- задачей вычисления дискретного логарифма: известны значения $y, g, p$, найти $x$ в уравнении $y = g^x \mod p$.

\textbf{Протокол Шамира распространения ключей}.
Выбирают большое простое\index{число!простое} число $p\sim 2^{1024}$ и используют его как открытый ключ.

\begin{enumerate}
    \item Сторона $A$ задает общий секретный ключ $K <p$ и выбирает целое число $a$, взаимно простое с $p-1$. $A$ вычисляет и посылает сообщение стороне $B$:
            \[ A \rightarrow B: ~ K^a \mod p. \]
        Существует число $c$ такое, что $a c =1 \mod (p-1)$, то есть $a c = 1 + l (p-1)$, где $l$ -- целое число. Число $c$ будет использовано стороной $A$ на следующем этапе.
    \item Сторона $B$ выбирает целое число $b$, взаимно простое с $p-1$, используя полученное сообщение, вычисляет и посылает сообщение стороне $A$:
            \[ A \leftarrow B: ~ (K^a)^b \mod p. \]
    \item Сторона $A$, получив сообщение, вычисляет
        \[ \left( K^{ab} \right)^c = K^{1 + l (p-1) b} = K^b \cdot K^{l (p-1) b} = K^b \mod p. \]
        Здесь применена малая теорема Ферма\index{теорема!Ферма малая}: $K^{p-1} = 1 \mod p$, поэтому $\left( K^{p-1} \right)^{lb} = 1 \mod p$.
        $A$ посылает $B$ сообщение:
            \[ A \rightarrow B: ~ K^b \mod p. \]
    \item Сторона $B$, получив сообщение $K^{b}\mod p$, вычисляет
        \[ (K^b \mod p)^d = K^{bd} \mod p = K. \]
\end{enumerate}

Теперь проверим криптостойкость этого протокола. Предположим, что криптоаналитик перехватил три сообщения:
\[ \begin{array}{l}
    y_1 = K^a \mod p, \\
    y_2 = K^{ab} \mod p, \\
    y_3 = K^b \mod p. \\
\end{array} \]
Чтобы найти ключ $K$, надо решить систему из этих трех уравнений, что имеет очень большую вычислительную сложность, неприемлемую с практической точки зрения, если все три числа $a, b, ab$ достаточно велики. Предположим, что $a$ (или $b$) мало. Тогда, вычисляя последовательные степени $y_3$ (или $y_1$), можно найти  $a$ (или $b$), сравнивая результат с $y_2$. Зная $a$, легко найти $a^{-1}\mod(p-1)$ и $K=(y_1)^{a^{-1}}\mod p$.

Недостатком этого протокола является отсутствие аутентификации сторон. Следовательно, нужно дополнительно использовать цифровую подпись при передаче сообщения.


\section{Протоколы с симметричными шифрами}

\subsection{Аутентификация и атаки воспроизведения}

Рассмотрим такую ситуацию: обе стороны $A$ и $B$ имеют общий долговременный ключ $K_{AB}$ и симметричную систему шифрования. Нужно выработать сеансовый секретный ключ $K$. Сторона $A$ создает ключ $K$ и желает его передать стороне $B$.

\begin{enumerate}
    \item Для этого сторона $A$ с помощью общего ключа $K_{AB}$ создает и передает $B$ шифрованное сообщение:
            \[ A \rightarrow B: ~ E_{K_{AB}}(K, B, A). \]
        В этом сообщении имеются так называемые поля -- $(B,A)$ -- информация для дополнительного подтверждения.
    \item Сторона $B$, используя общий ключ $K_{AB}$, расшифровывает полученное сообщение:
            \[ D_{K_{AB}}( E_{K_{AB}}( K, B, A)) = (K, B, A). \]
        В результате сторона $B$ получает сеансовый ключ $K$ и дополнительные данные $(B,A)$.
\end{enumerate}

Недостаток этого протокола состоит в том, что криптоаналитик может перехватывать сообщения и через некоторое время переслать их стороне $A$.

Рассмотрим другие варианты решения задачи о передаче сеансового ключа.
Задача остается прежней: обе стороны $A$ и $B$ имеют общий долговременный секретный ключ $K_{AB}$, сторона $A$ должна выработать сеансовый секретный ключ $K$ и доставить стороне $B$.

Протокол включает \textbf{метки времени} -- информацию о моменте $t_A$ отправки сообщения и моменте получения сообщения $t_B$.

\begin{enumerate}
    \item Сторона $A$ вырабатывает $K$ и с помощью долговременного ключа $K_{AB}$ создает шифрованное сообщение с меткой времени $t_A$ и передает его стороне $B$:
            \[ A \rightarrow B: ~ E_{K_{AB}}(K, t_A). \]
    \item Сторона $B$ получает сообщение и расшифровывает его с помощью общего ключа:
            \[ D_{K_{AB}}( E_{K_{AB}}( K, t_A)) = (K, t_A). \]
        В результате $B$ получает $(K, t_A)$, то есть секретный ключ и метку времени. $B$ измеряет время прихода $t_B$ и интервал запаздывания. Если $|t_B - t_A| \le \delta$, то $B$ аутентифицирует $A$.
\end{enumerate}
Метка времени является одноразовой меткой и защищает от атак воспроизведения ранее записанных сообщений.

Рассмотрим другой способ передачи ключа с дополнительной информацией в виде \textbf{одноразовых случайных меток} (nonce -- number used once) вместо меток времени. Протокол передачи состоит в следующем.

\begin{enumerate}
    \item Сторона $A$ вырабатывает случайное число $r_A$, шифрует сообщение, в котором  $(r_A, A)$ -- реквизиты $A$, и передает его стороне $B$:
            \[ A \rightarrow B: ~ E_{K_{AB}}(r_A, A). \]
    \item Сторона $B$ вырабатывает сеансовый ключ $K$, создает шифрованное сообщение и посылает его $A$:
            \[ A \leftarrow B: ~ E_{K_{AB}}(K, r_A, A). \]
    \item Сторона $A$ расшифровывает полученное сообщение:
            \[ D_{K_{AB}}( E_{K_{AB}}( K, r_A, A)) = (K, r_A, A). \]
        В результате $A$ получает сеансовый ключ и подтверждение своих реквизитов, что является дополнительной аутентификацией.
\end{enumerate}

Предположим, что сторона $B$ тоже желает убедиться, что имеет дело со стороной $A$. Тогда этот протокол следует дополнить передачей реквизитов $B$. По-прежнему считаем, что у $A$ и $B$ -- общая система шифрования с долговременным секретным ключом $K_{AB}$.

\begin{enumerate}
    \item Сторона $A$ вырабатывает случайное число $r_A$, шифрует и передает стороне $B$ сообщение, в котором  $(r_A, A)$ -- реквизиты $A$:
            \[ A \rightarrow B: ~ E_{K_{AB}}(r_A, A). \]
    \item Сторона $B$ вырабатывает случайное число $r_B$ и отправляет стороне $A$ зашифрованное сообщение:
            \[ A \leftarrow B: ~ E_{K_{AB}}(K_B, r_B, r_A, A), \]
        где $K_B$ -- ключ $B$.
     \item Сторона $A$ осуществляет расшифрование
            \[ D_{K_{AB}}(E_{K_{AB}}(K_B, r_B, r_A, A)) = (K_B, r_B, r_A, A) \]
        и получает ключ $K_B$ и реквизиты $r_B, r_A, A$. Для аутентификации себя сторона $A$ создает свой ключ $K_A$ и отправляет стороне $B$ шифрованное сообщение
            \[ A \rightarrow B: ~ E_{K_{AB}}(K_A, r_B, r_A, B). \]
     \item Сторона $B$ осуществляет расшифрование
            \[ D_{K_{AB}}(E_{K_{AB}}(K_A, r_B, r_A, B)) = (K_A, r_B, r_A, B), \]
        которое определяет ключ $K_A$ и аутентифицирует $A$.
\end{enumerate}

Таким образом, обе стороны имеют в своем распоряжении ключи $K_A, K_B$ в качестве сеансовых секретных ключей.


\subsection{Протокол с ключевым кодом аутентификации}

При использовании хэш-функции $K = h(K_{A} \| K_{B})$ происходит усиление секретности. Здесь $(K_{A} \| K_{B})$ -- конкатенация $K_{A} $ и $K_{B}$.

% Достоинства: предположим, $K_{A} ,K_{B} $ -- не обладают «хорошими» свойствами случайности (биты распределены неравномерно или зависимы друг от друга), т.~е., $P_{K_{A} ,K_{B} } (0)=\frac{1}{2} -\varepsilon $, где $\varepsilon $ - мало, но не 0. Тогда вероятность того, что этот бит в \textit{K }будет равным нулю, $P_{K} (0)=\frac{1}{2} -\varepsilon ',\varepsilon '<\varepsilon $- усиление секретности.

Вычисление хэш-значения, как правило, выполняется быстрее, чем расшифрование. Поэтому были разработаны протоколы, в которых вместо функции шифрования используется имитовставка\index{имитовставка} на основе хэш-функции ($\MAC_K$). Рассмотрим протокол такого рода.
\begin{enumerate}
    \item Сторона $A$ вырабатывает сеансовый ключ $K$, использует одноразовую метку $t_{A}$, создает и пересылает стороне $B$ сообщение:
            \[ A \rightarrow B: ~ t_A, ~ B, ~ K \oplus \MAC_{K_{AB}}( t_A, B), ~ \MAC_{K_{AB}}(K, t_A, B). \]
    \item Сторона $B$ вычисляет
            \[ \MAC_{K_{AB}}(t_A, B) \oplus K \oplus \MAC_{K_{AB}}(t_A, B) = K \]
        и получает сеансовый ключ $K$.
\end{enumerate}

Заметим, что криптоаналитик может добавить в поле случайную последовательность, тогда вместо $K$ получаем <<$K$ плюс помеха>>. Вмешательство криптоаналитика будет выявлено благодаря наличию четвертого поля в сообщении. Используя полученное значение $K$, вычисляют $\MAC_{K_{AB}}(K, t_A, B)$ и сравнивают с четвертым полем. Если совпадает, то вмешательства криптоаналитика не было.

\subsection{Протокол Нидхема~---~Шрёдера с доверенным центром}\index{протокол!Нидхема~---~Шрёдера}
\selectlanguage{russian}

Рассмотрим ситуацию, когда в сети имеется некоторый надежный (доверенный) сервер (центр) $T$, которому доверяют все пользователи сети. Сервер для работы с абонентами сети использует некоторую систему шифрования $E_S(*)$, где ключ $S=K_{AT}$  известен только $A$ и $T$, но неизвестен остальным участникам сети, $S = K_{BT}$ известен только $B$ и  $T$. Предполагаем, что сервер имеет хороший генератор случайных чисел. Сеансовый ключ сервер вырабатывает по запросу. Стороны $A$ и $B$ могут выбирать разные одноразовые метки.

Приведем в качестве примера упрощенную версию известного \textbf{протокола Нидхема~---~Шрёдера} (Needham~---~Schroeder protocol) с симметричным шифром.
\begin{enumerate}
    \item Сторона $A$ передает серверу $T$ реквизиты сторон $A$ и $B$  и некую одноразовую метку $N_A$, которая может быть, например, меткой времени или случайным (одноразовым) числом, что оговаривается заранее:
            \[ A \rightarrow T: ~ A, B, N_A. \]
    \item Сервер $T$ вырабатывает секретный сеансовый ключ $K$ для $A$ и $B$ и отправляет стороне $A$ шифрованное сообщение:
            \[ A \leftarrow T: ~ E_{K_{AT}}(N_A, B, K, E_{K_{BT}}(K, A)). \]
    \item Сторона $A$ расшифровывает сообщение
            \[ D_{K_{AT}}( E_{K_{AT}}(N_A, B, K, E_{K_{BT}}(K, A))) = (N_A, B, K, E_{K_{BT}}(K, A)) \]
        и, чтобы доставить ключ, передает стороне $B$ сообщение:
            \[ A \rightarrow B: ~ E_{K_{BT}}(K, A). \]
    \item Сторона $B$ расшифровывает полученное сообщение
            \[ D_{K_{BT}}( E_{K_{BT}}( K,A)) = (K,A) \]
        и получает ключ и реквизиты $A$, которые требуются для того, чтобы сторона $B$ знала, кому отвечать. Кроме того, сторона $B$ дополнительно желает идентифицировать сторону $A$. Для этого $B$ пересылает $A$ зашифрованную одноразовую метку:
            \[ A \leftarrow B: ~ E_{K}(N_B). \]
    \item Сторона $A$ расшифровывает
            \[ D_K( E_K( N_B)) = N_B \]
        и возвращает $B$ изменённую одноразовую метку
            \[ A \rightarrow B: ~ E_K(N_B + 1). \]
    \item Сторона $B$ расшифровывает
            \[ D_K( E_K( N_B + 1)) = N_B + 1, \]
        проверяет $N_B$ и убеждается, что имеет дело со стороной $A$.
    \item Если требуется двусторонняя аутентификация, то аналогично поступают со стороной $A$: на некотором этапе вносится одноразовая метка $N_A$.
\end{enumerate}


\section{Протоколы на криптосистемах с открытым ключом}

\subsection{Простой протокол}

Рассмотрим протокол распространения ключей с помощью асимметричных шифров. Введем обозначения: $K_B$ -- открытый ключ стороны $B$, а $K_A$ -- открытый ключ стороны $A$. Протокол включает три сеанса обмена информацией.
\begin{enumerate}
    \item В первом сеансе сторона $A$ посылает стороне $B$ сообщение
            \[ A \rightarrow B: ~ E_{K_B}(K_1, A), \]
        где $K_1$ -- ключ, выработанный стороной $A$.
    \item Сторона $B$ получает $(K_1, A)$ и передает стороне $A$ наряду с другой информацией свой ключ $K_2$ в сообщении, зашифрованном с помощью открытого ключа $K_A$:
            \[ A \leftarrow B: ~ E_{K_A}(K_2, K_1, B). \]
    \item Сторона $A$ получает и расшифровывает сообщение $(K_2, K_1, B)$. Во время третьего сеанса сторона $A$, чтобы подтвердить, что она знает ключ $K_2$, посылает стороне $B$ сообщение
            \[ A \rightarrow B: ~ E_{K_B}(K_2). \]
\end{enumerate}
Общий ключ формируется из двух ключей $K_1, K_2$.

\subsection{Протоколы с цифровыми подписями}

Существуют протоколы обмена, в которых перед началом обмена ключами генерируются подписи сторон $A$ и $B$, соответственно $S_A(m)$ и $S_B(m)$. В этих протоколах можно использовать различные одноразовые метки. Рассмотрим пример.
\begin{enumerate}
    \item Сторона $A$ выбирает ключ $K$ и вырабатывает сообщение
            \[ \left( K, ~ t_A, ~ S_A(K, t_A, B) \right), \]
        где $t_A$ -- метка времени. Зашифрованное сообщение передает стороне $B$:
        \[ A \rightarrow B: ~ E_{K_B}(K, ~ t_A, ~ S_A(K, t_A, B)). \]
    \item Сторона $B$ получает это сообщение, расшифровывает $\left( K, ~ t_A, ~ S_A(K, t_A, B) \right)$ и вырабатывает свою метку времени $t_B$. Проверка считается успешной, если $|t_B - t_A | < \delta $. Сторона $B$ знает свои реквизиты и может осуществлять проверку подписи.
\end{enumerate}

Имеется второй вариант протокола, в котором шифрование и подпись выполняются раздельно.
\begin{enumerate}
    \item Сторона $A$ вырабатывает ключ $K$, использует одноразовую метку (или метку времени) $t_{A}$ и передает стороне $B$ два различных зашифрованных сообщения
            \[ \begin{array}{ll}
                A \rightarrow B: & ~ E_{K_B}(K, t_A), \\
                A \rightarrow B: & ~ S_A(K, t_A, B). \\
            \end{array} \]
    \item Сторона $B$ получает это сообщение, расшифровывает $K, t_A$ и, добавив свои реквизиты $B$, может проверить подпись $S_A(K, t_A, B)$.
\end{enumerate}

В третьем варианте протокола сначала производится шифрование, потом подпись.
\begin{enumerate}
    \item Сторона $A$ вырабатывает ключ $K$, использует одноразовую случайную метку или метку времени $t_A$ и передает стороне $B$ сообщение
        \[ A \rightarrow B: ~ t_A, ~ E_{K_B}(K, A), ~ S_A(t_A, ~ K, ~ E_{K_B}(K, A)). \]
    \item Сторона $B$ получает это сообщение, расшифровывает $\left( t_A, ~ K, ~ A, ~ E_{K_B}(K, A) \right)$ и проверяет подпись $S_A(t_A, ~ K, ~ E_{K_B}(K, A))$.
\end{enumerate}

\subsection{Протокол Диффи~---~Хеллмана}
\selectlanguage{russian}

Алгоритм с открытым ключом впервые был предложен У.~Диффи (W.~Diffie) и М.~Хеллманом (M.E.~Hellman) в работе 1976 года <<Новые направления в криптографии>>\index{протокол!Диффи~---~Хеллмана}.

Рассмотрим \textbf{протокол Диффи~---~Хеллмана} обмена информацией двух сторон $A$ и $B$. Задача состоит в том, чтобы создать общий сеансовый ключ.

Пусть $p$ -- большое простое число, $g$ -- примитивный элемент группы $\Z_p^*$, ~ $y = g^x \mod p$, причем $p,y,g$ -- известны заранее. Функцию $y=g^{x} \mod p$ считаем однонаправленной, т.е. вычисление функции при известном значении аргумента является легкой задачей, а ее обращение (нахождение аргумента) при известном значении функции -- является трудной задачей.

Протокол обмена состоит из следующих действий.
\begin{enumerate}
    \item Сторона $A$ выбирает случайное число $x, ~ 2 \leq x \leq (p-1)$, вычисляет и передает стороне $B$ сообщение:
        \[ A \rightarrow B: ~ g^x \mod p. \]
    \item Сторона $B$ выбирает случайное число $y, ~ 2\leq y \leq (p-1)$, вычисляет и передает стороне $A$:
        \[ A \leftarrow B: ~ g^y \mod p. \]
    \item Сторона $A$, используя известные ей значения $x,g^{y} \mod p$, вычисляет ключ
        \[ K_{A} =(g^{y})^{x}\mod p=g^{xy} \mod p. \]
    \item Сторона $B$, используя известные ей значения $y,g^{x} \mod p$, вычисляет ключ
        \[ K_{B} =(g^{x})^{y}\mod p=g^{xy}\mod p. \]
        В результате получаем равенство  $K_A = K_B = K$.
\end{enumerate}

Таким способом создан общий секретный сеансовый ключ. В каждом новом сеансе используется этот же протокол для создания нового сеансового ключа.

Рассмотрим протокол Диффи~---~Хеллмана в ситуации, когда имеются три легальных пользователя $A,B,C$.

Каждая из сторон $A,B,C$ вырабатывает случайные числа $x,y,z$ соответственно и держит их в секрете.

\begin{enumerate}
    \item Первый этап обмена информацией аналогичен вышеописанному обмену информацией между двумя сторонами:
        \begin{enumerate}
            \item $A \rightarrow B: ~ g^x \mod p$.
            \item $B \rightarrow C: ~ g^y \mod p$.
            \item $C \rightarrow A: ~ g^z \mod p$.
        \end{enumerate}
    \item Второй этап состоит из передач сообщений:
        \begin{enumerate}
            \item $A \rightarrow B: ~ (g^z)^x = g^{zx} \mod p$.
            \item $B \rightarrow C: ~ (g^x)^y = g^{xy} \mod p$.
            \item $C \rightarrow A: ~ (g^y)^z = g^{yz} \mod p$.
        \end{enumerate}
    \item На завершающем третьем этапе стороны вычисляют:
        \begin{enumerate}
            \item $A: ~ K_A = (g^{yz})^x = g^{xyz} \mod p$.
            \item $B: ~ K_A = (g^{zx})^y = g^{xyz} \mod p$.
            \item $C: ~ K_A = (g^{xy})^z = g^{xyz} \mod p$.
        \end{enumerate}
\end{enumerate}

Как видно из произведенных действий, выработанные сторонами $A, B, C$ ключи совпадают: $K_A = K_B = K_C = K$. Следовательно, создан общий секретный сеансовый ключ $K$ для трех участников.

Таким же образом можно построить протокол Диффи~---~Хеллмана для любого числа легальных пользователей.

Рассмотрим этот двусторонний протокол с точки зрения криптоаналитика, желающего узнать ключ $K$. Предположим, ему удалось перехватить сообщения $g^{x}\mod p$ и $g^{y}\mod p $. Используя заранее известные данные $g,p $ и эти сообщения, криптоаналитик старается найти хотя бы одно из чисел $(x,y)$, то есть решить задачу дискретного логарифма. В настоящее время эта задача считается вычислительно трудной при обычно выбираемых значениях $p\sim 2^{1024}$.

Существует атака криптоаналитика, названная \textbf{<<человек посредине>>} (man-in-the-middle). Пусть имеются две легальные стороны $A$ и $B$ и нелегальная сторона $E$, криптоаналитик, который имеет возможность перехватывать сообщения как от $A$, так и от $B$:
    \[ A \leftrightsquigarrow E \leftrightsquigarrow B. \]
    %\[ A \leftrightarrow E \leftrightarrow B. \]

\begin{enumerate}
    \item Подмена ключей.
        \begin{enumerate}
            \item Сторона $A$ передает стороне $B$ сообщение:
                \[ A \overset{E}{\nrightarrow} B: ~ g^x \mod p. \]
            \item Сторона $E$ перехватывает сообщение $g^x \mod p$, сохраняет его и, зная $g$, передает стороне $B$ свое сообщение:
                \[ E \rightarrow B: ~ g^z \mod p. \]
            \item Сторона $B$ передает стороне $A$ сообщение:
                \[ A \overset{E}{\nleftarrow} B: ~ g^y \mod p. \]
            \item Сторона $E$ перехватывает сообщение $g^y \mod p$, сохраняет его и передает стороне $A$ свое сообщение
                \[ A \leftarrow E: ~ g^z \mod p \]
                или какое-то другое.
            \item Таким образом между сторонами $A$ и $E$ образуется общий секретный ключ $K_{AE}$, между $B$ и $E$ -- ключ $K_{BE}$, причем $A$ и $B$ не знают, что у них ключи со стороной $E$, а не с друг другом
                \[ \begin{array} {l}
                    K_{AE} = g^{xz} \mod p, \\
                    K_{BE} = g^{yz} \mod p. \\
                \end{array} \]

        \end{enumerate}
    \item Подмена сообщений.
        \begin{enumerate}
            \item Сторона $A$ посылает $B$ сообщение $m$, зашифрованное на ключе $K_{AE}$:
                % \rightsquigarrow
                \[ A \overset{E}{\nrightarrow} B: ~ E_{K_{AE}}(m). \]
            \item Сторона $E$ перехватывает сообщение, расшифровывает с ключом $K_{AE}$, возможно, подменяет на $m'$, зашифровывает с ключом $K_{BE}$ и посылает $B$:
                \[ E \rightarrow B: ~ E_{K_{BE}}(m'). \]
            \item То же самое происходит при обратной передаче от $B$ к $A$.
        \end{enumerate}
\end{enumerate}

Криптоаналитик $E$ перехватывает все сообщения, и может их подменять. Если по тексту письма нельзя обнаружить участие криптоаналитика в обмене информацией, то атака <<человек-посередине>>\index{криптоатака!человек-посередине} успешна.

Существует несколько протоколов для преодоления атаки этого типа.


%\section{Протоколы с аутентификацией}

\subsection{Односторонняя аутентификация}

\input{el-gamal_protocol}

\subsection{Взаимная аутентификация шифрованием}
\selectlanguage{russian}

К протоколам взаимной аутентификации принадлежит семейство протоколов, разработанных Т. Мацумото (T. Matsumoto), И. Такашима (Y. Takashima) и Х. Имаи (H. Imai) и названных по первым буквам фамилий авторов -- \textbf{протокол MTI}\index{протокол!MTI}.

Здесь к открытым данным относятся
    \[ p, ~~ g, ~~ \PK_A = g^a \mod p, ~~ \PK_B = g^b \mod p. \]
Каждый пользователь $A$ и $B$ обладает парой долговременных ключей для \emph{схемы шифрования с открытым ключом}: секретный ключ расшифрования $\SK$ и открытый ключ шифрования  $\PK$.
\[ \begin{array}{ll}
    A: & ~ \SK_A = a, ~~ \PK_A = g^a \mod p, \\
    B: & ~ \SK_B = b, ~~ \PK_B = g^b \mod p. \\
\end{array} \]

\textbf{Протокол MTI}
\begin{enumerate}
    \item Сторона $A$ генерирует случайное число $x, ~ 2\leq x\leq p-1$, создает и отправляет $B$ сообщение:
        \[ A \rightarrow B: ~ g^x \mod p. \]
    \item Сторона $B$ генерирует случайное число $y, ~ 2\leq y\leq p-1$, создает и отправляет $A$ сообщение:
        \[ A \leftarrow B: ~ g^y \mod p. \]
    \item Сторона $A$, используя открытые данные и полученное сообщение, создает сеансовый ключ:
        \[ K_A = (g^b)^x \cdot (g^y)^a = g^{bx+ay} \mod p. \]
    \item Сторона $B$, используя открытые данные и полученное сообщение, создает сеансовый ключ:
        \[ K_B = (g^x)^b \cdot (g^a)^y = g^{bx+ay} \mod p. \]
        Сеансовые ключи обоих сторон совпадают:
        \[ K_{A} =K_{B} = K. \]
\end{enumerate}

В описанном протоколе происходит взаимная аутентификация сторон как и в протоколе Эль-Гамаля\index{криптосистема!Эль-Гамаля}: открытые ключи сторон незаметно подменить невозможно. Наблюдая сообщения протокола, вычислить $g^{bx+ay}$ можно, только если известны значения $a,x$ или $b,y$, что представляет собой задачу дискретного логарифма, трудную в вычислительном смысле в настоящее время.


\input{sts}

\input{girault}

В этом разделе были рассмотрены протоколы, в которых ключи вырабатываются в процессе обмена информацией.
%Существует и другой подход, который будет рассмотрен в следующих разделах.

\chapter{Разделение секрета}

\section{Пороговые схемы}

Идея \textbf{пороговой} $(n, N)$-схемы\index{разделение секрета!пороговое} разделения общего секрета среди $N$ пользователей состоит в следующем.
%описывается так:
Доверенная сторона хочет распределить некий секрет $K_0$ между $N$ пользователями таким образом, что:
%. Поставлены следующие условия:
\begin{itemize}
    \item любые $m, ~ n \le m \le N$ легальных пользователей могут получить секрет (или доступ к секрету), если предъявят свои секретные ключи;
    \item любые $m, ~ m < n$, легальных пользователей не могут получить секрет и не могут определить (вычислить) этот секрет, пытаясь решить трудную в вычислительном смысле задачу.
\end{itemize}

Далее рассмотрены два случая: $(n, N)$-схема Шамира и простая $(N,N)$-схема.

\subsection[Схема Шамира]{Схема распределения секрета Шамира}
\selectlanguage{russian}


\subsubsection{Необходимые сведения из линейной алгебры}

Рассмотрим матрицу $V$ размера $(n \times n)$, где
\[
    V = \left(\begin{array}{cccc}
        {1} & {1} & { \ldots } & {1} \\
        {x_{1} } & {x_{2} } & { \ldots } & {x_{n} } \\
        {\begin{array}{l} {} \\ {x_{1}^{2} } \end{array}} &
            {\begin{array}{l} {} \\ {x_{2}^{2} } \end{array}} &
            {\begin{array}{l} {} \\ { \ldots } \end{array}} &
            {\begin{array}{l} {} \\ {x_{n}^{2} } \end{array}} \\
        {\begin{array}{l} { \ldots } \\ {x_{1}^{n-1} } \end{array}} &
            {\begin{array}{l} { \ldots } \\ {x_{2}^{n-1} } \end{array}} &
            {\begin{array}{l} { \ldots } \\ { \ldots } \end{array}} &
            {\begin{array}{l} { \ldots } \\ {x_{n}^{n-1} } \end{array}}
    \end{array}\right),
\]
где $x_{i}$ -- элемент поля $\Z_{p}$, $x_{i} \ne x_{j}$, $p$ -- большое простое\index{число!простое} число.

Определитель матрицы равен
    \[ \det V = \prod_{1\le i < j\le n} \left(x_j - x_i \right) \mod p. \]
В частности, при $n=2$ определитель
    \[ \det V = x_2 - x_1, \]
при $n=3$ определитель равен
    \[ \det V = (x_3 - x_2) (x_3 - x_1) (x_2 - x_1). \]

Если все элементы $x_{i}$ имеют различные значения, то $\det V \ne 0$, и матрица является невырожденной. Это значит, что существует обратная матрица.

Определим вектор-строку
    \[ \left(K_{0}, K_{1},  \ldots,  K_{n-1}\right), ~ K_{i} \in \Z_{p} \]
как решение уравнения
%Ставим задачу решить уравнение
    \[ (K_{0} ,K_{1} ,  \ldots, K{}_{n-1} )V=(y_{1} ,  \ldots, y_{n} ) \]
или, что эквивалентно, решение системы уравнений
\[ \begin{array}{l}
    y_1 = K_0 + K_1 x_1 + K_2 x_1^2 + \dots + K_{n-1} x_1^{n-1}, \\
    y_2 = K_0 + K_1 x_2 + K_2 x_2^2 + \dots + K_{n-1} x_2^{n-1}, \\
    \dots \\
\end{array} \]

Используя методы линейной алгебры, получим решение в виде
    \[ (K_{0} ,K_{1} ,  \ldots,  K_{n-1} )=(y_{1} ,  \ldots,  y_{n} )V^{-1}, \]
где $V^{-1}$ -- обратная матрица.

Однако прямого вычисления обратной матрицы здесь можно избежать. Введем для этого многочлены
\[ \begin{array}{c}
    K(x)= K_{0} +K_{1} x+ \ldots +K_{n-1} x^{n-1}, \\
    y_{i} =K(x_{i}). \\
\end{array} \]
Теперь решение этой задачи можно задать интерполяционной формулой Лагранжа:
\[
    K(x) = y_1 \frac{(x - x_2)(x - x_3) \dots (x - x_n)} {(x_1-x_2)(x_1-x_3) \dots (x_1-x_n)} +
\] \[
    + y_2 \frac{(x-x_1)(x-x_3) \dots (x-x_n)} {(x_2-x_1)(x_2-x_3) \dots (x_2-x_n)} + \dots +
\] \[
    + y_n \frac{(x-x_1)(x-x_2) \dots (x-x_{n-1})} {(x_n-x_1)(x_n-x_2) \dots (x_n-x_{n-1})}.
\]

Первое слагаемое равно нулю в точках $x_2, x_3, \dots, x_n$,   равно $y_1$ в точке $x_1$. Знаменатель не обращается в нуль, так как все $x_1, \dots, x_n$ имеют различные значения. Второе слагаемое равно $y_2$ в точке $x_2$, а при всех других значениях $x_i$ обращается в нуль. Аналогично обстоят дела с остальными слагаемыми.

Из всех коэффициентов $K_0, K_1, \dots, K_{n-1}$ нас интересует только $K_0$.
Положив $x=0$, получаем выражение для $K_{0} $ в виде
\[
    K(x) = (-1)^{n-1} y_1 \frac{x_2 x_3 \dots x_n} {(x_1-x_2) \dots (x_1-x_n)} +
\] \[
    + (-1)^{n-1} y_2 \frac{x_1 x_3 \dots x_n} {(x_2-x_1) \dots (x_2-x_n)} + \dots +
\] \[
    + (-1)^{n-1} y_n \frac{x_1 x_2 \dots x_{n-1}} {(x_n-x_1) \dots (x_n-x_{n-1})}.
\]

Интерполяционный многочлен Лагранжа принимает заданные значения в заданных точках. В нашей задаче $K(x)=K_{0}$ при $x=0$.


\subsubsection{Описание схемы Шамира}

В пороговой \textbf{схеме Шамира}\index{распределение секрета!Шамира} распределения секретов доверенная сторона предварительно производит следующие действия:
\begin{itemize}
    \item Выбирает большое простое\index{число!простое} число $p: ~ p \sim 2^{512} \dots 2^{1024}$.
    \item Выбирает $N$ различных чисел $x_1, x_2, \dots, x_N$, каждое из которых меньше $p$.
    \item Выбирает прямоугольную матрицу:
        \[
            V_{n \times N} = \left( \begin{array}{cccc}
                {1} & {1} & { \ldots } & {1} \\
                {x_{1} } & {x_{2} } & { \ldots } & {x_{N} } \\
                { \ldots } & { \ldots } & { \ldots } & { \ldots } \\
                {x_{1}^{n-1} } & {x_{2}^{n-1} } & { \ldots } & {x_{N}^{n-1} }
            \end{array} \right) \mod p.
        \]
    \item Выбирает секрет $K_0$, а также выбирает случайные числа $K_1, K_2, \dots, K_{n-1}$.
    \item Вычисляет частичные секреты -- числа $y_1, y_2, \dots, y_N$:
        \[ (y_1, y_2, \dots, y_N) = (K_0, K_1, \dots, K_{n-1}) V. \]
\end{itemize}

Распределение секрета $K_0$ между $N$ сторонами состоит в том, что доверенная сторона выдает легальному пользователю $i$  открытый ключ $\PK_i$, который известен всем, и секретный ключ $\SK_i$ (секрет только $i$-го пользователя):
\[ \begin{array}{ll}
    \PK_1 = x_1, & \SK_1 = y_1, \\
    \PK_2 = x_2, & \SK_2 = y_2, \\
    \cdots & \\
    \PK_N = x_N, & \SK_N = y_N. \\
\end{array} \]

Покажем, что такое распределение удовлетворяет поставленным требованиям.

Пусть собрались любые $n$ из общего числа $N$ пользователей, имеющих значения $(x_i, y_i)$. Каждому из них можно поставить в соответствие один столбец матрицы: $y_i = K(x_i)$. Как показано в предыдущем параграфе, это позволяет найти значение $K_0$.

Предположим, что собралось $m$ ($m<n$) пользователей. Заметим, что число неизвестных $K_0, K_1, \dots, K_{n-1}$ в системе уравнений осталось неизменным и равным $n$, а число уравнений меньше, так как $m<n$. В этом случае решение существует, но не является единственным. Если коэффициенты многочленов взяты из поля $\Z_p$, то число решений является конечным.

Например, если $m = n - 1$, тогда
    \[ K_0 + K_1 x_j + K_2 x_j^2 + \dots + K_{n-1} x_j^{n-1} = y_j, \]
и $K_0$ может принимать $p$ значений. Найти все решения перебором -- вычислительно трудная задача.

Если $m = n - 2$, то число различных решений равно $p^2$. Это число экспоненциально возрастает по мере уменьшения числа собравшихся вместе получателей секрета.

Таким образом, схема Шамира распределения секрета удовлетворяет предъявленным требованиям.

\subsubsection{Пример схемы Шамира}

Метод Шамира, называемый также схемой интерполяционных полиномов Лагранжа\index{многочлен!интерполяционный Лагранжа}, основывается на том, что для восстановления многочлена $f(x)$ степени $k-1$ необходимо и достаточно знать значения многочлена в любых $k$ разных точках.

Для секрета $M$ формируется многочлен
    \[ f(x) = \sum\limits_{i=1}^{k-1} a_i x^i + M, \]
где коэффициенты $a_i$ выбираются случайно. Вычисляются значения $y_i = f(x_i)$ в $n$ различных точках. Пользователю $i$ выдается тень $(x_i, y_i)$.

Для восстановления секрета по любым $k$ точкам $(x_i, y_i)$ используется интерполяционный многочлен Лагранжа:
    \[ f(x) = \sum\limits_{i=0}^{k-1} y_i \cdot l_i(x), ~~ l_i(x) = \prod\limits_{j=0, j \neq i}^{k-1} \frac{x - x_j}{x_i - x_j}. \]
Общий секрет $M$ является свободным коэффициентом $f(x)$.
    \[ M = \sum\limits_{i=0}^{k-1} y_i \prod\limits_{j=0, j \neq i}^{k-1} \frac{x_j}{x_j - x_i}. \]

\example
Приведем схему Шамира в поле $\GF{p}$. Для разделения секрета $M$ в $(3,n)$ схеме используется
    \[ f(x) = a x^2 + b x + M \mod p, \]
где $p$ -- простое\index{число!простое} число. Пусть $p=23$. Восстановим секрет $M$ по \emph{теням}
    \[ (1,14), (4,21), (15,6). \]

Последовательно вычисляем

    \[ M = \sum\limits_{i=0}^{k-1} y_i \prod\limits_{j=0, j \neq i}^{k-1} \frac{x_j}{x_j - x_i} \mod p = \]
    \[= 14 \cdot \frac{4}{4-1} \cdot \frac{15}{15-1} + 21 \cdot \frac{1}{1-4} \cdot \frac{15}{15-4} + 6 \cdot \frac{1}{1-15} \cdot \frac{4}{4-15} \mod 23 = \]
    \[ =14 \cdot \frac{4}{3} \cdot \frac{15}{14} + 21 \cdot \frac{1}{-3} \cdot \frac{15}{11} + 6 \cdot \frac{1}{-14} \cdot \frac{4}{-11} \mod 23 = \]
    \[= 20 - 7 \cdot 15 \cdot 11^{-1} + 12 \cdot 7^{-1} \cdot 11^{-1} \mod 23 = \]
    \[ = 13 \mod 23.\]
\exampleend


\input{xor_secret_sharing}

\section[Распределение секрета по коалициям]{Распределение секрета по \protect\\ коалициям}

\subsection{Схема для нескольких коалиций}

Предположим, что имеется $N$ легальных пользователей
    \[ \{ U_1, U_2, \dots, U_N \}, \]
которым нужно сообщить (открыть, получить доступ к) общий секрет $K$.

Секрет может быть доступен только определенным коалициям\index{распределение секрета!по коалициям}, например
\[ \begin{array}{l}
    C_1 = \{ U_1, U_2 \}, \\
    C_2 = \{ U_1, U_3, U_4 \}, \\
    C_3 = \{ U_2, U_3 \}, \\
    \dots
\end{array} \]
При этом ни одна из коалиций $C_i, ~ i = 1, 2, \dots$ не должна быть подмножеством другой коалиции.


\subsubsection{Пример 1}

Имеется 4 участника
    \[ \{ U_1, U_2, U_3, U_4 \}, \]
которые образуют 3 коалиции
\[ \begin{array}{l}
    C_1 = \{ U_1, U_2 \}, \\
    C_2 = \{ U_1, U_3 \}, \\
    C_3 = \{ U_2, U_3, U_4 \}. \\
\end{array} \]
Распределение частичных секретов между ними представлено в виде табл. \ref{tab:secret-share-coalition-1}, в которой введены следующие обозначения: $a_1, b_1, c_2, c_3$ -- случайные числа из кольца $\Z_M$. В строках таблицы содержатся частичные секреты каждого из пользователей, в столбцах таблицы показаны частичные секреты, соответствующие каждой из коалиций.

\begin{table}[!ht]
    \centering
    \caption{Распределение секрета по определенным коалициям\label{tab:secret-share-coalition-1}}
    \begin{tabular}{|c||c|c|c|}
        \hline
              & $C_1 = \{ U_1, U_2 \}$ & $C_2 = \{U_1, U_3 \}$ & $C_3 = \{ U_2, U_3, U_4 \}$ \\
        \hline \hline
        $U_1$ & $a_1$     & $b_1$     & -- \\
        $U_2$ & $K - a_1$ & --        & $c_2$ \\
        $U_3$ & --        & $K - b_1$ & $c_3$  \\
        $U_4$ & --        & --        & $K - c_2 - c_3$ \\
        \hline
    \end{tabular}
\end{table}

Как видно из приведенных данных, суммирование по модулю $M$ чисел, приведенных в каждом из столбцов таблицы, открывает секрет $K$.


\subsubsection{Пример 2}

%\section{Схема разделения секрета на монотонных булевых функциях}
%\example
В системе распределения секрета доверенный
%с использованием монотонных булевых функций
центр использует кольцо $\Z_m$ целых чисел по модулю $m$. Требуется разделить секрет $K$ между $5$ пользователями
    \[ \{ U_1, U_2, U_3, U_4, U_5 \} \]
так, чтобы восстановить секрет могли только коалиции
\[ \begin{array}{lll}
    C_1 = \{ U_1, U_2 \},      & & C_2 = \{ U_1, U_3 \}, \\
    C_3 = \{ U_2, U_3, U_4 \}, & & C_4 = \{ U_2, U_3, U_5 \}, \\
    C_5 = \{ U_3, U_4, U_5 \}, & & C_6 = \{ U_1, U_2, U_3 \}. \\
\end{array} \]

Заданное множество коалиций с доступом не является минимальным, так как одни коалиции входят в другие:
    \[ C_1 \subset C_6, ~ C_2 \subset C_6. \]
Исключая $C_6$, получим минимальное множество коалиций с доступом к секрету -- ни одна из оставшихся коалиций не входит в другую $C_i \nsubseteq C_j$ для $i \neq j$. Пользователям выдаются тени по минимальному множеству коалиций с доступом. В строках таблицы \ref{tab:secret-share-coalition-2} содержатся частичные секреты каждого из пользователей, в столбцах таблицы показаны частичные секреты, соответствующие каждой из коалиций.

\begin{table}[!ht]
    \centering
    \caption{Распределение секрета по определенным коалициям\label{tab:secret-share-coalition-2}}
    \begin{tabular}{|c||c|c|c|c|c|}
        \hline
              & $C_1$     & $C_2$     & $C_3$           & $C_4$           & $C_5$  \\
        \hline \hline
        $U_1$ & $a_1$     & $b_1$     & --              & --              & -- \\
        $U_2$ & $K - a_1$ & --        & $c_2$           & $d_2$           & --\\
        $U_3$ & --        & $K - b_1$ & $c_3$           & $d_3$           & $e_3$ \\
        $U_4$ & --        & --        & $K - c_2 - c_3$ & --              & $e_4$ \\
        $U_5$ & --        & --        & --              & $K - d_2 - d_3$ & $K - e_3 - e_4$ \\
        \hline
    \end{tabular}
\end{table}

Тени выбираются случайно из кольца $\mathbb{\Z}_m$. В результате у пользователей будут тени:
%\exampleend

\subsection{Схема Брикелла для нескольких коалиций}
\selectlanguage{russian}

Рассмотрим \textbf{схему Брикелла}\index{распределение секрета!Бриккела} (Brickell) распределения секрета по коалициям.

По-прежнему,
    \[ \{ U_1, U_2, \dots, U_N \} \]
легальные пользователи. Пусть $\Z_p$ -- кольцо целых чисел по модулю $p$. Рассмотрим векторы
    \[ \mathcal{U} = \left\{ (u_1, u_2, \dots, u_d) \right\}, ~~ u_i \in \Z_p \]
длины $d$. Каждому пользователю $U_i, ~ i = 1, \dots, N$ ставится в соответствие вектор
    \[ \varphi(U_i) \in \mathcal{U}, ~~ i = 1, \dots, N. \]

Тогда каждой из коалиций, например
    \[ C_1 = \{ U_1, U_2, U_3 \}, \]
соответствует набор векторов
    \[ \varphi(U_1), \varphi(U_2), \varphi(U_3). \]
Эти векторы должны быть выбраны так, чтобы их линейная оболочка \emph{содержала} вектор
    \[ (1, 0, 0, \dots, 0) \]
длины $d$. Линейная оболочка любого набора векторов, не образующих коалицию, \emph{не должна} содержать вектор $(1, 0, 0, \dots, 0)$ длины $d$.

Пусть $K_0 \in \Z_p$ -- общий секрет. Распределение секрета производится следующим образом. Сначала вычисляется вектор $(K_0, K_1, \dots, K_{d-1})$, где первая координата -- это общий секрет, а остальные координаты выбираются из $\Z_p$ случайно. Затем вычисляются скалярные произведения
\[
    \Big( \left( K_0, K_1, \dots, K_{d-1} \right), ~ \varphi(U_1) \Big) ~=~ a_1,
\] \[
    \Big( \left( K_0, K_1, \dots, K_{d-1} \right), ~ \varphi(U_2) \Big) ~=~ a_2,
\] \[
    \dots
\] \[
    \Big( \left( K_0, K_1, \dots, K_{d-1} \right), ~ \varphi(U_N) \Big) ~=~ a_N.
\]

Пользователям $U_i, ~ i = \overline{1,N}$ выдаются их частичные секреты
    \[ U_i \colon \left\{ \varphi(U_i), a_i \right\}. \]

Пусть коалиция $C$ -- допустимая, например
    \[ C = C_1 = \{ U_1, U_2, U_3 \}. \]
Тогда члены коалиции совместно находят такие коэффициенты $\lambda_1, \lambda_2, \lambda_3$, что
    \[ \lambda_1\varphi(U_1)+\lambda_2\varphi(U_2)+\lambda_3\varphi(U_3) ~=~ (1,0, \dots, 0). \]
После этого вычисляется выражение
\[
    \lambda_1 a_1 + \lambda_2 a_2 + \lambda_3 a_3 =
\] \[
    = \Big( \left( K_0, K_1, \dots, K_{d-1} \right), ~ \lambda_1 \varphi(U_1) + \lambda_2 \varphi(U_2) + \lambda_3 \varphi(U_3) \Big) =
\] \[
    = \Big( \left( K_0, K_1, \dots, K_{d-1} \right), ~ \left( 1, 0, \dots, 0 \right) \Big) =  K_0,
\]
которое и является общим секретом.

%\section{Схема разделения секрета в векторном пространстве Бриккела}
%
%В схеме Бриккела для $n$ пользователей $\{ U_1, U_2, \ldots, U_n \}$ Центр выбирает $k$-мерные векторы $\varphi(U_i)$ над полем $\mathbb{\Z}_p$ так, чтобы их линейная нетривиальная комбинация над полем $\mathbb{\Z}_p$ могла равняться единичному вектору
%    \[ (1,0,0, \ldots, 0) = \sum\limits_{i=1}^{n} c_i \ \varphi(U_i), ~ c_i \in  \mathbb{\Z}_p. \]
%Центр пользователю $i$ присваивает \emph{открытый, публично доступный} вектор $\varphi(U_i)$.
%
%Для разделения секрета $K$ Центр выбирает случайные числа $a_2, a_3, \ldots, a_n$, составляет вектор $\bar{a} = (K, a_2, a_3, \ldots, a_n)$ и выдает каждому пользователю \emph{секретную} тень $s_i = \bar{a} \cdot \varphi(U_i)$.
%
%Восстановление секрета производится
%    \[ K = \bar{a} \cdot (1,0,\ldots,0) = \bar{a} \cdot \sum\limits_{i=1}^{n} c_i \varphi(U_i) = \sum\limits_{i=1}^{n} c_i s_i, \]
%так как из открытых векторов $\varphi(U_i)$ пользователи могут найти $c_i$.
%
%Восстановить секрет могут только те коалиции пользователей, для которых нетривиальная комбинация векторов $\varphi(U_i)$ дает единичный вектор.

\example
Для сети из $n=4$ участников
    \[ \{ U_1, U_2, U_3, U_4 \} \]
выбраны следующие векторы длины $k=3$ над полем $\Z_{23}$:
\[ \begin{array}{l}
    \varphi(U_1) = (0,2,0), \\
    \varphi(U_2) = (2,0,7), \\
    \varphi(U_3) = (0,5,7), \\
    \varphi(U_4) = (0,2,9). \\
\end{array} \]
Найдем все коалиции, которые могут раскрыть секрет.

Запишем
    \[ (1,0,0) = c_1 (0,2,0) + c_2 (2,0,7) + c_3 (0,5,7) + c_4 (0,2,9). \]
Ясно, что $c_2 \neq 0$, и коалициями пользователей, которые дают единичный вектор и, следовательно, могут восстановить секрет, являются:
\[ \begin{array}{l}
    C_1 = \{ U_1, U_2, U_3 \}, \\
    C_2 = \{ U_1, U_2, U_4 \}, \\
    C_3 = \{ U_2, U_3, U_4 \}. \\
\end{array} \]

Пусть доверенный Центр для секрета $K=4$ выбрал вектор $\bar{a} = (4, 2, 9)$. Тогда участники получают тени:
    \[ s_1 = (4,2,9) \cdot (0,2,0) = 4 \mod 23, \]
    \[ s_2 = (4,2,9) \cdot (2,0,7) = 2 \mod 23, \]
    \[ s_3 = (4,2,9) \cdot (0,5,7) = 4 \mod 23, \]
    \[ s_4 = (4,2,9) \cdot (0,2,9) = 16 \mod 23. \]

Возьмем коалицию $C_1 = \{ U_1, U_2, U_3 \}$ и вычислим коэффициенты $c_i$:
    \[ (1,0,0) = c_1 (0,2,0) + c_2 (2,0,7) + c_3 (0,5,7), \]
\[ \begin{array}{l}
    c_1 = 7 \mod 23, \\
    c_2 = 12 \mod 23, \\
    c_3 = 11 \mod 23. \\
\end{array} \]

Найдем секрет:
    \[ K = 7 \cdot 4 + 12 \cdot 2 + 11 \cdot 4 = 4 \mod 23.\]
\exampleend


\subsection{Схема Блома распределения парных ключей}
\selectlanguage{russian}

Рассмотрим распределение ключей по \textbf{схеме Блома}\index{распределение секрета!Блома} (Blom), в которой каждые два пользователя из общего числа $N$ пользователей имеют доступ к общему секретному ключу, ключи различных пар различны.

По-прежнему,
    \[ \{ U_1, U_2, \dots, U_N \} \]
легальные пользователи, $\Z_p$ -- кольцо целых чисел.

Построим симметричный многочлен
    \[ f(x,y) = \sum_{i=1}^k \sum_{j=1}^k a_{ij} x^i y^j, \]
    \[ a_{ij} \in \Z_p, ~ a_{ij} = a_{ji}. \]

Возьмем набор чисел $r_1, r_2, \dots, r_N$, где $r_i$ -- открытый ключ пользователя $U_i$, ~ $r_i \in \Z_p$.

Каждый пользователь $U_i$ получает многочлен $f(x,y)$ и вместо $y$ подставляет свое значение $r_i$, так что получается $N$ многочленов $f(x, r_i), ~ i = \overline{1,N}$.

Каждые два участника коалиции должны иметь общий ключ. Пусть, например, $U_1$ и $U_2$ хотят создать общий ключ. Тогда пользователь $U_1$, используя $f(x, r_1)$ и зная $r_2$, вычисляет
    \[ K_{21} = f(r_2, r_1). \]

Пользователь $U_2$, используя $f(x, r_2)$ и зная $r_1$, вычисляет
    \[ K_{12} = f(r_1, r_2). \]

Так как для выбранного многочлена справедливо равенство
    \[ f(r_1, r_2) = f(r_2, r_1), \]
то
    \[ K_{12} = K_{21}. \]
Таким образом, два участника коалиции создали общий ключ. Таким же образом поступают и другие пары пользователей. Третий пользователь, не участник коалиции, не может подобрать ключ, так как это представляет собой трудную задачу в вычислительном смысле.

%\section{Пример предварительного распределения ключей и разделения секрета в схеме Блома в виде многочленов}
\example
В схеме распределения ключей Блома для $N=4$ пользователей доверенный центр выбирает:
\begin{enumerate}
    \item модуль $p = 17$ поля $\GF{p}$;
    \item свой секретный симметричный многочлен от двух переменных
        \[ f(x,y) = a + b (x + y) + c x y \mod p \]
        над полем $\GF{p}$;
    \item открытые ключи для каждого пользователя
        \[ r_1 = 5, ~ r_2 = 9, ~ r_3 = 14, ~ r_4 = 3; \]
    \item вычисляет и секретно раздает многочлен $S_i(x)$ каждому пользователю $U_i$:
        \[ \begin{array}{l}
            S_1(x) = f(x, r_1) = 1 + 2x \mod p, \\
            S_2(x) = f(x, r_2) = 3 + 10x \mod p, \\
            S_3(x) = f(x, r_3) = 14 + 3x \mod p, \\
            S_4(x) = f(x, r_4) = 0 + 15x \mod p. \\
        \end{array} \]
\end{enumerate}
Найдем ключи и восстановим секретный многочлен доверенного центра.

Секретные сеансовые ключи пользователей равны
    \[ K_{ij} = K_{ji} = S_i(r_j) = S_j(r_i): \]
\[ \begin{array}{lcl}
    K_{12} = K_{21} = 2, & & K_{13} = K_{31} = 12, \\
    K_{14} = K_{41} = 7, & & K_{23} = K_{32} = 7, \\
    K_{24} = K_{42} = 16, & & K_{34} = K_{43} = 6. \\
\end{array} \]

По любым 3 многочленам пользователей можно восстановить секретный многочлен Центра. Коэффициенты секретного многочлена Центра равны $a=7, b=9, c=2$.
\exampleend

%\section{Схема предварительного распределения ключей и разделения секрета Блома}
%
%Центр выбирает секретную симметрическую $(k \times k)$-матрицу $F$ над полем $\GF{p}$, где $p$ -- простое. Каждому пользователю $i$ Центр создает и выдает открытый ключ $P_i$, который является $k$-мерным вектором над $\GF{p}$, и секретный ключ $S_i = F \cdot P_i$, $k$-мерный вектор.
%
%Когда два пользователя $i$ и $j$ хотят создать секретный сессионный ключ для обмена сообщениями, они обмениваются открытыми ключами $P_i, P_j$ и вычисляют секретный сессионный ключ $K_{ij} = S_i P_j^T = S_j P_i^T$.
%
%%Если известны открытые ключи $k$ пользователей, то...
%%TODO
%
%%Схема Блома используется в High-bandwidth Digital Content Protection (HDCP), разрботанной Intel для применения в DVD плеерах и телевидении высокой четкости.


\chapter{Примеры систем защиты}

\section{Система Kerberos для локальной сети}
\selectlanguage{russian}

Система аутентификации и распределения ключей Kerberos основана на протоколе Нидхема~---~Шрёдера. Самые известные реализации протокола Kerberos включают Microsoft Active Directory и ПО Kerberos с открытым кодом для Unix.

Протокол предназначен для решения задачи аутентификации и распределения ключей в рамках локальной сети, в которой есть группа пользователей, имеющих доступ к набору сервисов, и требуется обеспечить единую аутентификацию для всех сервисов. Протокол Kerberos использует только симметричное шифрование. Секретный ключ используется для взаимной аутентификации.

Естественно, что в нелокальной сети Интернет невозможно секретно создать и распределить пары секретных ключей, и поэтому Kerberos построен для (виртуальной) локальной сети.

В протоколе используется 4 типа субъектов:

\begin{itemize}
    \item пользователи системы $C_i$,
    \item сервисы $S_i$, доступ к которым имеют пользователи,
    \item сервер аутентификации AS (Authentication Server), который производит аутентификацию пользователей по паролям и/или смарт-картам только один раз и выдает секретные сеансовые ключи для дальнейшей аутентификации,
    \item сервер выдачи мандатов TGS (ticket granting server) для аутентификации доступа к запрашиваемым сервисам, аутентификация выполняется по сеансовым ключам\index{ключ!сеансовый}, выданным сервером AS.
\end{itemize}

Для работы протокола требуется заранее распределить следующие секретные симметричные ключи для взаимной аутентификации.
\begin{itemize}
    \item Ключи $K_{C_i}$ между пользователем $i$ и сервером AS. Как правило, ключом служит обычный пароль\index{пароль}, точнее, результат хэширования пароля. Может быть использована и смарт-карта.
    \item Ключ $K_{TGS}$ между серверами AS и TGS.
    \item Ключи $K_{S_i}$ между сервисами $S_i$ и сервером TGS.
\end{itemize}

\begin{figure}[!ht]
	\centering
	\includegraphics[width=\textwidth]{pic/kerberos}
	\caption{Схема аутентификации и распределения ключей Kerberos\label{fig:kerberos}}
\end{figure}

На рис. \ref{fig:kerberos} представлена схема протокола, состоящая из 6 шагов.

Введем обозначения для протокола между пользователем $C$ с ключом $K_C$ и сервисом $S$ с ключом $K_S$.
\begin{itemize}
    \item $ID_C, ID_{TGS}, ID_S$ -- идентификаторы пользователя, сервера TGS и сервиса $S$ соответственно,
    \item $t_i, \tilde{t}_i$ -- запрашиваемые и выданные границы времени действия сеансовых ключей аутентификации,
    \item $ts_i$ -- метка текущего времени (timestamp),
    \item $N_i$ -- одноразовая метка (nonce)\index{одноразовая метка}, псевдослучайное число для защиты от атак воспроизведения сообщений,
    \item $K_{C,TGS}, K_{C,S}$ -- выданные сеансовые ключи аутентификации пользователя и сервера TGS, пользователя и сервиса $S$, соответственно,
    \item $T_{TGS} = E_{K_{TGS}}(K_{C,TGS} ~\|~ ID_C ~\|~ \tilde{t}_1)$ -- мандат (ticket) для TGS, который пользователь расшифровать не может,
    \item $T_{S} = E_{K_S}(K_{C,S} ~\|~ ID_C ~\|~ \tilde{t}_2)$ -- мандат для сервиса $S$, который пользователь расшифровать не может,
    \item $K_1, K_2$ -- обмен информацией для генерирования общего секретного симметричного ключа дальнейшей коммуникации, например, по протоколу Диффи~---~Хеллмана\index{протокол!Диффи~---~Хеллмана}.
\end{itemize}

Схема протокола следующая.
\begin{enumerate}
    \item Первичная аутентификация пользователя по паролю, получение сеансового ключа $K_{C,TGS}$ для дальнейшей аутентификации. Это действие выполняется один раз для каждого пользователя, чтобы уменьшить риск компроментации пароля.
        \begin{enumerate}
            \item $C \rightarrow AS: ~~ ID_C ~\|~ ID_{TGS} ~\|~ t_1 ~\|~ N_1$.
            \item $C \leftarrow AS: ~~ ID_C ~\|~ T_{TGS} ~\|~ E_{K_C}( K_{C,TGS} ~\|~ \tilde{t}_1 ~\|~ N_1 ~\|~ ID_{TGS})$.
        \end{enumerate}
    \item Аутентификация сеансовым ключом $K_{C,TGS}$ на сервере TGS для запроса доступа к сервису выполняется один раз для каждого сервиса. Получение другого сеансового ключа аутентификации $K_{C,S}$.
        \begin{enumerate}
            \item $C \rightarrow TGS: ~~ ID_S ~\|~ t_2 ~\|~ N_2 ~\|~ T_{TGS} ~\|~ E_{K_{C,TGS}}(ID_C ~\|~ ts_1)$.
            \item $C \leftarrow TGS: ~~ ID_C ~\|~ T_{S} ~\|~ E_{K_{C,TGS}}( K_{C,S} ~\|~ \tilde{t}_2 ~\|~ N_2 ~\|~ ID_S)$.
        \end{enumerate}
    \item Аутентификация сеансовым ключом $K_{C,S}$ на сервисе $S$ -- создание общего сеансового ключа дальнейшего взаимодействия.
        \begin{enumerate}
            \item $C \rightarrow S: ~~ T_{S} ~\|~ E_{K_{C,S}}(ID_C ~\|~ ts_2 ~\|~ K_1)$.
            \item $C \leftarrow S: ~~ E_{K_{C,S}}( ts_2 ~\|~ K_2)$.
        \end{enumerate}
\end{enumerate}

Аутентификация и проверка целостности достигается сравнением идентификаторов, одноразовых меток и меток времени внутри зашифрованных сообщений после расшифрования с их действительными значениями.

Некоторым недостатком схемы является необходимость синхронизации часов между субъектами сети.


\input{pgp}

\input{tls}

\input{ipsec}

\section[Защита персональных данных в мобильной связи]{Защита персональных данных в \protect\\ мобильной связи}

\subsection{GSM2}
\selectlanguage{russian}

Регистрация телефона в сети GSM2 построена с участием трех сторон: SIM-карты мобильного устройства, базовой станции и центра аутентификации. SIM-карта и центр аутентификации обладают общим секретным 128-битовым ключом $K_i$. Вначале телефон сообщает базовой станции уникальный идентификатор SIM-карты IMSI открытым текстом. Базовая станция запрашивает в Центре аутентификации для данного IMSI набор параметров для аутентификации. Центр генерирует псевдослучайное 128-битовое число $\textrm{RAND}$ и алгоритмами A3 и A8\index{алгоритм!A3, A5, A8} создает симметричный 54-битовый ключ $K_c$ и 32-битовый аутентификатор $\textrm{RES}$. Базовая станция передает мобильному устройству число $\textrm{RAND}$ и ожидает результат вычисления SIM-картой числа $\textrm{XRES}$, которое должно совпадать с $\textrm{RES}$ в случае успешной аутентификации. Схема аутентификации показана на рис. \ref{fig:gsm2}.

\begin{figure}[!ht]
	\centering
	\includegraphics[width=0.85\textwidth]{pic/gsm2}
	\caption{Односторонняя аутентификация и шифрование в GSM2\label{fig:gsm2}}
\end{figure}

Все вычисления для аутентификации выполняет SIM-карта. Ключ $K_c$ далее используется для создания ключа шифрования каждого фрейма $K = K_c \| n_F$, где $n_F$~--~22-битный номер фрейма. Шифрование выполняет уже само мобильное устройство. Алгоритм шифрования фиксирован в каждой стране и выбирается из семейства алгоритмов A5 (A5/1, A5/2, A5/3). В GSM2 применяется либо алгоритм A5/1, либо A5/2 (используется в России). Алгоритм A5/3 применяется уже в сети GSM3.

Аутентификация в сети GSM2 односторонняя. При передаче данных не используются проверка целостности и аутентификация сообщений. Передача данных между базовыми станциями происходит в открытом незашифрованном виде. Алгоритмы шифрования A5/1 и A5/2 не стойкие, количество операций для взлома A5/1~--~$2^{40}$, A5/2~--~$2^{16}$. Кроме того, длина ключа $K_c$ всего 54 бита. Передача в открытом виде уникального идентификатора IMSI позволяет однозначно определить абонента.


\input{gsm3}

%\section{Беспроводная сеть Wi-Fi}
%\subsection{WPA-PSK2, 802.11n, Radix?}
%\subsection{Wimax 802.16(?)}

\chapter{Аутентификация пользователя}


\section{Многофакторная аутентификация}

Для защищенных приложений применяется \textbf{многофакторная} аутентификация одновременно по факторам различной природы:
\begin{enumerate}
    \item Свойство, которым обладает субъект. Например, биометрия, природные уникальные отличия: лицо, отпечатки пальцев, радужная оболочка глаз, капиллярные узоры, последовательность ДНК.
    \item Знание -- информация, которую знает субъект. Например, пароль, PIN-код.
    \item Владение -- вещь, которой обладает субъект. Например, электронная или магнитная карта, флеш-память.
%    \item Факторы присвоения. Например, номер машины, RFID-метка.
\end{enumerate}

В обычных массовых приложениях из-за удобства использования применяется аутентификация только по \textbf{паролю}\index{пароль}, который является общим секретом пользователя и информационной системы. Биометрическая аутентификация по отпечаткам пальцев применяется существенно реже. Как правило, аутентификация по отпечаткам пальцев является дополнительным, а не вторым обязательным фактором (тоже из-за удобства ее использования).

%Так же явно или неявно используется аутентификация по факторам:
%\begin{enumerate}
%    \item Социальная сеть. Доверие к индивидууму в личном общении или интернет на основании общих связей.
%    \item Географическое положение. Например, для проверки оплаты товаров по кредитной карте.
%    \item Время. Доступ к сервисам или местам только в определенное время.
%    \item И др.
%\end{enumerate}


\section[Энтропия и криптостойкость паролей]{Энтропия и криптостойкость \protect\\ паролей}

Стандартный набор символов паролей, которые можно набрать на клавиатуре, используя английские буквы и небуквенные символы, состоит из $D=94$ символов. При длине пароля $L$ символов и предположении равновероятного использования символов энтропия паролей равна
    \[ H = L \log_2 D. \]

Клод Шеннон, исследуя энтропию символов английского текста, изучал вероятность успешного предсказания людьми следующего символа по первым нескольким символам слов или текста. В результате Шеннон получил оценку энтропии первого символа $s_1$ текста порядка $H(s_1) \approx 4{,}6$--$4{,}7$ бит/символ и оценки энтропий последующих символов, постепенно уменьшающиеся до $H(s_9) \approx 1{,}5$ бит/символ для 9-го символа. Энтропия для длинных текстов литературных произведений получила оценку $H(s_\infty) \approx 0{,}4$ бит/символ.

Статистические исследования баз паролей показывают, что наиболее часто используются буквы <<a>>, <<e>>, <<o>>, <<r>> и цифра <<1>>.

NIST использует следующие рекомендации для оценки энтропии паролей\index{энтропия!пароля}, создаваемых людьми.
\begin{enumerate}
    \item Энтропия первого символа $H(s_1) = 4$ бит/символ.
    \item Энтропия со 2-го по 8-й символы $H(s_{2 \leq i \leq 8}) = 2$ бит/символ.
    \item Энтропия с 9-го по 20-й символы $H(s_{9 \leq i \leq 20}) = 1{,}5$ бит/символ.
    \item Энтропия с 21-го символа $H(s_{i \geq 21}) = 1$ бит/символ.
    \item Проверка композиции на использование символов разных регистров и небуквенных символов добавляет до 6 бит энтропии пароля.
    \item Словарная проверка на слова и часто используемые пароли добавляет до 6 бит энтропии для коротких паролей. Для 20-символьных и более длинных паролей прибавка к энтропии 0 бит.
\end{enumerate}

Для оценки энтропии пароля нужно сложить энтропии символов $H(s_i)$ и сделать дополнительные надбавки, если пароль удовлетворяет тестам на композицию и отсутствие в словаре.

\begin{table}[!ht]
    \centering
    \caption{Оценка NIST предполагаемой энтропии паролей\label{tab:password-entropy}}
    \resizebox{\textwidth}{!}{ \begin{tabular}{|c||c|c|c||c|}
        \hline
        \multirow{2}{*}{\parbox{1.5cm}{Длина пароля, символы}} & \multicolumn{3}{|c||}{\parbox{6cm}{Энтропия паролей пользователей по критериям NIST}} & \multirow{2}{*}{\parbox{2.5cm}{Энтропия случайных равновероятных паролей}} \\
        \cline{2-4}
        & \parbox{1.5cm}{Без проверок} & \parbox{2cm}{Словарная проверка} & \parbox{2.5cm}{Словарная и композиционная проверка} & \\
        \hline
        4  & 10 & 14 & 16 & 26.3 \\
        6  & 14 & 20 & 23 & 39.5 \\
        8  & 18 & 24 & 30 & 52.7 \\
        10 & 21 & 26 & 32 & 65.9 \\
        12 & 24 & 28 & 34 & 79.0 \\
        16 & 30 & 32 & 38 & 105.4 \\
        20 & 36 & 36 & 42 & 131.7 \\
        24 & 40 & 40 & 46 & 158.0 \\
        30 & 46 & 46 & 52 & 197.2 \\
        40 & 56 & 56 & 62 & 263.4 \\
        \hline
    \end{tabular} }
\end{table}

В табл. \ref{tab:password-entropy} приведена оценка NIST на величину энтропии пользовательских паролей в зависимости от их длины и сравнение с энтропией случайных паролей с равномерным распределением символов из набора в $D=94$ символов клавиатуры. Вероятное число попыток для подбора пароля составляет $O(2^H)$. Из таблицы видно, что по критериям NIST энтропия реальных паролей в 2--4 раза меньше энтропии случайных паролей с равномерным распределением символов.

\example
Оценим общее количество существующих паролей. Население Земли -- 7 млрд. Предположим, что все население использует компьютеры, Интернет, и у каждого человека по 10 паролей. Общее количество существующих паролей -- $7 \cdot 10^{10} \approx 2^{36}$.
%Следовательно, \emph{реальная энтропия паролей не превышает 36 бит}.

Имея доступ к наиболее массовым интернет-сервисам с количеством пользователей десятки и сотни миллионов, в которых пароли часто хранятся в открытом виде из-за необходимости обновления ПО и, в частности, выполнения аутентификации, мы 1) имеем базу паролей, покрывающую существенную часть пользователей, 2) можем статистически построить правила генерирования паролей. Даже если пароль хранится в защищенном виде, то при вводе пароль, как правило, в открытом виде пересылается по Интернету, и все преобразования пароля для аутентификации осуществляет интернет-сервис, а не веб-браузер. Следовательно, интернет-сервис имеет доступ к исходному паролю.
\exampleend

В 2002 г. был подобран ключ для 64-битового блокового шифра RC5 сетью \texttt{distributed.net} персональных компьютеров, выполнявших вычисления в фоновом режиме. Суммарное время вычислений всех компьютеров -- 1757 дней, было проверено 83\% пространства всех ключей. Это означает, что пароли с оценочной энтропией менее 64 бит, то есть \emph{все пароли} до 40 символов по критериям NIST, могут быть подобраны в настоящее время. Конечно, с оговорками на то, что 1) нет ограничений на количество и скорость попыток аутентификаций, 2) алгоритм генерирования вероятных паролей эффективен.

Строго говоря, использование даже 40-символьного пароля для аутентификации или в качестве ключа блокового шифрования является небезопасным.


\subsubsection{Число паролей}

Приведем различные оценки числа паролей, создаваемых людьми.

Пароли, создаваемые людьми, основаны на словах или закономерностях естественного языка. В английском языке всего около $1\ 000\ 000 \approx 2^{20}$ слов, включая термины.

%http://www.springerlink.com/content/bh216312577r6w64/fulltext.pdf
%http://www.antimoon.com/forum/2004/4797.htm

Используемые слоги английского языка имеют вид V, CV, VC, CVV, VCC, CVC, CCV, CVCC, CVCCC, CCVCC, CCCVCC, где C -- согласная (consonant), V -- гласная (vowel). 70\% слогов имеют структуру VC или CVC. Общее число слогов $S = 8000 - 12000$. Средняя длина слога -- 3 буквы.

Предполагая равновероятное распределение всех слогов английского языка, для числа паролей из $r$ слогов получим верхнюю оценку
    \[ N_1 = S^r = 2^{13 r} \approx 2^{4.3 L_1}. \]
Средняя длина паролей составит
    \[ L_1 \approx 3 r. \]

Теперь предположим, что пароли могут состоять только из 2--3 буквенных слогов вида CV, VC, CVV, VCC, CVC, CCV с равновероятным распределением символов. Подсчитаем число паролей $N_2$, которые могут быть построены из $r$ таких слогов. В английском алфавите $n_v = 10, n_c = 16, n = n_v + n_c = 26$. Верхняя оценка числа $r$-слоговых паролей:
    \[ N_2 = (n_c n_v + n_v n_c + n_c n_v n_v + n_v n_c n_c + n_c n_v n_c + n_c n_c n_v)^r \approx \]
        \[ \approx \left( n_c n_v(3 n_c + n_v) \right)^r, \]
    \[ N_2 \approx \left( \frac{n^3}{2} \right)^r \approx 2^{13 r} \approx 2^{4.3 L_2}. \]
Средняя длина паролей:
    \[ L_2 = \frac{n_c n_v(2 + 2 + 3 n_v + 3 n_c + 3 n_c + 3 n_c)}{n_c n_v (1 + 1 + n_v + n_c + n_c + n_c)} \cdot r \approx 3 r. \]

Как видно, получились одинаковые оценки числа и длины паролей.

Подсчитаем верхние оценки числа паролей из $L$ символов, предполагая равномерное распределение символов из алфавита в $D$ символов: a) $D_1 = 26$ строчных буквы, б) все $D_2 = 94$ печатных символа клавиатуры (латиница и небуквенные символы):
    \[ N_3 = D_1^L \approx 2^{4.7 L}, \]
    \[ N_4 = D_2^L \approx 2^{6.6 L}. \]

\begin{table}[!ht]
    \centering
    \caption{Различные верхние оценки числа паролей\label{tab:password-number}}
    \resizebox{\textwidth}{!}{ \begin{tabular}{|c||c|c|c|}
        \hline
        \multirow{2}{*}{\parbox{1.5cm}{Длина пароля}} & \multicolumn{3}{|c|}{Число паролей} \\
        \cline{2-4}
            & \parbox{3cm}{На основе слоговой композиции} &
            \parbox{3cm}{Алфавит $D=26$ символов} &
            \parbox{3cm}{Алфавит $D=94$ символа} \\
        \hline \hline
        6  & $2^{26}$ & $2^{28}$ & $2^{39}$ \\
        9  & $2^{39}$ & $2^{42}$ & $2^{59}$ \\
        12 & $2^{52}$ & $2^{56}$ & $2^{79}$ \\
        15 & $2^{65}$ & $2^{71}$ & $2^{98}$ \\
        \hline
        21 & $2^{91}$ & $2^{99}$ & $2^{137}$ \\
        \hline
        39 & $2^{169}$ & $2^{183}$ & $2^{256}$ \\
        \hline
    \end{tabular} }
\end{table}

Из таблицы \ref{tab:password-number} видно, что при доступном объеме вычислений в $2^{60 \ldots 70}$ операций, пароли вплоть до 15 символов, построенные на словах, слогах, изменениях слов, вставках цифр, небольшом изменении регистров и других простейших модификациях, могут быть найдены перебором на кластере (или ПК) в настоящее время.

Для достижения криптостойкости паролей, сравнимой со 128- или 256-битовым секретным ключом, требуется выбирать пароль из 20 и 40 символов соответственно, что, как правило, не реализуется из-за сложности запоминания и ввода без ошибок.


%Подсчитаем число паролей $N_1$, которые могут могут построены из $r$ ~ 2-3 буквенных слогов: $cv, vc, ccv, cvc, vcc$, где $c$ -- согласная, $v$ -- гласная. В английском алфавите $n_v = 10, n_c = 16, n = n_v + n_c = 26$. Число паролей
%    \[ N_1 = \left( n_v n_c (1 + 1 + n_c + n_c + n_c) \right)^r \approx 3^r n_v^r n_c^{2r}. \]
%Средняя длина паролей
%    \[ L = r \left( \frac{2 + 2 + 3 n_c + 3 n_c + 3 n_c}{1 + 1 + n_c + n_c + n_c} \right) \approx 3r. \]
%
%%Учтем, что $b \leq r$ символов могут быть заглавными: $N_1 \rightarrow N_2 < N_1 \binom{L}{b} \left( \frac{n}{n_v} \right)^b$. Вставим $d$ цифр в случайные места: $N_2 \rightarrow N_3 = N_2 (10 (1 + L))^d \approx N_2 (10 L)^d$.
%%
%%Общее число паролей
%%    \[ N = N_3 = 3^r 10^r 16^{2r} \binom{3r}{b} 2.6^b \left(10 \cdot 3 r \right)^d. \]
%%
%%\begin{table}[!ht]
%%    \centering
%%    \small
%%    \begin{tabular}{|c|c|c|c|c||cr|}
%%        \hline
%%        \parbox{1.3cm}{Слогов, $r$} & \parbox{1.8cm}{Заглавных букв, $b$} & \parbox{1.5cm}{Вставок цифр, $d$} & \parbox{2.8cm}{Средняя длина пароля, $L+d$} & \parbox{3cm}{Верхняя оценка числа паролей $N$} & \multicolumn{2}{|c|}{\parbox{3.2cm}{Число всех паролей}} \\
%%        \hline
%%        $2$ & $0$ & $0$ & $6$ & $2^{26}$ & $2^{36}$ & a-z \\
%%        $2$ & $2$ & $0$ & $6$ & $2^{32}$ & $2^{48}$ & A-Z, a-z \\
%%        $2$ & $2$ & $2$ & $8$ & $2^{45}$ & $2^{48}$ & A-Z, a-z, 0-9 \\
%%        \hline
%%        $3$ & $0$ & $0$ & $9$ & $2^{39}$ & $2^{54}$ & a-z \\
%%        $3$ & $3$ & $0$ & $9$ & $2^{49}$ & $2^{54}$ & A-Z, a-z \\
%%        $3$ & $3$ & $2$ & $11$ & $2^{63}$ & $2^{65}$ & A-Z, a-z, 0-9 \\
%%        \hline
%%        $4$ & $0$ & $0$ & $12$ & $2^{52}$ & $2^{93}$ & a-z \\
%%        $4$ & $3$ & $0$ & $12$ & $2^{64}$ & $2^{186}$ & A-Z, a-z \\
%%        $4$ & $3$ & $2$ & $14$ & $2^{78}$ & $2^{222}$ & A-Z, a-z, 0-9 \\
%%        \hline
%%    \end{tabular}
%%    \caption{Сравнение верхней оценки числа паролей, построенных на слогах, со всем доступным множеством паролей.}
%%    \label{tab:password-number}
%%\end{table}
%
%Учтем, что $b$ символов в пароле могут быть взяты не из 26-символьного алфавита строчных букв, а из всего алфавита в $D=94$ печатных символа клавиатуры (латиница и небуквенные символы):
%\[
%    \begin{array}{ll}
%    b=1 & N_1 \rightarrow N_2 = \frac{n_v}{n_v+n_c} 3^r n_v^{r-1} n_c^{2r} \cdot L. \]
%
%    \[ N_1 \rightarrow N_2 < N_1 \binom{L}{b} \left( \frac{D}{n_v} \right)^b. \]
%
%
%
%Общее число паролей
%    \[ N < 3^r n_v^r n_c^{2r} \binom{L}{b} \left( \frac{D}{n_v} \right)^b = 3^r 10^r 16^{2r} \binom{3r}{b} \left( \frac{94}{10} \right)^b. \]
%
%\begin{table}[!ht]
%    \centering
%    \small
%    \begin{tabular}{|c|c|c|c||cr|}
%        \hline
%        \parbox{1.5cm}{Слогов, $r$} & \parbox{3cm}{Средняя длина пароля, $L$} & \parbox{3cm}{Символов из всего алфавита, $b$} & \parbox{3cm}{Верхняя оценка числа паролей $N$} & \multicolumn{2}{|c|}{\parbox{3.2cm}{Число всех паролей, $D^L$}} \\
%        \hline
%        \multirow{3}{*}{2} & \multirow{3}{*}{6} & $0$ & $2^{26}$ & $2^{28}$ & a-z \\
%        & & $1$ & $2^{32}$ & $2^{34}$ & A-Z, a-z \\
%        & & $3$ & $2^{40}$ & $2^{39}$ & Весь алфавит \\
%        \hline
%        \multirow{3}{*}{3} & \multirow{3}{*}{9} & $0$ & $2^{39}$ & $2^{42}$ & a-z \\
%        & & $2$ & $2^{50}$ & $2^{51}$ & A-Z, a-z \\
%        & & $4$ & $2^{59}$ & $2^{59}$ & Весь алфавит \\
%        \hline
%        \multirow{3}{*}{4} & \multirow{3}{*}{12} & $0$ & $2^{52}$ & $2^{56}$ & a-z \\
%        & & $3$ & $2^{69}$ & $2^{68}$ & A-Z, a-z \\
%        & & $6$ & $2^{81}$ & $2^{77}$ & Весь алфавит \\
%        \hline
%    \end{tabular}
%    \caption{Сравнение верхней оценки числа паролей, построенных на слогах, со всем доступным множеством паролей в алфавите из $D$ символов.}
%    \label{tab:password-number}
%\end{table}
%
%Из таблицы \ref{tab:password-number} видно, что при доступном объеме вычислений в $2^{60 \ldots 70}$ операций, пароли вплоть до 12 символов, построенные на словах, слогах, изменениях слов, вставках цифр, небольшого изменения регистров и другой простейшей обфускации, могут быть найдены перебором на кластере (или ПК) в настоящее время.


\subsubsection{Атака для подбора паролей и ключей шифрования}

В схемах аутентификации по паролю иногда используется хэширование и хранение хэша пароля на сервере. В таких случаях применима словарная атака или атака с применением заранее вычисленных таблиц для ускорения поиска.

Для нахождения пароля, прообраза хэш-функции, или для нахождения ключа блокового шифрования по атаке с выбранным шифротекстом (для одного и того же известного открытого текста и соответствующего шифротекста) может быть применен метод перебора с балансом между памятью и временем вычислений. Самый быстрый метод радужных таблиц (rainbow tables)\index{радужные таблицы}, 2003 г., заранее вычисляет следующие цепочки и хранит результат в памяти.

Для нахождения пароля, прообраза хэш-функции $H$, цепочка строится как
    \[ M_0 \xrightarrow{H(M_0)} h_0 \xrightarrow{R_0(h_0)} M_1 \ldots M_t \xrightarrow{H(M_t)} h_t \xrightarrow{R_t(h_t)} M_{t+1}, \]
где $R_i(h)$ -- функция редуцирования, преобразования хэша в пароль для следующего хэширования.

Для нахождения ключа блокового шифрования для одного и того же известного открытого текста $M$ таблица строится как
    \[ K_0 \xrightarrow{E_{K_0}(M)} c_0 \xrightarrow{R_0(c_0)} K_1 \ldots K_t \xrightarrow{E_{K_t}(M)} c_t \xrightarrow{R_t(c_t)} K_{t+1}, \]
где $R_i(c)$ -- функция редуцирования, преобразования шифротекста в новый ключ.

Функция редуцирования $R_i$ зависит от номера итерации, чтобы избежать дублирующиеся подцепочки, которые возникают в случае коллизий между значениями в разных цепочках в разных итерациях, если $R$ постоянна. Rainbow-таблица размера $(m \times 2)$ состоит из строк $(M_{0,j}, M_{t+1,j})$ или $(K_{0,j}, K_{t+1,j})$, вычисленных для разных значений стартовых паролей $M_{0,j}$ или $K_{0,j}$ соответственно.

Опишем атаку на примере нахождения прообраза $\overline{M}$ хэша $\overline{h} = H(\overline{M})$. На первой итерации исходный хэш $\overline{h}$ редуцируется в сообщение $\overline{h} \xrightarrow{R_t(\overline{h})} \overline{M}_{t+1} $ и сравнивается со всеми значениями последнего столбца $M_{t+1,j}$ таблицы. Если нет совпадения, переходим ко второй итерации. Хэш $\overline{h}$ дважды редуцируется в сообщение $\overline{h} \xrightarrow{R_{t-1}(\overline{h})} \overline{M}_t \xrightarrow{H(\overline{M}_t)} \overline{h}_t \xrightarrow{R_t(\overline{h}_t)} \overline{M}_{t+1}$ и сравнивается со всеми значениями последнего столбца $M_{t+1,j}$ таблицы. Если не совпало, то переходим к третьей итерации и т.д. Если для $r$-кратного редуцирования сообщение $\overline{M}_{t+1}$ содержится в таблице во втором столбце, то из совпавшей строки берется $M_{0,j}$, и вся цепочка пробегается заново для поиска искомого сообщения $\overline{M}: ~ \overline{h} = H(\overline{M})$.

Найдем вероятность нахождения пароля в таблице. Пусть мощность множества всех паролей $N$. Изначально в столбце $M_{0,j}$ содержится $m_0 = m$ различных паролей. Предполагая случайное отображение с пересечениями паролей $M_{0,j} \rightarrow M_{1,j}$, в $M_{1,j}$ будет $m_1$ различных паролей. Согласно задаче о размещении,
\[
    m_{i+1} = N \left( 1 - \left( 1 - \frac{1}{N} \right)^{m_i} \right) \approx N \left( 1 - e^{-\frac{m_i}{N}} \right).
\]
Вероятность нахождения пароля
\[
    P = 1 - \prod \limits_{i=1}^t \left( 1 - \frac{m_i}{N} \right).
\]

Чем больше таблица из $m$ строк, тем больше шансов найти пароль или ключ, выполнив в наихудшем случае   $O \left( m \frac{t(t+1)}{2} \right)$ операций.

Примеры применения атаки на хэш-функциях $\textrm{MD5}$\index{хэш-функция!MD5}, $\textrm{LM} \sim \textrm{DES}_{\textrm{Password}} (\textrm{const})$ приведены в табл. \ref{tab:rainbow-tables}.

\begin{table}[!ht]
    \centering
    \caption{Атаки на радужных таблицах на \emph{одном} ПК\label{tab:rainbow-tables}}
    \resizebox{\textwidth}{!}{ \begin{tabular}{|c|c|c|c|c|c|c|}
        \hline
        \multirow{2}{*}{\parbox{1.0cm}{Длина, биты}} & \multicolumn{3}{|c|}{Пароль или ключ} &
            \multicolumn{3}{|c|}{Радужная таблица} \\
        \cline{2-7}
        & \parbox{1.2cm}{Длина, симв.} & \parbox{1cm}{Множе- ство} & \parbox{1cm}{Мощн- ость} &
            Объем & \parbox{1.5cm}{Время вычисления таблиц} & \parbox{1.3cm}{Время поиска} \\
        \hline \hline
        \multicolumn{7}{|c|}{Хэш LM} \\
        \hline
        \multirow{3}{*}{$2 \times 56$} & \multirow{3}{*}{14} &
            A--Z & $2^{33}$ & 610 MB &  & 6 с \\
        & & A--Z, 0-9 & $2^{36}$ & 3 GB &  & 15 с \\
        & & все & $2^{43}$ & 64 GB & \parbox{1.5cm}{несколько лет} & 7 мин \\
        \hline \hline
        \multicolumn{7}{|c|}{Хэш MD5} \\
        \hline
        128 & 8 & a-z, 0-9 & $2^{41}$ & 36 GiB & - & 4 мин \\
        \hline
    \end{tabular} }
\end{table}


\section{Аутентификация по паролю}

Из-за малой энтропии пользовательских паролей во всех системах регистрации и аутентификации пользователей применяется специальная политика безопасности. Типичные минимальные требования:
\begin{enumerate}
    \item Длина пароля от 8 символов. Использование разных регистров и небуквенных символов в паролях. Запрет паролей из словаря слов или часто используемых паролей. Запрет паролей в виде дат, номеров машин и других номеров.
    \item Ограниченное время действия пароля. Обязательная смена пароля по истечении срока действия.
    \item Блокирование возможности аутентификации после нескольких неудачных попыток. Ограниченное число актов аутентификаций в единицу времени. Временная задержка перед выдачей результата аутентификации.
\end{enumerate}

Дополнительные рекомендации (требования) пользователям:
\begin{enumerate}
    \item Не использовать одинаковые или похожие пароли для разных систем. Например, электронная почта, вход в ОС, электронная платежная система, форумы, социальные сети. Пароль часто передается в открытом виде по сети. Пароль доступен администратору системы, возможны утечки конфиденциальной информации с серверов. Стараться выбирать случайные стойкие пароли.
    \item Не записывать пароли. Никому не сообщать пароль, даже администратору. Не передавать пароли по почте, телефону, Интернету и т.д.
    \item Не использовать одну и ту же учетную запись для разных пользователей даже в виде исключения.
    \item Всегда блокировать компьютер, когда пользователь отлучается от него даже на короткое время.
\end{enumerate}

\input{os_passwords}

\input{http_auth}

\chapter{Программные уязвимости}

\section[Контроль доступа в информационных системах]{Контроль доступа в \protect\\ информационных системах}
\selectlanguage{russian}

%http://www.acsac.org/2005/papers/Bell.pdf
%http://www.dranger.com/iwsec06_co.pdf
%http://csrc.nist.gov/groups/SNS/rbac/documents/design_implementation/Intro_role_based_access.htm
%http://en.wikipedia.org/wiki/Access_control#Computer_security
%http://en.wikipedia.org/wiki/Discretionary_access_control
%http://en.wikipedia.org/wiki/Mandatory_access_control
%http://en.wikipedia.org/wiki/Role-Based_Access_Control

В информационных системах контроль доступа вводится на \emph{действия} \emph{субъектов} над \emph{объектами}. В операционных системах под субъектами почти всегда понимаются процессы, под объектами -- процессы, разделяемая память, объекты файловой системы, порты ввода-вывода и т.д., под действием -- чтение (файла или содержимого директории), запись (создание, добавление, изменение, удаление, переименование файла или директории) и исполнение (файла-программы). Система контроля доступа в информационной системе (операционной системе, базе данных и т.д.) определяет множество субъектов, объектов и действий.

Применение контроля доступа создается 1) \emph{аутентификацией} субъектов и объектов, 2) \emph{авторизацией} допустимости действия, 3) \emph{аудитом} (проверкой и хранением) ранее совершенных действий.

Различают три основные модели контроля доступа -- дискреционная\index{управление доступом!дискреционное} (discretionary access control\index{access control!discretionary}, DAC\index{DAC}), мандатная\index{управление доступом!мандатное} (mandatory access control\index{access control!mandatory}, MAC\index{MAC}) и ролевая\index{управление доступом!ролевое} (role-based access control\index{access control!role-based}, RBAC\index{RBAC}) модели. Современные операционные системы используют \emph{комбинации} двух или трех моделей доступа, причем решения о доступе принимаются в порядке убывания приоритета: ролевая, мандатная, дискреционная модели.

Использование систем контроля доступа и защиты информации в операционных системах используется не только для защиты от злоумышленника, но и для повышения устойчивости системы в целом. Однако появление новых механизмов в новых версиях ОС может привести к проблемам совместимости с уже существующим программным обеспечением.

\subsection{Дискреционная модель}

Классическое определение дискреционной модели\index{контроль доступа!дискреционный} из так называемой Оранжевой книги 1985 г. (Trusted Computer System Evaluation Criteria, устаревший стандарт министерства обороны США 5200.28-STD) следующее -- средства ограничения доступа к объектам, основанные на сущности (identity) субъекта и/или группы, к которой они принадлежат. Субъект, имеющий определенный доступ к объекту, имеет возможность полностью или частично передать право доступа другому субъекту.

На практике дискреционная модель доступа предполагает, что для каждого объекта в системе определен субъект-владелец. Этот субъект может самостоятельно устанавливать необходимые, по его мнению, права доступа к любому из своих объектов для остальных субъектов, в том числе и для себя самого. Логически владелец объекта является владельцем информации, находящейся в этом объекте. При доступе некоторого субъекта к какому-либо объекту система контроля доступа лишь считывает установленные для объекта права доступа и сравнивает их с правами доступа субъекта. Кроме того, предполагается наличие в ОС некоторого выделенного субъекта -- администратора дискреционного управления доступом, который имеет привилегию устанавливать дискреционные права доступа для любых, даже ему не принадлежащих, объектов в системе.

Дискреционную модель реализуют почти все популярные ОС, в частности, Windows и Unix. У каждого субъекта (процесса пользователя или системы) и объекта (файла, другого процесса и т.д.) есть владелец, который может делегировать доступ другим субъектам, изменяя атрибуты на чтение, запись файлов для других пользователей и групп пользователей. Администратор системы имеет возможность назначить нового владельца и другие права доступа к объектам.


\subsection{Мандатная модель}

Классическое определение мандатной модели\index{контроль доступа!мандатный} из Оранжевой книги -- средства ограничения доступа к объектам, основанные на важности (секретности) информации, содержащейся в объектах, и обязательная авторизация действий субъектов для доступа к информации с присвоенным уровнем важности. Важность информации определяется уровнем доступа, приписываемым всем объектам и субъектам. Исторически мандатная модель определяла важность информации в виде иерархии, например, совершенно секретно (СС), секретно (С), конфиденциально (К) и рассекречено (Р). При этом верно следующее: СС $>$ C $>$ K $>$ P, то есть каждый уровень включает сам себя и все уровни, находящиеся ниже в иерархии.

Современное определение мандатной модели -- применение явно указанных правил доступа субъектов к объектам, определяемых только администратором системы. Сами субъекты (пользователи) не имеют возможности для изменения прав доступа. Правила доступа описаны матрицей, в которой столбцы соответствуют субъектам, строки -- объектам, а в ячейках -- допустимые действия субъекта над объектом. Матрица покрывает все пространство субъектов и объектов. Также определены правила наследования доступа для новых создаваемых объектов. В мандатной модели матрица может быть изменена только администратором системы.

Модель Белла --- ЛаПадулы\index{модель!Белла --- ЛаПадулы} (Bell --- LaPadula,~\cite{Bell:LaPadula:1973, Bell:LaPadula:1976}) использует два мандатных и одно дискреционное правила политики безопасности.
\begin{enumerate}
    \item Субъект с определенным уровнем секретности не может иметь доступ на \emph{чтение} объектов с более \emph{высоким} уровнем секретности (no read-up).
    \item Субъект с определенным уровнем секретности не может иметь доступ на \emph{запись} объектов с более \emph{низким} уровнем секретности (no write-down).
    \item Использование матрицы доступа субъектов к объектам для описания дискреционного доступа.
\end{enumerate}

\subsection{Ролевая модель}

Ролевая модель доступа основана на определении ролей в системе\index{контроль доступа!ролевой}. Понятие <<роль>> в этой модели -- это совокупность действий и обязанностей, связанных с определенным видом деятельности. Таким образом, достаточно указать тип доступа к объектам для определенной роли и определить группу субъектов, для которых она действует.
Одна и та же роль может использоваться несколькими различными субъектами (пользователями). В некоторых системах пользователю разрешается выполнять несколько ролей одновременно, в других есть ограничение на одну или несколько не противоречащих друг другу ролей в каждый момент времени.

Ролевая модель, в отличие от дискреционной и мандатной, позволяет реализовать разграничение полномочий пользователей, в частности, на системного администратора и офицера безопасности, что повышает защиту от человеческого фактора.


\section{Контроль доступа в ОС}
\selectlanguage{russian}

\subsection{Windows}
%http://www.gentlesecurity.com/blog/andr/cracking_windows_access_control.pdf
%http://msdn.microsoft.com/en-us/library/bb250462(VS.85).aspx#upm_ovwim
%http://msdn.microsoft.com/en-us/library/bb625963.aspx
%http://msdn.microsoft.com/en-us/library/bb625964.aspx

Операционные системы Windows до Windows Vista использовали только дискреционную модель безопасности. Владелец файла имел возможность изменить права доступа или разрешить доступ другому пользователю.

Начиная с Windows Vista, в дополнение к стандартной дискреционной модели субъекты и объекты стали обладать мандатным уровнем доступа, устанавливаемым администратором (или по умолчанию системой для новых созданных объектов) и имеющим приоритет над стандартным дискреционным доступом, который может менять владелец.

В Vista мандатный уровень доступа предназначен в большей степени для обеспечения \emph{целостности} и устойчивости системы, чем для обеспечения секретности.

Уровень доступа объекта (integrity level в терминологии Windows) помечается шестнадцатеричным числом в диапазоне от \texttt{0} до \texttt{0x4000}, большее число означает более высокий уровень доступа. В Vista определены 5 базовых уровней:
\begin{itemize}
    \item ненадежный (Untrusted, \texttt{0x0000}),
    \item низкий (Low Integrity, \texttt{0x1000}),
    \item средний (Medium Integrity, \texttt{0x2000}),
    \item высокий (High Integrity, \texttt{0x3000}) и
    \item системный (System Integrity, \texttt{0x4000}).
\end{itemize}

Дополнительно объекты имеют три атрибута, которые, если они установлены, запрещают доступ субъектов с более низким уровнем доступа к ним: cубъекты с более низким уровнем доступа не могут
\begin{itemize}
    \item читать (no read-up),
    \item изменять (no write-up),
    \item исполнять (no execute-up)
\end{itemize}
объекты с более высоким уровнем доступа. Для всех объектов по умолчанию установлен атрибут запрета записи объектов с более высоким уровнем доступа, чем имеет субъект (no write-up).

Субъекты имеют два атрибута:
\begin{itemize}
    \item запрет записи объектов с более высоким уровнем доступа, чем у субъекта (no write-up, эквивалентно аналогичному атрибуту объекта),
    \item установка уровня доступа созданного процесса-потомка как минимума от уровня доступа родительского процесса (субъекта) и исполняемого файла (объекта файловой системы).
\end{itemize}
Оба атрибута установлены по умолчанию.

Все пользовательские данные и процессы по умолчанию имеют средний уровень доступа, а системные файлы -- системный. Например, если в Internet Explorer, который в защищённом (protected) режиме запускается с низким уровнем доступа, обнаружится уязвимость, злоумышленник не будет иметь возможности изменить системные данные на диске, даже если браузер запущен администратором.

Уровень доступа процесса соответствует уровню доступа пользователя (процесса), который запустил процесс. Например, пользователи LocalSystem, LocalService, NetworkService получают системный уровень, администраторы -- высокий, обычные пользователи системы -- средний, остальные (everyone) -- низкий.

По каким-то причинам, вероятно, для целей совместимости с ранее разработанными программами и/или для упрощения разработки и настройки новых сторонних программ других производителей, субъекты с системным, высоким и средним уровнями доступа создают объекты или владеют объектами со \emph{средним} уровнем доступа. И только субъекты с низким уровнем доступа создают объекты с низким уровнем доступа. Это означает, что системный процесс может владеть файлом или создать файл со средним уровнем доступа, и другой процесс с более низким уровнем доступа, например средним, может получить доступ к файлу, в т.~ч. и на запись. Это нарушает принцип запрета записи в объекты, созданные субъектами с более высоким уровнем доступа.


\subsection{Linux}

Стандартная ОС Unix обеспечивает дискреционную модель контроля доступа на следующей основе.
\begin{itemize}
    \item Каждый субъект (процесс) и объект (файл) имеют владельца пользователя и группу, которые могут изменять доступ к данному объекту для себя и других пользователей и групп.
    \item Каждый объект (файл) имеет атрибуты доступа на чтение (r), запись (w) и исполнение (x) для трех типов пользователей: владельца-пользователя (u), владельца-группы (g), остальных пользователей (o) -- (u:rwx, g:rwx, o:rwx).
    \item Субъект может входить в несколько групп.
\end{itemize}

В 2000 г. Агентство Национальной Безопасности США (NSA) выпустило набор изменений SELinux с открытым исходным кодом к ядру ОС Linux версии 2.4. Начиная с версии ядра 2.6, SELinux входит как часть стандартного ядра. SELinux реализует комбинацию ролевой, мандатной и дискреционной моделей контроля доступа, которые могут быть изменены только администратором системы (и/или администратором безопасности). По сути, SELinux каждому субъекту приписывает одну или несколько ролей, и для каждой роли указано, к объектам с какими атрибутами они могут иметь доступ и какого вида.

Основная проблема ролевых систем контроля доступа -- очень большой список описания ролей и атрибутов объектов, что увеличивает сложность системы и приводит к регулярным ошибкам в таблицах описания контроля доступа.


\section{Виды программных уязвимостей}

\textbf{Вирусом} называется самовоспроизводящаяся часть кода (подпрограмма)\index{вирус}, которая встраивается в носители (другие программы) для своего исполнения и распространения. Вирус не может исполняться и передаваться без своего носителя.

\textbf{Червем} называется самовоспроизводящаяся отдельная (под)программа\index{червь}, которая может исполняться и распространяться самостоятельно, не используя программу-носитель.

Первой вехой в изучении компьютерных вирусов можно назвать 1949 год, когда Джон фон Нейман прочёл курс лекций в Университете Иллинойса под названием <<Теория самовоспроизводящихся машин>> (изданы в 1966~\cite{Neumann:1966}, переведены на русский язык издательством <<Мир>> в 1971 году~\cite{Neumann:1971}), в котором ввёл понятие самовоспроизводящихся механических машин. Первым сетевым вирусом считается вирус Creeper 1971 г., распространявшийся в сети ARPANET, предшественнике Интернета. Для его уничтожения был создан первый антивирус, Reaper, который находил и уничтожал Creeper.

Первый червь для Интернета, червь Морриса 1988 г., уже использовал \emph{смешанные} атаки для заражения UNIX машин~\cite{EichinRochlis:1988}\cite{Spafford:1989}. Сначала программа получала доступ к удалённому запуску команд, эксплуатируя уязвимости в сервисах \texttt{sendmail}, \texttt{finger} (с использованием атаки переполнением буфера) или \texttt{rsh}. Далее с помощью механизма подбора паролей червь получал доступ к локальным аккаунтам пользователей:
\begin{itemize}
    \item получение доступа к учётным записям с очевидными паролями:
		\begin{itemize}
			\item без пароля вообще;
			\item имя аккаунта в качестве пароля;
			\item имя аккаунта в качестве пароля, повторенное дважды;
			\item использование <<ника>> (англ. <<nickname>>);
			\item фамилия (англ. <<lastname>>);
			\item фамилия, записанная задом наперёд;
		\end{itemize}
		\item перебор паролей на основе встроенного словаря из 432 слов;
		\item перебор паролей на основе системного словаря \texttt{/usr/dict/words}.
\end{itemize}

\textbf{Программной уязвимостью}\index{программная уязвимость} называется свойство программы, позволяющее нарушить ее работу. Программные уязвимости могут приводить к отказу в обслуживании (Denial of Service, DoS-атака)\index{атака!отказ в обслуживании}, утечке и изменению данных, появлению и распространению вирусов и червей.

Одной из распространенных атак для заражения персональных компьютеров является переполнение буфера в стеке. В интернет-сервисах наиболее распространенной программной уязвимостью в настоящее время является межсайтовый скриптинг (Cross-Site Scripting, XSS-атака)\index{атака!XSS}.

Наиболее распространенные программные уязвимости можно разделить на классы:
\begin{enumerate}
    \item Переполнение буфера -- копирование в буфер данных большего размера, чем длина выделенного буфера. Буфером может быть контейнер текстовой строки, массив, динамически выделяемая память и т.д. Переполнение становится возможным вследствие либо отсутствия контроля над длиной копируемых данных, либо из-за ошибок в коде. Типичная ошибка -- разница в 1 байт между размерами буфера и данных при сравнении.
    \item Некорректная обработка (парсинг) данных, введенных пользователем, является причиной большинства программных уязвимостей в веб-приложениях. Под обработкой понимаются:
        \begin{enumerate}
            \item проверка на допустимые значения и тип (числовые поля не должны содержать строки и т.д.);
            \item фильтрация и экранирование специальных символов, имеющих значения в скриптовых языках или для декодирования из одной текстовой кодировки в другую. Примеры символов: \texttt{\textbackslash},  \texttt{\%}, \texttt{<}, \texttt{>}, \texttt{"},  \texttt{'};
            \item фильтрация ключевых слов языков разметки и скриптов. Примеры: \texttt{script}, \texttt{JavaScript};
            \item декодирование различными кодировками при парсинге. Распространенный способ обхода системы контроля парсинга данных состоит в однократном или множественном последовательном кодировании текстовых данных в шестнадцатеричные кодировки \texttt{\%NN} ASCII и UTF-8. Например, браузер или веб-приложения производят $n$ -- кратные последовательные декодирования, в то время как система контроля делает $k$-кратное декодирование, $0 \leq k < n$, и, следовательно, пропускает закодированные запрещенные символы и слова.
        \end{enumerate}
    \item Некорректное использование синтаксиса функций. Например, \texttt{printf(s)} может привести к уязвимости записи в указанный адрес памяти. Если злоумышленник вместо обычной текстовой строки введет в качестве \texttt{s = "текст некоторой длины\%n"}, то функция \texttt{printf()}, ожидающая первым аргументом строку формата \texttt{printf(fmt, \dots)}, обнаружив \texttt{\%n}, возьмет значения из ячеек памяти, следующих перед текстовой строкой (устройство стека функции описано далее), и запишет в адрес памяти, равный считанному значению, количество выведенных символов на печать функцией \texttt{printf()}.
\end{enumerate}


\section{Переполнение буфера в стеке}
\selectlanguage{russian}

В качестве примера переполнения буфера опишем самую распространенную атаку, направленную на исполнение кода злоумышленника.

В 64-битовой x86\_64 архитектуре основное пространство виртуальной памяти процесса из 16 эксабайтов ($2^{64}$ байт) свободно, и только малая часть занята (выделена). Виртуальная память выделяется процессу операционной системой блоками по 4 Кб, называемыми страницами памяти. Выделенные страницы соответствуют страницам физической оперативной памяти или страницам файлов.

Пример выделенной виртуальной памяти процесса представлен в табл.~\ref{tab:virtual-memory}. Локальные переменные функций хранятся в области памяти, называемой стеком.
\begin{table}[!ht]
    \centering
    \caption{Пример структуры виртуальной памяти процесса\label{tab:virtual-memory}}
    \resizebox{\textwidth}{!}{ \begin{tabular}{r|c|}
        \multicolumn{2}{c}{Адрес ~~~~~~~~~~~~~~ Использование} \\
        \cline{2-2}
        \texttt{0x00000000 00000000} & \\
        & \\
        \cdashline{2-2}
        \texttt{0x00000000 0040063F} & \multirow{2}{*}{\parbox{6cm}{Исполняемый код, динамические библиотеки}} \\
        & \\
        \cdashline{2-2}
        & \\
        & \\
        & \\
        \cdashline{2-2}
        \texttt{0x00000000 0143E010} & \multirow{2}{*}{Динамическая память} \\
        & \\
        \cdashline{2-2}
        & \\
        & \\
        & \\
        \cdashline{2-2}
        \texttt{0x00007FFF A425DF26} & \multirow{2}{*}{Переменные среды} \\
        & \\
        \cdashline{2-2}
        & \\
        & \\
        & \\
        \cdashline{2-2}
        \texttt{0x00007FFF FFFFEB60} & \multirow{2}{*}{Стек функций} \\
        & \\
        \cdashline{2-2}
        & \\
        & \\
        \texttt{0xFFFFFFFF FFFFFFFF} & \\
        \cline{2-2}
    \end{tabular} }
\end{table}

Приведем пример переполнения буфера в стеке\index{стек}, которое дает возможность исполнить код, для 64-разрядной ОС Linux. Ниже приводится листинг исходной программы, которая печатает расстояние Хэмминга между векторами $b1 = \text{\texttt{0x01234567}}$ и $b2 = \text{\texttt{0x89ABCDEF}}$.

\begin{verbatim}
#include <stdio.h>
#include <string.h>

int hamming_distance(unsigned a1, unsigned a2, char *text,
                     size_t textsize) {
  char buf[32];
  unsigned distance = 0;
  unsigned diff = a1 ^ a2;
  while (diff) {
    if (diff & 1) distance++;
    diff >>= 1;
  }
  memcpy(buf, text, textsize);
  printf("%s: %i\n", buf, distance);
  return distance;
}

int main() {
  char text[68] = "Hamming";
  unsigned b1 = 0x01234567;
  unsigned b2 = 0x89ABCDEF;
  return hamming_distance(b1, b2, text, 8);
}
\end{verbatim}

Вывод программы при запуске:
\begin{verbatim}
$ ./hamming
Hamming: 8
\end{verbatim}

При вызове вложенных функций вызывающая функция выделяет стековый кадр для вызываемой функции в сторону уменьшения адресов. Стековый кадр в порядке уменьшения адресов состоит из следующих частей.
\begin{enumerate}
    \item Аргументы вызова функции, расположенные в порядке уменьшения адреса (за исключением тех, которые передаются в регистрах процессора).
    \item Сохраненный регистр процессора \texttt{rip} внешней функции, также называемый адресом возврата. Регистр процессора \texttt{rip} содержит адрес следующей инструкции для исполнения. При входе во вложенную функцию адрес инструкции текущей функции запоминается в стеке, в регистре записывается новое значение адреса первой инструкции из вложенной функции, а по завершении функции регистр восстанавливается из стека, и, таким образом, исполнение возвращается назад.
    \item Сохраненный регистр процессора \texttt{rbp} внешней функции. Регистр процессора \texttt{rbp} содержит адрес сохраненного регистра \texttt{rbp} в стековом кадре вызывающей функции. Процессор обращается к локальным переменным функций по смещению относительно регистра \texttt{rbp}. При вызове вложенной функции регистр сохраняется в стеке, в регистр записывается текущее значение адреса стека (\texttt{rsp}), а по завершении функции регистр восстанавливается.
    \item Локальные переменные, как правило расположенные в порядке уменьшения адреса при объявлении новой переменной (порядок может быть изменён в результате оптимизаций и использования механизмов защиты, таких как Stack Smashing Protection в компиляторе GCC).
\end{enumerate}

Адрес начала стека, а также, возможно, адреса локальных массивов и переменных выровнены на границу параграфа в 16 байт, из-за чего в стеке могут образоваться неиспользуемые байты.

Если в программе есть ошибка, которая может привести к переполнению выделенного буфера в стеке при копировании, есть возможность записать вместо сохраненного значения регистра \texttt{rip} новое. В результате по завершении данной функции исполнение начнется с указанного адреса. Если есть возможность записать в переполняемый буфер исполняемый код, а затем на место сохраненного регистра \texttt{rip} адрес на этот код, то получим исполнение заданного кода в стеке функции.

На рис.~\ref{fig:stack-overflow} приведены исходный стек и стек с переполненным буфером, из-за которого записалось новое сохраненное значение \texttt{rip}.

\begin{figure}[!ht]
	\centering
	\includegraphics[width=0.95\textwidth]{pic/stack-overflow}
	\caption{Исходный стек и стек с переполнением буфера\label{fig:stack-overflow}}
\end{figure}


Изменим программу для демонстрации, поместив в копируемую строку исполняемый код для вызова \texttt{/bin/sh}.
{ \small
\begin{verbatim}
...
int main() {
  char text[68] =
    // 28 байт исполняемого кода
    "\x90" "\x90" "\x90"                // nop; nop; nop
    "\x48\x31" "\xD2"                   // xor %rdx, %rdx
    "\x48\x31" "\xF6"                   // xor %rsi, %rsi
    "\x48\xBF" "\xDC\xEA\xFF\xFF"
    "\xFF\x7F\x00\x00"                  // mov $0x7fffffffeadc,
                                        //   %rdi
    "\x48\xC7\xC0" "\x3B\x00\x00\x00"   // mov $0x3b, %rax
    "\x0F\x05"                          // syscall
    // 8 байт строки /bin/sh
    "\x2F\x62\x69\x6E\x2F\x73\x68\x00"  // "/bin/sh\0"
    // 12 байт заполнения и 16 байт новых
    // значений сохраненных регистров
    "\x00\x00\x00\x00"                  // незанятые байты
    "\x00\x00\x00\x00"                  // unsigned distance
    "\x00\x00\x00\x00"                  // unsigned diff
    "\x50\xEB\xFF\xFF"                  // регистр
    "\xFF\x7F\x00\x00"                  //   rbp=0x7fffffffeb50
    "\xC0\xEA\xFF\xFF"                  // регистр
    "\xFF\x7F\x00\x00"                  //   rip=0x7fffffffeac0
    ;
  ...
  return hamming_distance(b1, b2, text, 68);
  ...
}
\end{verbatim} }

Код эквивалентен вызову функции \texttt{execve(``/bin/sh'', 0 0)} через системный вызов функции ядра Linux для запуска оболочки среды \texttt{/bin/sh}. При системном вызове нужно записать в регистр \texttt{rax} номер системной функции, а в другие регистры процессора -- аргументы. Данный системный вызов с номером \texttt{0x3b} требует в качестве аргументов регистры \texttt{rdi} с адресом строки исполняемой программы, \texttt{rsi} и \texttt{rdx} с адресами строк параметров запускаемой программы и переменных среды. В примере в \texttt{rdi} записывается адрес \texttt{0x7fffffffeadc}, который указывает на строку \texttt{``/bin/sh''} в стеке после копирования. Регистры \texttt{rdx} и \texttt{rsi} обнуляются.

На рис.~\ref{fig:stack-overflow} приведен стек с переполненным буфером, в результате которого записалось новое сохраненное значение \texttt{rip}, указывающее на заданный код в стеке.

Начальные инструкции \texttt{nop} с кодом \texttt{0x90} означают пустые операции. Часто точные значения адреса и структуры стека неизвестны, поэтому злоумышленник угадывает предполагаемый адрес стека. В начале исполняемого кода создается массив из операций \texttt{nop} с надеждой, что предполагаемое значение стека, то есть требуемый адрес rip, попадет на эти операции, повысив шансы угадывания. Стандартная атака на переполнение буфера с исполнением кода также подразумевает последовательный перебор предполагаемых адресов для нахождения правильного адреса для \texttt{rip}.

В результате переполнения буфера в примере по завершении функции \texttt{hamming\_distance()} начнет исполняться инструкция с адреса строки \texttt{buf}, то есть заданный код.


\subsection{Защита}

Самый лучший способ защиты от атак переполнения буфера -- создание программного кода со слежением за размером данных и длиной буфера. Однако ошибки все равно происходят. Существует несколько стандартных способов защиты от исполнения кода в стеке в архитектуре x86 (x86\_64).

\begin{enumerate}
	\item Современные 64-разрядные x86\_64 процессоры включают поддержку флаги доступа к страницам памяти. В таблице виртуальной памяти, выделенной процессу, каждая страница имеет набор флагов, отвечающих за защиту страниц от некорректных действий программы.
	\begin{itemize}
		\item Флаг разрешения доступа из пользовательского режима. Если флаг не установлен, то доступ к данной области памяти возможен только из режима ядра.
		\item Флаг запрета записи. Если флаг установлен, то попытка выполнить запись в данную область памяти приведёт к возникновению исключения.
		\item Флаг запрета исполнения\index{бит запрета исполнения} (NX-Bit, No eXecute Bit в терминологии AMD; XD-Bit, Execute Disable Bit в терминологии Intel; DEP, Data Executuion Prevention -- соответствующая опция защиты в операционных системах). Если флаг установлен, при попытке передачи управления на данную область памяти возникнет исключение. Для совместимости со старым программным обеспечением есть возможность отключить использование данного флага на уровне операционной системы целиком или для отдельных программ.
	\end{itemize}
	Попытка выполнить операции, которые запрещены соответствующими настройками виртуальной памяти, вызывает ошибку сегментации (segmentation fault, segfault).

    \item Второй стандартный способ -- вставка проверочных символов (называемых canaries, guards) после массивов и в конце стека и их проверка перед выходом из функции. Если произошло переполнение буфера, программа аварийно завершится. Данный способ защиты реализован с помощью модификации конечного кода программы во время компиляции\footnote{см. опции \texttt{-fstack-protector} для GCC, \texttt{/GS} для компиляторов от Microsoft и другие}, его нельзя включить или отключить без перекомпиляции программного обеспечения.

    \item Третий способ -- рандомизация адресного пространства (address space layout randomization, ASLR), то есть случайное расположение стека, кода и т.~д. В настоящее время используется в большинстве современных операционных систем (Android, iOS, Linux, OpenBSD, OS X, Windows). Это приводит к маловероятному угадыванию адресов и значительно усложняет использование уязвимости.
\end{enumerate}


\subsection{Другие атаки с переполнением буфера}

Почти любую возможность для переполнения буфера в стеке или динамической памяти можно использовать для получения критической ошибки в программе из-за обращения к адресам виртуальной памяти, страницы которых не были выделены процессу. Следовательно, можно проводить атаки отказа в обслуживании (Denial of Service (DoS) атаки).

Переполнение буфера в динамической памяти в случае хранения в ней адресов для вызова функций может привести к подмене адресов и исполнению другого кода.

В описанных DoS-атаках NX-бит не защищает систему.


\input{xss}

\input{sql-injections}

%\chapter{Послесловие}
%Это должно быть что-то в виде заключения, объяснения, почему именно эти темы выбраны, насколько актуален материал с теоретической и практической точки зрения.


\appendix

\chapter{Математическое приложение}\label{chap:discrete-math}

\section{Общие определения}

Выражением $\mod n$ обозначается вычисление остатка от деления произвольного целого числа на целое число $n$. В полиномиальной арифметике эта операция означает вычисление остатка от деления многочленов.
%далее будем обозначать целые числа или операции с целыми числами, взятыми \textbf{по модулю}\index{модуль} целого числа $n$ (остаток от целочисленного деления). Примеры выражений:
    \[ a\mod n, \]
    \[ (a + b) c\mod n. \]
Равенство
    \[ a = b \mod n \]
означает, что выражения $a$ и $b$ равны (говорят также <<сравнимы>>, <<эквивалентны>>) по модулю $n$.

Множество
    \[ \{ 0, 1, 2, 3,  \dots,  n-1 \mod n\} \]
состоит из $n$ элементов, где каждый элемент $i$ представляет все целые числа, сравнимые с $i$ по модулю $n$.

Наибольший общий делитель (НОД) двух чисел $a,b$ обозначается $\gcd(a,b)$ (greatest common divisor).

Два числа $a,b$ называются взаимно простыми, если они не имеют общих делителей, $\gcd(a,b) = 1$.

Выражение $a \mid b$ означает, что $a$ делит $b$.

\input{birthdays_paradox}

\section{Группы}\label{section-groups}
\selectlanguage{russian}

\subsection{Свойства групп}

\textbf{Группой}\index{группа} называется множество $\Gr$, на котором задана бинарная операция <<$\cdot$>>, удовлетворяющая следующим аксиомам:
\begin{enumerate}
    \item замкнутость
        \[ \forall a,b \in \Gr: a \cdot b = c \in \Gr; \]
    \item ассоциативность
        \[ \forall a,b,c \in \Gr: (a \cdot b) \cdot c = a \cdot (b \cdot c); \]
    \item существование единичного элемента
        \[ \exists ~ e \in \Gr: e\cdot a = a \cdot e = a; \]
    \item существование обратного элемента
        \[ \forall a \in \Gr ~ \exists ~ b \in \Gr: a \cdot b = b \cdot a = e. \]
\end{enumerate}
Если
    \[ \forall a,b \in \Gr: a \cdot b = b \cdot a, \]
то группа коммутативная.

Если операция в группе задана как умножение $\cdot$, то группа называется \textbf{мультипликативной}, $e = 1$, обратный элемент -- $a^{-1}$, возведение в степень $k$ -- $a^k$.

Если операция задана как сложение $+$, то группа называется \textbf{аддитивной}, $e = 0$, обратный элемент $-a$, сложение $k$ раз -- $ka$.

Подмножество группы, удовлетворяющее аксиомам группы, называется \textbf{подгруппой}\index{подгруппа}.

\textbf{Порядком} $|\Gr|$ \textbf{группы}\index{порядок группы} $\Gr$ называется число элементов в группе. Пусть группа мультипликативная. Для любого элемента $a \in \Gr$ выполняется $a^{|\Gr|} = 1$.

\textbf{Порядком элемента} $a$ называется минимальное число
    \[ ord(a): a^{ord(a)} = 1. \]
 Порядок элемента делит порядок группы:
    \[ ord(a) \mid \left|\Gr\right|. \]


\subsection{Циклические группы}

\textbf{Генератором} $g \in \Gr$ называется элемент, \emph{порождающий} всю группу\index{генератор группы}
    \[ \Gr = \{g, g^2, g^3,  \ldots,  g^{|\Gr|} = 1\}. \]
Группа, в которой существует генератор, называется \textbf{циклической}\index{группа!циклическая}.

Если конечная группа не циклическая, то в ней существуют циклические подгруппы, порожденные всеми элементами. Любой элемент $a$ группы порождает либо циклическую \emph{подгруппу}
    \[ \{ a, a^2, a^3,  \dots,  a^{ord(a)} = 1 \} \]
порядка $ord(a)$, если порядок элемента $ord(a) < |\Gr|$, либо \emph{всю} группу
    \[ \Gr = \{ a, a^2, a^3,  \dots,  a^{|\Gr|} = 1 \}, \]
если порядок элемента равен порядку группы $ord(a) = |\Gr|$. Порядок любой подгруппы, как и порядок элемента, делит порядок всей группы.

Представим циклическую группу через генератор $g$ как
    \[ \Gr = \{g, g^2,  \ldots,  g^{|\Gr|} = 1\} \]
и каждый элемент $g^i$  возведем в степени $1, 2,  \ldots,  |\Gr|$. Тогда
\begin{itemize}
    \item элементы $g^i$, для которых число $i$ взаимно просто с $|\Gr|$, породят снова всю группу
            \[ \Gr = \{ g^i, g^{2i}, g^{3i},  \dots,  g^{|\Gr| i} = 1 \}, \]
        так как степени $\{i, 2i, 3i, \dots |\Gr| i \}$ по модулю $|\Gr|$ образуют перестановку чисел $\{1, 2, 3, \dots, |\Gr|\}$; следовательно, $g^i$ -- тоже генератор, число таких чисел $i$ по определению функции Эйлера $\varphi(|\Gr|)$ ($\varphi(n)$ -- количество взаимно простых с $n$ целых чисел в диапазоне $[1,n-1]$);
    \item элементы $g^i$, для которых $i$ имеют общие делители
            \[ d_i = \gcd(i, |\Gr|) \neq 1 \]
        c $|\Gr|$, породят подгруппы
            \[ \{ g^i, g^{2i}, g^{3i},  \dots,  g^{\frac{i}{d_i} |\Gr|} = 1\}, \]
        так как степень последнего элемента кратна $|\Gr|$; следовательно, такие $g^i$ образуют циклические подгруппы порядка $d_i$.
\end{itemize}
%TODO Гашков, Болотов, Часовских "Эллиптическая криптография" или "Методы элл. кри-ии"

Из предыдущего утверждения следует, что число генераторов в циклической группе равно
    \[ \varphi(|\Gr|). \]

Для проверки, является ли элемент генератором всей группы, требуется знать разложение порядка группы $|\Gr|$ на множители. Далее элемент возводится в степени, равные всем делителям порядка группы, и сравнивается с единичным элементом $e$. Если ни одна из степеней не равна $e$, то этот элемент является примитивным элементом, или генератором группы. В противном случае элемент будет генератором какой-либо подгруппы.

Задача разложения числа на множители является трудной для вычисления. На сложности ее решения, например, основана криптосистема RSA\index{криптосистема!RSA}. Поэтому при создании больших групп желательно заранее знать разложение порядка группы на множители для возможности выбора генератора.


\subsection{Группа $\Z_p^*$}\label{section-group-multiplicative}

\textbf{Группой $\Z_p^*$} называется группа\index{группа!$\Z_p^*$}
    \[ \Z_p^* = \{1, 2,  \dots,  p-1 \mod p\}, \]
где $p$ -- простое\index{число!простое} число, операция в группе -- умножение $\ast$ по $\mod p$.

Группа $\Z_p^*$ -- \textbf{циклическая}, порядок
    \[ |\Z_p^*| = \varphi(p) = p - 1. \]
Число генераторов в группе --
    \[ \varphi(|\Z_p^*|) = \varphi(p-1). \]

Из того, что $\Z_p^*$ -- группа, для простого\index{число!простое} $p$ и любого $a \in [2, p-1] \mod p$ следует \textbf{малая теорема Ферма}\index{теорема!Ферма малая}:
    \[ a^{p-1} = 1 \mod p. \]
На малой теореме Ферма основаны многие тесты проверки числа на простоту.

\example
Рассмотрим группу $\Z_{19}^*$. Порядок группы -- 18. Делители: 2, 3, 6, 9. Является ли 12 генератором?
\[ \begin{array}{l}
    12^2 = -8 \mod 19, \\
    12^3 = -1 \mod 19, \\
    12^6 = 1 \mod 19, \\
\end{array} \]
12 -- генератор подгруппы 6 порядка. Является ли 13 генератором?
\[ \begin{array}{l}
    13^2 = -2 \mod 19, \\
    13^3 = -7 \mod 19, \\
    13^6 = -8 \mod 19, \\
    13^9 = -1 \mod 19, \\
    13^{18} = 1 \mod 19, \\
\end{array} \]
13 -- генератор всей группы.
\exampleend

\example
В таб.~\ref{tab:Zp-sample} приведен пример группы $\Z_{13}^*$. Число генераторов -- $\varphi(12) = 4$. Подгруппы --
    \[ \Gr^{(1)}, \Gr^{(2)}, \Gr^{(3)}, \Gr^{(4)}, \Gr^{(6)}; \]
верхний индекс обозначает порядок подгруппы.

\begin{table}[!ht]
    \centering
    \caption {Генераторы и циклические подгруппы группы $\Gr=\Z_{13}^*$\label{tab:Zp-sample}}
    \resizebox{\textwidth}{!}{ \begin{tabular}{|c|p{0.66\textwidth}|c|}
        \hline
        Элемент & Порождаемая группа или подгруппа & Порядок \\
        \hline
        1 & $\Gr^{(1)} = \{ 1 \}$ & 1 \\
        2 & $\Gr = \{ 2, 4,  8 = -5, -10 = 3, 6, 12 = -1, -2, -4, -8 = 5, 10 = -3, -6, -12 = 1 \}$ & 12, ген. \\
        3 & $\Gr^{(3)} = \{ 3, 9 = -4, -12 = 1 \}$ & 3 \\
        4 & $\Gr^{(6)} = \{ 4, 16 = 3, 12 = -1, -4, -3, -12 = 1 \}$ & 6 \\
        5 & $\Gr^{(4)} = \{ 5, 25 = -1, -5, 1 \}$ & 4 \\
        6 & $\Gr = \{6, 36 = -3, -5, -4, 2, -1, -6, 3, 5, 4, -2, -12 = 1 \}$ & 12, ген. \\
        7 = -6 & $\Gr = \{ -6, 36 = -3, 5, -4, -2, -1, 6, 3, -5, 4, 2, -12 = 1 \}$ & 12, ген. \\
        8 = -5 & $\Gr^{(4)} = \{ -5, 25 = -1, 5, 1 \}$ & 4 \\
        9 = -4 & $\Gr^{(3)} = \{ -4, 16 = 3, -12 = 1 \}$ & 3 \\
        10 = -3 & $\Gr^{(6)} = \{ -3, 9 = -4, 12 = -1, 3, -9 = 4, -12 = 1 \}$ & 6 \\
        11 = -2 & $\Gr = \{ -2, 4, 5, 3, -6, -1, 2, -4, -5, -3, 6, -12 = 1 \}$ & 12, ген. \\
        12 = -1 & $\Gr^{(2)} = \{ -1, 1 \}$ & 2 \\
        \hline
    \end{tabular} }
\end{table}
\exampleend


\subsection{Группа $\Z_n^*$}

\textbf{Функция Эйлера}\index{функция!Эйлера} $\varphi(n)$ определяется как количество чисел, взаимно простых с $n$ , в интервале от 1 до $n-1$.

Если $n=p$ -- простое\index{число!простое} число, то
    \[ \varphi(p) = p - 1, \]
    \[ \varphi(p^k) = p^k - p^{k-1} = p^{k-1}(p - 1). \]
Если $n$ -- составное число и
    \[ n = \prod \limits_{i} p_i^{k_i} \]
разложено на простые множители $p_i$, то
    \[ \varphi(n) = \prod \limits_{i} \varphi(p_i^{k_i}) =  \prod \limits_{i} p_i^{k_i - 1}(p_i - 1). \]

\textbf{Группой $\Z_n^*$} называется группа\index{группа!$\Z_n^*$}
    \[ \Z_n^* = \left\{ \forall a \in \left\{ 1, 2,  \dots,  n-1 \mod n \right\} : \gcd(a,n) = 1 \right\}, \]
с операцией умножения $\ast$ по $\mod n$.

Порядок группы --
    \[ |\Z_n^*| = \varphi(n). \]
Группа $\Z_p^*$ -- частный случай группы $\Z_n^*$.

Если $n$ \emph{составное}\index{число!составное} (не простое) число, то группа $\Z_n^*$ \textbf{нециклическая}.

Из того, что $\Z_n^*$ -- группа, для любых $a \neq 0, n > 1: \gcd(a,n) = 1$ следует \textbf{теорема Эйлера}\index{теорема!Эйлера}:
    \[ a^{\varphi(n)} = 1 \mod n. \]

При возведении в степень, если $\gcd(a,n) = 1$, выполняется
    \[ a^b = a^{b \mod \varphi(n)} \mod n. \]

\example
В табл.~\ref{tab:Zn-sample} приведена нециклическая группа $\Z_{21}^*$ и ее циклические подгруппы
    \[ \Gr^{(1)}, \Gr_1^{(2)}, \Gr_2^{(2)}, \Gr_3^{(2)}, \Gr_1^{(3)}, \Gr_1^{(6)}, \Gr_2^{(6)}, \Gr_3^{(6)}, \]
верхний индекс обозначает порядок подгруппы, нижний индекс нумерует различные подгруппы одного порядка.

\begin{table}[!ht]
    \centering
    \caption{Циклические подгруппы нециклической группы $\Z_{21}^*$\label{tab:Zn-sample}}
    \begin{tabular}{|c|l|c|}
        \hline
        Элемент & Порождаемая циклическая подгруппа & Порядок \\
        \hline
        1  & $\Gr^{(1)} = \{ 1 \}$ & 1 \\
        2  & $\Gr_1^{(6)} = \{ 2, 4, 8, 16, 11, 1 \}$ & 6 \\
        4  & $\Gr_1^{(3)} = \{ 4, 16, 1 \}$ & 3 \\
        5  & $\Gr_2^{(6)} = \{ 5, 4, 20, 16, 17, 1 \}$ & 6 \\
        8  & $\Gr_1^{(2)} = \{ 8, 1 \}$ & 2 \\
        10 & $\Gr_3^{(6)} = \{ 10, 16, 13, 4, 19, 1 \}$ & 6 \\
        11 & $\Gr_1^{(6)} = \{ 11, 16, 8, 4, 2, 1 \}$ & 6 \\
        13 & $\Gr_2^{(2)} = \{ 13, 1 \}$ & 2 \\
        16 & $\Gr_1^{(3)} = \{ 16, 4, 1 \}$ & 3 \\
        17 & $\Gr_2^{(6)} = \{ 17, 16, 20, 4, 5, 1 \}$ & 6 \\
        19 & $\Gr_3^{(6)} = \{ 19, 4, 13, 16, 10, 1 \}$ & 6 \\
        20 & $\Gr_3^{(2)} = \{ 20, 1 \}$ & 2 \\
        \hline
    \end{tabular}
\end{table}
\exampleend

\subsection{Конечные поля}

\textbf{Полем} называется множество $\F$, для которого\index{поле}:
\begin{itemize}
    \item заданы две бинарные операции, условно называемые операциями умножения <<$\cdot$>> и сложения <<$+$>>;
    \item выполняются аксиомы группы для операции <<сложения>>: \\
        1. замкнутость:
		\[\forall a, b \in \F: a + b \in \F;\]
        2. ассоциативность:
		\[\forall a, b, c \in \F: (a+b)+c = a+(b+c);\]
        3. существование нейтрального элемента по сложению (часто обозначаемого как <<0>>):
		\[\exists 0 \in \F: \forall a \in \F: a + 0 = 0 + a = a; \]
        4. существование обратного элемента:
		\[\forall a \in \F: \exists -a: a + (-a) = 0; \]
    \item выполняются аксиомы группы для операции <<умножения>>, за одним исключением: \\
        1. замкнутость:
		\[\forall a, b \in \F: a \cdot b \in \F; \]
        2. ассоциативность:
		\[\forall a, b, c \in \F: (a \cdot b) \cdot c = a \cdot (b \cdot c);\]
        3. существование нейтрального элемента по умножению (часто обозначаемого как <<1>>):
		\[\exists 1 \in \F: \forall a \in \F: a \cdot 1 = 1 \cdot a = a;\]
        3. существование обратного элемента по умножению для всех элементов множества, кроме нейтрального элемента по сложению:
		\[\forall a \in {\F \backslash 0}: \exists a^{-1}: a \cdot a^{-1} = a^{-1} \cdot a = 1;\]
    \item операции <<сложения>> и <<умножения>> коммутативны
        \[ \begin{array}{l}
            \forall a, b \in \F: a + b = b + a, \\
            \forall a, b \in \F: a \cdot b = b \cdot a; \\
        \end{array} \]
    \item выполняется свойство дистрибутивности
        \[ \forall a, b, c \in \F: a \cdot (b + c) = (a \cdot b) + (a \cdot c). \]
\end{itemize}

Примеры \emph{бесконечных} полей (с бесконечным числом элементов) -- поле рациональных чисел $\group{Q}$, поле вещественных чисел $\group{R}$, поле комплексных чисел $\group{C}$ с обычными операциями сложения и умножения.

В криптографии рассматриваются \emph{конечные} поля (с конечным числом элементов), называемые также \textbf{полями Галуа}.

Число элементов в любом конечном поле равно $p^n$, где $p$ -- простое\index{число!простое} число и $n$ -- натуральное число. Обозначения поля Галуа: $\GF{p}, \GF{p^n}, \F_p, \F_{p^n}$ (аббревиатура от Galois field). Все поля Галуа $\GF{p^n}$ изоморфны друг другу (существует взаимно однозначное отображение между полями, сохраняющее действие всех операций). Другими словами, существует только одно поле Галуа $\GF{p^n}$ для фиксированных $p, n$.

Приведем примеры конечных полей.

Двоичное поле $\GF{2}$ состоит из двух элементов. Однако задать его можно разными способами:
\begin{itemize}
	\item Как множество из двух чисел <<0>> и <<1>> с определёнными на нём операциями <<сложение>> и <<умножение>> как сложение и умножение чисел по модулю 2. Нейтральным элементом по сложению будет <<0>>, по умножению -- <<1>>:
\[\begin{array}{ll}
	0 + 0 = 0,	& 	0 \cdot 0 = 0, \\
	0 + 1 = 1,	& 	0 \cdot 1 = 0, \\
	1 + 0 = 1,	& 	1 \cdot 0 = 0, \\
	1 + 1 = 0,	& 	1 \cdot 1 = 1. \\
\end{array}\]
	\item Как множество из двух логических объектов <<ЛОЖЬ>> ($F$) и <<ИСТИНА>> ($T$) с определёнными на нём операциями <<сложение>> и <<умножение>> как булевые операции <<исключающее или>> и <<и>> соответственно. Нейтральным элементом по сложению будет <<ЛОЖЬ>>, по умножению -- <<ИСТИНА>>:
\[\begin{array}{ll}
	F + F = F,	& 	F \cdot F = F, \\
	F + T = T,	& 	F \cdot T = F, \\
	T + F = T,	& 	T \cdot F = F, \\
	T + T = F,	& 	T \cdot T = T. \\
\end{array}\]
	\item Как множество из двух логических объектов <<ЛОЖЬ>> ($F$) и <<ИСТИНА>> ($T$) с определёнными на нём операциями <<сложение>> и <<умножение>> как булевые операции <<эквивалентность>> и <<или>> соответственно. Нейтральным элементом по сложению будет <<ИСТИНА>>, по умножению -- <<ЛОЖЬ>>:
\[\begin{array}{ll}
	F + F = T,	& 	F \cdot F = F, \\
	F + T = F,	& 	F \cdot T = T, \\
	T + F = F,	& 	T \cdot F = T, \\
	T + T = T,	& 	T \cdot T = T. \\
\end{array}\]
	\item Как множество из двух чисел <<0>> и <<1>> с определёнными на нём операциями <<сложение>> и <<умножение>>, заданными в табличном представлении. Нейтральным элементом по сложению будет <<1>>, по умножению -- <<0>>:
\[\begin{array}{ll}
	0 + 0 = 1,	& 	0 \cdot 0 = 0, \\
	0 + 1 = 0,	& 	0 \cdot 1 = 1, \\
	1 + 0 = 0,	& 	1 \cdot 0 = 1, \\
	1 + 1 = 1,	& 	1 \cdot 1 = 1. \\
\end{array}\]
\end{itemize}

Все перечисленные выше варианты множеств изоморфны друг другу. Поэтому в дальнейшем под конечным полем $\GF{p}$, где $p$ -- простое\index{число!простое} число, будем понимать поле, заданное как множество целых чисел от $0$ до $p-1$ включительно, на котором операции <<сложение>> и <<умножение>> заданы как операции сложения и умножения чисел по модулю числа $p$. Например, поле $\GF{7}$ будем считать состоящим из 7-и чисел $\{0, 1, 2, 3, 4, 5, 6\}$ с операциями умножения $(\cdot \mod 7)$ и сложения $(+ \mod 7)$ по модулю.

Конечное поле $\GF{p^n}, n > 1$ строится \textbf{расширением} \emph{базового} поля $\GF{p}$. Элемент поля представляется как многочлен степени $n-1$ (или меньше) с коэффициентами из базового поля $\GF{p}$:
    \[ \alpha = \sum\limits_{i=0}^{n-1} a_i x^i, ~ a_i \in \GF{p}. \]

Операция сложения элементов в таком поле традиционно задаётся как операция сложения коэффициентов при одинаковых степенях в базовом поле $\GF{p}$. Операция умножения -- как умножение многочленов со сложением и умножением коэффициентов в базовом поле $\GF{p}$ и дальнейшим приведением результата по модулю некоторого заданного (для поля) неприводимого\footnote{Многочлен называется \textbf{неприводимым}\index{многочлен!неприводимый}, если он не раскладывается на множители, и \textbf{приводимым}\index{многочлен!приводимый}, если раскладывается.} многочлена $m(x)$. Количество элементов в поле равно $p^n$.

Многочлен $g(x)$ называется \textbf{примитивным элементом}\index{многочлен!примитивный} (генератором) поля, если его степени порождают все ненулевые элементы, т.~е. $\GF{p^n} \setminus \{0\}$, заданное неприводимым многочленом $m(x)$, за исключением нуля:
    \[ \GF{p^n} \setminus \{0\} = \{ g(x), g^2(x), g^3(x), \dots, g^{p^n-1}(x) = 1 \mod m(x) \}. \]

Неприводимый многочлен $\mod m(x)$ называется  \textbf{примитивным}\index{многочлен!примитивный}, если $g(x)=x$.

\example
В табл.~\ref{tab:irreducible-gf2} приведены примеры многочленов \emph{над полем} $\GF{2}$.
\begin{table}[!ht]
    \centering
    \caption{Пример многочленов над полем $\GF{2}$\label{tab:irreducible-gf2}}
    \begin{tabular}{|c|c|c|}
        \hline
        Многочлен & \parbox{2.5cm}{Упрощенная запись} & Разложение \\
        \hline
        $'1' x + '0'$ & $x$ & неприводимый \\
        $'1' x + '1'$ & $x+1$ & неприводимый \\
        $'1' x^2 + '0' x + '0'$ & $x^2$ & $x \cdot x$ \\
        $'1' x^2 + '0'x + '1'$ & $x^2 + 1$ & $(x+1) \cdot (x+1)$ \\
        $'1' x^2 + '1' x + '0'$ & $x^2 + x$ & $x \cdot (x+1)$ \\
        $'1' x^2 + '1' x + '1'$ & $x^2 + x + 1$ & неприводимый \\
        $'1' x^3 + '0' x^2 + '0' x + '1'$ & $x^3 + 1$ & $(x+1) \cdot (x^2+x+1)$ \\
        \hline
    \end{tabular}
\end{table}
\exampleend


\input{aes_math}

\section{Модульная арифметика}
\selectlanguage{russian}

\subsection{Сложность модульных операций}

Криптосистемы с открытым ключом, как правило, построены в модульной арифметике с длиной модуля от сотни до нескольких тысяч разрядов. Сложность алгоритмов оценивают как количество битовых операций в зависимости от длины. В табл. \ref{tab:mod-binary-complexity} приведены оценки (с точностью до порядка) сложности модульных операций\index{битная сложность} для простых (или "школьных") алгоритмов вычислений. На самом деле, для реализации арифметики длинных чисел (сотни или тысячи двоичных разрядов) следует применять существенно более эффективные (более "хитрые") алгоритмы вычислений, использующие, например, специальный вид быстрого преобразования Фурье и другие приемы.

\begin{table}[!ht]
    \centering
    \caption{Битная сложность операций по модулю $n$ длиной $k= \log n$ бит\label{tab:mod-binary-complexity}}
    \begin{tabular}{| p{0.7\textwidth} | c |}
        \hline
        Операция, алгоритм & Сложность \\
        \hline
        1. $a \pm b \mod n$ & $O(k)$ \\
        2. $a \cdot b \mod n$ & $O(k^2)$ \\
        3. $\gcd(a, b)$, алгоритм Евклида & $O(k^2)$ \\
        4. $(a,b) \rightarrow (x,y,d) : ax + by = d = \gcd(a,b)$, расширенный алгоритм Евклида & $O(k^2)$ \\
        5. $a^{-1} \mod n$, расширенный алгоритм Евклида & $O(k^2)$ \\
        6. Китайская теорема об остатках & $O(k^2)$ \\
        7. $a^b \mod n$ & $O(k^3)$ \\
        \hline
    \end{tabular}

\end{table}


\subsection{Возведение в степень по модулю}

Метод называется <<возводи в квадрат и перемножай>>. Найдем $a^b \mod n$.
    \[ b = \sum_{i=0}^{k-1} b_i 2^i, \]
    \[ a^b = a^{\sum\limits_{i=0}^{k-1} b_i 2^i} = \prod_{i=0}^{k-1} (a^{{2^i} b_i} \mod n) \mod n. \]
Последовательно вычисляем квадраты
    \[ a_0 = a, ~ a_1 = a_0^2 \mod n, ~ a_2 = a_1^2 \mod n,  \ldots  \]
по модулю $n$ и перемножаем $a_i$, которым соответствует $b_i = 1$. Число возведений в квадрат равно $k-1$ (если $b_{k-1} =1$), а число умножений меньше или равно $k-1$. Возведение в квадрат и умножение можно считать операцией с квадратичной битной сложностью $O(k^2)$. Поэтому общая битовая сложность возведения в степень -- кубическая
    \[ O(2(k-1)k^2) = O(k^3). \]

\example
\[ \begin{array}{l}
    8^{24} \mod 25 = 8^8 \cdot 8^{16} \mod 25, \\
    8^2 = 14, \\
    8^4 = -4, \\
    8^8 = 16, \\
    8^{16} = 6, \\
    8^{24} = 16 \cdot 6 = -4 \mod 25.
\end{array} \]
\exampleend


\subsection{Алгоритм Евклида}\index{алгоритм!Евклида}
\selectlanguage{russian}

Рекурсивная форма алгоритма Евклида вычисления $\gcd(a,b)$ имеет следующий вид:
    \[\set(a,b): a>b;  \gcd(a,b) = \gcd(b, a \mod b). \]
Редуцирование чисел продолжается, пока не получим
    \[ a \mod b = 0, \]
тогда $b$ и будет искомым НОД.

\example
Вычислим $\gcd(56, 35)$:
\[ \begin{array}{ll}
    \gcd(56, 35) & =~ \gcd(35, ~ 56 \mod 35 = 21) ~= \\
    & =~ \gcd(21, ~ 35 \mod 21 = 14) ~= \\
    & =~ \gcd(14, ~ 21 \mod 14 = 7) ~= \\
    & =~ \gcd(7, ~ 14 \mod 7 = 0) ~= \\
    & =~ 7. \\
\end{array} \]
\exampleend


\subsection{Расширенный алгоритм Евклида}\index{алгоритм!Евклида!расширенный}

\textbf{Расширенный алгоритм Евклида} (см. например~\cite[8.8 Наибольшие общие делители и алгоритм Евклида]{Aho:1979}) для целых $a$ и $b$ ($a > b$) находит
    \[ x, y, d = \gcd(a,b): ax + by = d. \]

Введем обозначения: $g_i$ -- частное от деления, $r_i$ -- остаток от деления на $i$-ом шаге. Алгоритм:

\[\begin{array}{ll}
	r_{-1} & := a, \\
	r_0 & := b, \\
	y_0 & := x_{-1} := 1, \\
	y_{-1} & := x_0 := 0. \\
\end{array}\]

\[\begin{array}{ll}
	g_i & := \left\lfloor r_{i-2} / r_{i-1} \right\rfloor, \\
	r_i & := r_{i-2} - g_i \times r_{i-1}, \\
	y_i & := y_{i-2} - g_i \times y_{i-1} , \\
	x_i & := x_{i-2} - g_i \times x_{i-1} . \\
\end{array}\]

Алгоритм останавливается, когда $r_i = 0$.

%Вычисление осуществляется точно так же, как и в обычном алгоритме Евклида, только на каждой итерации дополнительно находится частное и остаток от деления.

\example
В табл.~\ref{tab:extended-euclid} приведен числовой пример алгоритма для $a=136, b=36$.
\begin{table}[!ht]
    \centering
    \caption{Пример расширенного алгоритма Евклида для \\ $a=136, b=36$\label{tab:extended-euclid}}
    \begin{tabular}{|r|r|r|r|r|rrr|}
        \hline
        $i$ & $g_i$ & $r_i$ & $x_i$ & $y_i$ & & & \\
        \hline
        $-1$ &  --- & $136$ &   $1$ &   $0$ & $136 =$ & $ 1 \cdot 136$ & $ + 0 \cdot 36$ \\
	 $0$ &  --- &  $36$ &   $0$ &   $1$ &  $36 =$ & $ 0 \cdot 136$ & $ + 1 \cdot 36$ \\
	 $1$ &  $3$ &  $28$ &  $+1$ &  $-3$ &  $28 =$ & $+1 \cdot 136$ & $ - 3 \cdot 36$ \\
	 $2$ &  $1$ &   $8$ &  $-1$ &  $+4$ &   $8 =$ & $-1 \cdot 136$ & $ + 4 \cdot 36$ \\
	 $3$ &  $3$ &   $4$ &  $+5$ & $-19$ &  $-4 =$ & $+5 \cdot 136$ & $- 19 \cdot 36$ \\
	 $4$ &  $2$ &   $0$ &   --- &   --- & & & --- \\
        \hline
    \end{tabular}
\end{table}
Найдено $x = 5, ~ y = -19, ~ d = 4$.
\exampleend

\subsection[Нахождение мультипликативного обратного]{Нахождение мультипликативного \protect\\ обратного по модулю}

Расширенный алгоритм Евклида можно использовать для вычисления обратного элемента\index{обратный элемент} -- для заданных $a, n$ найти $x, y, d = \gcd(a,n): ax + ny = d$. Пусть $a,n$ -- взаимно простые, тогда
\[\begin{array}{l}
	ax + ny = 1, \\
	ax \equiv 1 \mod n, \\
	x \equiv a^{-1} \mod n. \\
\end{array}\]

\example
В табл.~\ref{tab:extended-euclid-inverse} приведен числовой пример вычисления расширенным алгоритмом Евклида для $a=142, b=33$ обратных элементов $33^{-1} \equiv -43 \mod 142$ и $142^{-1} \equiv 10 \mod 33$.

\begin{table}[!ht]
    \centering
    \caption{Пример вычисления обратных элементов $33^{-1} \equiv -43 \mod 142$ и $142^{-1} \equiv 10 \mod 33$ из уравнения $142 x + 33 y = 1$ расширенным алгоритмом Эвклида\label{tab:extended-euclid-inverse}}
    \begin{tabular}{|r|r|r|r|r|rrr|}
        \hline
        $i$ & $g_i$ & $r_i$ & $x_i$ & $y_i$ & & & \\
        \hline
        $-1$ &  --- & $142$ &   $1$ &   $0$ & $142 =$ & $  1 \cdot 142$ & $ + 0 \cdot 33$ \\
	 $0$ &  --- &  $33$ &   $0$ &   $1$ &  $33 =$ & $  0 \cdot 142$ & $ + 1 \cdot 33$ \\
	 $1$ &  $4$ &  $10$ &  $+1$ &  $-4$ &  $10 =$ & $ +1 \cdot 142$ & $ - 4 \cdot 33$ \\
	 $2$ &  $3$ &   $3$ &  $-3$ & $+13$ &   $3 =$ & $ -3 \cdot 142$ & $+ 13 \cdot 33$ \\
	 $3$ &  $3$ &   $1$ & $+10$ & $-43$ &   $1 =$ & $+10 \cdot 142$ & $- 43 \cdot 33$ \\
	 $4$ &  $3$ &   $0$ &   --- &   --- & & & --- \\
        \hline
    \end{tabular}
\end{table}
\exampleend

Для $k$-битового $n$ битовая сложность вычисления обратного элемента имеет порядок $O(k^2)$. Если известно разложение числа $n$ на множители, то по теореме Эйлера
    \[ a^{-1} = a^{\varphi(n) - 1} \mod n \]
и вычисление обратного элемента реализуется с битовой сложностью $O(k^3),~ k = \lceil \log_2 n \rceil$. Сложность вычислений по этому алгоритму можно уменьшить, если известно разложение на сомножители числа $\varphi(n) - 1$.


\input{chinese_remainder_theorem}

\section{Псевдопростые числа}

\subsection{Оценка числа простых чисел}

Функция $\pi(n)$ определяется как количество простых чисел из диапазона $[2, n]$.
Существует предел~\cite{Selberg:1949}
    \[ \lim\limits_{n \rightarrow \infty}\frac{ \pi(n)}{ \frac{n}{\ln n}}=1. \]

Для $n \geq 17$ верно неравенство $\pi(n) > \frac{n}{\ln n}$.

Идея создания простых чисел состоит в случайном выборе числа и тестировании его на простоту.

Вероятность $P_k$ того, что случайное $k$-битовое число $n$ будет простым, равна
    \[ \lim\limits_{k \rightarrow \infty} P_k = \frac{1}{\ln n} = \frac{1}{k \ln 2}. \]

\example
    Вероятность того, что случайное 500-битовое число (включая четные числа) будет простым, примерно равна $\frac{1}{347}$, вероятность простоты случайного 2000-битового числа примерно равна $\frac{1}{1836}$.
\exampleend

\input{fermas_test}

\input{miller-rabins_test}

\input{aks}

\input{pseudo-primes_generation}

\input{groups_of_ec_points_over_finite_fields}

\section[Полиномиальные и экспоненциальные алгоритмы]{Полиномиальные и \\ экспоненциальные алгоритмы}

Данный раздел поясняет обоснованность стойкости криптосистем с открытым ключом и имеет лишь косвенное отношение к дискретной математике.

Машина Тьюринга (МТ) (модель, представляющая любой вычислительный алгоритм) состоит из следующих частей:
\begin{itemize}
    \item неограниченной ленты, разделенной на клетки; в каждой клетке содержится символ из конечного алфавита, содержащего пустой символ blank; если символ ранее не был записан на ленту, то он считается blank;
    \item печатающей головки, которая может считать, записать символ $a_i$ и передвинуть ленту на 1 клетку влево-вправо $d_k$;
    \item конечной таблицы действий
    \[ (q_i, a_j) \rightarrow (q_{i1}, a_{j1}, d_k), \]
где $q$ -- состояние машины.
\end{itemize}

Если таблица переходов однозначна, то машина Тьюринга\index{машина Тьюринга} называется детерминированной. \textbf{Детерминированная} машина Тьюринга может \emph{имитировать} любую существующую детерминированную ЭВМ. Если таблица переходов не однозначна, то есть $(q_i, a_j)$ может переходить по нескольким правилам, то машина \textbf{недетерминированная}. \emph{Квантовый компьютер} является примером недетерминированной машины Тьюринга.

Класс задач $\set{P}$ -- задачи, которые могут быть решены за \emph{полиномиальное} время\index{задача!полиномиальная} на \emph{детерминированной} машине Тьюринга. Пример полиномиальной сложности (количество битовых операций)
    \[ O(k^{\textrm{const}}), \]
где $k$ -- длина входных параметров алгоритма. Операция возведения в степень в модульной арифметике $a^b \mod n$ имеет кубическую сложность $O(k^3)$, где $k$ -- двоичная длина чисел $a,b,n$.

Класс задач $\set{NP}$ -- обобщение класса $\set{P} \subseteq \set{NP}$, включает задачи, которые могут быть решены за \emph{полиномиальное} время на \emph{недетерминированной} машине Тьюринга. Пример сложности задач из $\set{NP}$ -- экспоненциальная сложность\index{задача!экспоненциальная}
    \[ O(\textrm{const}^k). \]
Описанный алгоритм Гельфонда (в разделе криптостойкости системы Эль-Гамаля\index{криптосистема!Эль-Гамаля}) решения задачи дискретного логарифма по нахождению $x$ для заданных $g \mod p$ и $a = g^x \mod p$ имеет сложность $O(e^{k/2})$, где $k$ -- двоичная длина чисел.

В криптографии полиномиальные $\set{P}$ алгоритмы считаются \emph{легкими и вычислимыми} на ЭВМ, которые являются детерминированными машинами Тьюринга. Неполиномиальные (экспоненциальные) $\set{NP}$ алгоритмы считаются \emph{трудными и невычислимыми} на ЭВМ, так как из-за экспоненциального роста сложности всегда можно выбрать такой параметр $k$, что время вычисления станет сравнимым с возрастом Вселенной.

Задача факторизации числа, задача дискретного логарифмирования в группе считаются $\set{NP}$-задачами.

Класс $\set{NP}$-полных задач -- подмножество задач из $\set{NP}$, для которых не известен полиномиальный алгоритм для детерминированной машины Тьюринга, и все задачи могут быть сведены друг к другу за полиномиальное время на \emph{детерминированной} машине Тьюринга. Например, задача об укладке рюкзака является $\set{NP}$-полной.

Стойкость криптосистем с \emph{открытым} ключом, как правило, основана на $\set{NP}$ или $\set{NP}$-полных задачах:
\begin{enumerate}
    \item RSA\index{криптосистема!RSA} -- $\set{NP}$-задача факторизации (строго говоря, на трудности извлечения корня степени $e$ по модулю $n$).
    \item Криптосистемы типа Эль-Гамаля\index{криптосистема!Эль-Гамаля} -- $\set{NP}$-задача дискретного логарифмирования.
\end{enumerate}

\emph{Нерешенной} проблемой является доказательство неравенства
    \[ \set{P} \neq \set{NP}. \]
Именно на гипотезе о том, что для для некоторых задач не существует полиномиальных алгоритмов, и основана стойкость криптосистем с открытым ключом.

\input{coincide-index_method}

%\chapter{Задачи и упражнения}
%
%К \textbf{примерам 1, 2, 3} \textbf{упражнение 1}. Указать способ расшифрования для легального получателя шифротекста и указать способ дешифрования для криптоаналитика, не знающего ключа.
%
%\textbf{Упражнение 2}. Пусть $M_1, M_2, M_3,\ldots,M_s$ -- набор перестановок. Показать, что существует единственная перестановка $M=M_1,M_2,M_3,\ldots M_s$.
%
%\textbf{Упражнение 3}. Вскрыть одиночную ячейку Фейстеля. Для этого задать конкретную функцию $F(K,R)$ и по конкретным значениям $L_{1}$ и $R_{1}$ найти $K$.
%
%\textbf{Упражнение 4}. Разделим последовательность на блоки, каждый из которых содержит 2 бита.
%
%\[\begin{array}{cc} {z_{1} } & {z_{2} } \end{array}|\begin{array}{cc} {z_{3} } & {z_{4} } \end{array}| \ldots |\begin{array}{cc} {z_{N} } & {z_{N+1} } \end{array}\]
%Блок может принимать значения $z_{1} z_{2} =\begin{array}{c} {11} \\ {10} \\ {01} \\ {00} \end{array}$
%Преобразуем последовательность символов:
% если $z_{1} z_{2} =11$ или $z_{1} z_{2} =00$, то пара выбрасывается;
%если $[z_{1} z_{2} =10$, то записываем новый символ $u=1$; если
%$z_{1} z_{2} =01$, то записываем новый символ $u=0$.
%Получаем новую двоичную последовательность.
%
%Показать, что вероятностное распределение символов в новой последовательности является равномерным.
%
%\textbf{Упражнение 5}.Предположим, что криптоаналитик знает, что период генерируемой $M$ -последовательности равен $T=2^{L} -1$. Пусть ему известна часть последовательности длины, меньшей периода: $T_{1}<2^{L} -1$.
% При каком значении $T_{1}$  криптоаналитик может найти многочлен обратной связи.
%
%\textbf{Упражнение 6}. Ответить на вопрос: <<Как подделать ЭП, не зная закрытого ключа?>>
%
%\textbf{Упражнение 7}. При помощи формул Виета найти дискриминант многочлена, представляющего эллиптическую кривую.

\printindex

\chapter*{Литература}
\addcontentsline{toc}{chapter}{Литература}
\begingroup
\renewcommand{\chapter}[2]{}%
%\bibliographystyle{ugost2008s}
%\bibliography{bibliography}
\printbibliography
\endgroup

\end{document}
